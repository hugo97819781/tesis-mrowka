%Formato e idioma
    \documentclass[letterpaper,DIV=12,12pt]{scrbook}
    \usepackage[spanish,mexico,shorthands=off,es-lcroman]{babel}
%Utilidades
    \usepackage{array}
    \usepackage[x11names]{xcolor}
    \usepackage{lipsum}
%Matemáticos
    \usepackage{amsmath}
    \usepackage{amsthm}
    \usepackage{amssymb} % eliminar si se usa unicode-math
    \usepackage{mathrsfs} % agregar después de fuente unicode-math, si se usa
%Etiquetas de las enumeraciones
    \usepackage[shortlabels]{enumitem}
    \setenumerate[1]{label=\MakeLowercase{\roman*}), ref=\roman*}
    \setenumerate[2]{label=\MakeLowercase{\alph*}), ref=\alph*}
%Cajas de Teoremas
    \usepackage{thmtools}
    \usepackage[framemethod=TikZ]{mdframed}
    \newtheorem{definicion}{Definición}[section]
    \newtheorem{proposicion}[definicion]{Proposición}
    \newtheorem{lema}[definicion]{Lema}
    \newtheorem{corolario}[definicion]{Corolario}
    \newtheorem{observacion}[definicion]{Observación}
    \newtheorem{ejemplo}[definicion]{Ejemplo}
    \newtheorem{consideracion}[definicion]{Consideración}
    \newtheorem{teorema}[definicion]{Teorema}
%Comandos Creados
	%C O M A N D O S
%GENERALES	
	\newcommand{\tq}{\text{ $|$ }}
	\newcommand{\midcup}{\mbox{$\bigcup$}\,}
	\newcommand{\midcap}{\mbox{$\bigcap$}\,}
	\newcommand{\ms}[1]{\mathscr{#1}}
%CONJUNTOS POR COMPRENSIÓN
	\providecommand\given{}
	\newcommand\SetSymbol[1][]{\nonscript\:#1\vert\allowbreak\nonscript\:\mathopen{}}
	\DeclarePairedDelimiterX\Set[1]\{\}{\renewcommand\given{\SetSymbol[\delimsize]}#1}
	\DeclarePairedDelimiterX\Par[1]\langle\rangle{\renewcommand\given{\SetSymbol[\delimsize]}#1}
	\DeclarePairedDelimiterX\Bra[1](){#1}
%RENOVACIÓN DE COMANDOS
	\renewcommand{\emptyset}{\varnothing}
	\renewcommand{\tau}{\ms{T}}
	%Nuevo "Setminus"
	\newcommand{\mysetminusD}{\hbox{\tikz{\draw[line width=0.6pt,line cap=round] (3pt,0) -- (0,6pt);}}}
	\newcommand{\mysetminusT}{\mysetminusD}
	\newcommand{\mysetminusS}{\hbox{\tikz{\draw[line width=0.45pt,line cap=round] (2pt,0) -- (0,4pt);}}}
	\newcommand{\mysetminusSS}{\hbox{\tikz{\draw[line width=0.4pt,line cap=round] (1.5pt,0) -- (0,3pt);}}}
	\newcommand{\mysetminus}{\mathbin{\mathchoice{\mysetminusD}{\mysetminusT}{\mysetminusS}{\mysetminusSS}}}
	\renewcommand{\setminus}{\mysetminus}
%OPERADORES
	\DeclareMathOperator{\inte}{int}
	\DeclareMathOperator{\ext}{ext}
	\DeclareMathOperator{\cla}{cl}
	\DeclareMathOperator{\der}{der}
	\DeclareMathOperator{\fron}{fr}
	\DeclareMathOperator{\scl}{sqcl}
	\DeclareMathOperator{\cf}{cf}
	\DeclareMathOperator{\Id}{Id}
	\DeclareMathOperator{\ima}{ima}
	\DeclareMathOperator{\dom}{dom}
	\DeclareMathOperator{\St}{St}
	\DeclareMathOperator{\Osq}{O_{sq}}
%TEXTOS
	\newcommand{\T}{\text{\textsf{\small T}}}
	\newcommand{\AN}{\text{\textsf{\small AN}}}
	\newcommand{\OR}{\text{\textsf{\small ON}}}
	\newcommand{\CAR}{\text{\textsf{\small CAR}}}
	\newcommand{\zfc}{\text{\textsf{\small ZFC}}}
	\newcommand{\zf}{\text{\textsf{\small ZF}}}
	\newcommand{\HC}{\text{\textsf{\small HC}}}
	\newcommand{\Ma}{\text{\textsf{\small MA}}}
	\newcommand{\Ac}{\text{\textsf{\small AC}}}
	\newcommand{\Pm}{\text{\textsf{\small MC}}}
	\newcommand{\Pdm}{\text{\textsf{\small WMC}}}
	\newcommand{\Ad}{\operatorname{\text{\textsf{\small AD}}}\,}
	\newcommand{\Mad}{\operatorname{\text{\textsf{\small MAD}}}\,}
%COCIENTE
	%\newcommand{\quot}[2]{{\raisebox{.2em}{$#1$}\left/\raisebox{-.2em}{$#2$}\right.}}

%Exclusivo del minimal
    \renewcommand{\ref}[1]{??}
    \renewcommand{\pageref}[1]{??}
    \newcommand{\autoref}[1]{??}
    \renewcommand{\cite}[2][]{??}
    \renewcommand{\index}[2][]{}
    \newcommand{\hyperref}[2][]{#2}
    \providecommand{\texorpdfstring}[2]{#1}

\begin{document}
    \frontmatter
        \pagestyle{empty}
        \cleardoublepage
    \vspace*{3cm}
    \begin{flushright}
        \textit{Una frase bien cool acá}. \\
        - Don Juan.
    \end{flushright}
\cleardoublepage
        \chapter*{Agradecimientos}
        \lipsum[1-2]
        \tableofcontents
        \chapter*{Introducción}
        \addcontentsline{toc}{chapter}{Introducción}
    \mainmatter
        \lipsum[3-8]
        \setcounter{chapter}{-1}
\chapter{Preeliminares}
%\pagestyle{chapstyle}

    \emph{\small En este capítulo se establecen las bases conceptuales y la notación que se utilizarán a lo largo de este trabajo. Se asume que el lector posee un conocimiento fundamental de la teoría de conjuntos axiomática y de la topología general, todo al nivel de cursos estándar de licenciatura. El propósito de este capítulo no es ser un tratado exhaustivo, sino fijar la terminología, los convenios y los resultados clásicos que se darán por sentados. Para una revisión más profunda, se remite al lector a textos de referencia como:}

    \section{Teoría  de Conjuntos}
    \subsection{Notación y convenciones básicas}

    \index[sym]{$\zfc$}\index[sym]{$\Ac$}
    Sea adopatará como marco axiomático a la teoría usual de conjuntos; $\zfc$. Se comprenden, por tanto, los axiomas de: existencia, extensionalidad, buena fundación, esquema de separación, par, unión, infinito, esquema de reemplazo y el axioma de elección (\textit{denotado a partir de ahora por $\Ac$}); mismos que pueden consultarse en \cite[p.~xv]{kunenSet}.

    Se asume que el lector está familiarizado con los objetos clásicos de la teoría de conjuntos, conviniendo las notaciones pertinentes a: los símbolos lógicos $\forall$, $\exists$, $\neg$, $\lor$, $\land$, $\rightarrow$, $\leftrightarrow$ y $\exists!$ para existencia y unicidad; el conjunto vacío $\emptyset$; la pertenencia $\in$, la contención $\subseteq$ y contención propia $\subsetneq$; la diferencia de conjuntos $X \setminus Y$; el par ordenado $(x,y)$, el conjunto potencia $\mathscr{P}(X)$; y claro, las operaciones conjuntistas: unión, intersección, producto cartesiano ($\cup$, $\cap$ y $\times$; junto con sus homólogos unarios: $\bigcup$, $\bigcap$ y $\prod$, respectivamente). A lo largo del presente texto se jerarquizarán las operaciones anteriores de la siguiente manera: se aplicarán siempre de izquierda a derecha, priorizando la diferencia de conjuntos, el producto cartesiano y la unión e intersección, en tal orden.

    \index[alph]{clase}\index[alph]{clase!propia}\index[alph]{clase!conjunto}
    Dado un conjunto $A$, se denotará por $\{x \in A \tq \varphi (x)\}$ al conjunto de todos los elementos $x$ de $A$ que satisfacen la fórmula $\varphi (x)$ (siendo tal colección un conjunto, debido al esquema de separación \cite[p.~xv]{kunenSet}). Una \textit{clase} es una ``coleccion'' del estilo $\mathcal{C} = \{x \tq \varphi (x)\}$, se dice $\mathcal{C}$ \textit{es conjunto} si y sólo si se satisface:
    $$\exists y \forall x \; ( x \in y \leftrightarrow \varphi(x)  ) $$
    en caso contrario, ésta se denomina \textit{clase propia}. Como abuso de notación, si un conjunto $x$ hace verdadera la fórmula $\varphi(x)$, se escribirá $x \in \mathcal{C}$. Se denotará por $\mathcal{V}$ a la clase $\{x \tq x=x\}$.
    
    \index[sym]{$\mapsto$}
    Se dará por sentado el conocimiento de la teoría elemental de relaciones y funciones, manteniéndose al margen de las notaciones típicas para: el dominio $\dom(f)$ e imagen $\ima(f)$; la imagen directa $f[A]$ e inversa $f^{-1}[A]$ y las funciones identidad $\Id_X$. La composición de funciones (o relaciones) será denotada por yuxtaposición $fg$ y, la restricción de una función (o relación) $f$ a un subconjunto $A \subseteq \dom(f)$, por $f \upharpoonright A$. Se señala además el uso ocasional de la expresion ``\textit{$A \to B$ dada por $x \mapsto f(x)$}'' (o simplemente ``$x \mapsto f(x)$'') para hacer referencia a la relga de correspondencia de la función $f:A \to B$, en caso su nombre carezca de interés.

    \subsection{Órdenes parciales}

    Los órdenes parciales reflexivos y antirreflexivos serán denotados por los símbolos $\leq$ y $<$, respectivamente y el término \textit{orden parcial} hará referencia a cualquiera de ellos; la posible diferencia no es sustancial, pues ambas versiones son fácilmente intercambiables al añadir o eliminar la identidad del conjunto sobre el cual se definen. Un \textit{conjunto parcialmente ordenado} se concive como un par $(P, R)$, donde $R$ es un orden parcial en $P$. En lo que sigue, fiíese conjunto parcialmente ordenado $(P, \leq)$.
    
    Para cada $A \subseteq P$: $\min(A)$, $\max(A)$, $\sup(A)$ e $\inf(A)$ denotarán el máximo, mínimo, supremo e ínfimo de $A$, respectivamente (en caso de existir). Además, cierto $p \in P$ es \textit{$R$-minimal} de $A$ si $p \in A$ y no existe $q \in A$ tal que $q < p$, definiendo el concepto \textit{$R$-maximal} de forma dual.

    \index[alph]{elementos!comparables}\index[alph]{elementos!incomparables}\index[alph]{elementos!compatibles}\index[alph]{elementos!incompatibles}\index[alph]{cadena}\index[alph]{anticadena}\index[sym]{$p \parallel q$}\index[sym]{$p \perp q$}
    Se conviene que dos elementos $p,q \in P$ son \textit{comparables} si y sólo si $p \leq q$ o $q \leq p$; en caso contrario, son \textit{incomparables}. Así mismo, $p$ y $q$ serán \textit{compatibles} ($p \parallel q$) cuando exista $r \in P$ de modo que $r \leq p$ y $r \leq q$; en caso contrario, serán \textit{incompatibles} ($p \perp q$). Una $(P,\leq)$-\textit{cadena} (\textit{anticadena, respectivamente}) es un subconjunto de $P$ de elementos comparables (incompatibles, respectivamente) dos a dos; y cuando el contexto lo permita, se omitirá el prefijo $(P,\leq)$.

    La caracterización típica para $\Ac$ es clave:

    \begin{teorema}[Principio de Maximalidad de Hausdorff]\label{teo-PMH}\index[alph]{Principio!de Maximalidad de Hausdorff}\index[alph]{Hausdorff!Principio de Maximalidad de}
        $\Ac$ se satsiface si y sólo si todo conjunto parcialmente ordenado $(P, \leq)$, no vacío, posee una $(P,\leq)$-cadena $\subseteq$-maximal (del conjunto de cadenas de $P$).
    \end{teorema}

    \index[alph]{orden!total}\index[alph]{orden!bueno}\index[alph]{orden!bien fundado}\index[alph]{orden!completo}
    Se dice que $\leq$ ($(P,\leq)$ o $(P,<)$, indistintamente) es: \textit{total} si cualesquiera dos elementos de $P$ son comparables, \textit{buen orden} (\textit{bien fundado, o completo, respectivamente}) si y sólo si cada $A \in \ms{P}(P) \setminus \{\emptyset\}$ tiene elemento mínimo (minimal, o supremo si $A$ es acotado superiormente, respectivamente). Nótese que todo buen orden es total, bien fundado y completo.

    \index[alph]{isomorfismo(de orden)}\index[alph]{morfismo (de orden)}\index[alph]{orden!isomorfo a otro}\index[alph]{función!creciente}\index[alph]{función!decreciente}\index[sym]{$(P,<) \cong (Q,\sqsubset)$}
    Dados dos ordenes parciales $(P,R)$ y $(Q,S)$, se dice que una función $f:P \to Q$ es: \textit{$S$-creciente} (\textit{decreciente}, respectivamente) si y sólo si dados $p,q \in P$, se tiene que $p \mathrel{R} q$ implica $f(p) \mathrel{S} f(q)$ (o $f(p) \mathrel{S} f(q)$, respectivamente). En cualquier caso, se dice que $f$ es un \textit{morfismo de orden}; y, si además $f$ es biyectiva, se dice que $f$ es un \textit{isomorfismo} y que los órdenes $(P,R)$ y $(Q,S)$ son \textit{isomorfos}, denotado $(P,R) \cong (Q,S)$.

    \subsection{Ordinales y Cardinales}

    \index[alph]{ordinal}\index[alph]{natural}\index[sym]{$\OR$}
    Siguiendo la hoy conocida como construccion de John von Neumann, se declara que un conjunto $\alpha$ es: \textit{ordinal} si es transitivo (esto es, $\alpha \subseteq \ms{P}(\alpha)$) y $(\alpha,\in)$ es un buen orden; y, \text{natural} si es un ordinal tal que $(\alpha,\ni)$ es un buen orden. Se denota por $\OR$ a la clase (propia) de todos los ordinales.

    Los ordinales se denotan, típicamente, por las primeras letras griegas minúsculas: $\alpha, \beta, \gamma$, etcétera; y, los naturales por: $m,n,k$, etcétera. Se seguirá esta convención, salvo que se indique lo contrario.

    \index[alph]{ordinal!cero}\index[alph]{ordinal!sucesor}\index[alph]{ordinal!límite}\index[sym]{$\omega$}\index[sym]{$0$ (cero)}\index[sym]{$\alpha < \beta$}\index[sym]{$\alpha+1$}
    Si $\alpha$ y $\beta$ son ordinales, se conviene que $\alpha$ es menor que $\beta$ ($\alpha<\beta$) cuando $\alpha \in \beta$; en este sentido, es un hecho que toda clase no vacía de ordinales, $X$, tiene un mínimo (a saber, $\midcap X$). Y consecuentemente, todo conjunto de ordinales $A$ tiene supremo (a saber, $\midcup A$). Un ordinal $\alpha$ es: \textit{cero} si $\alpha=0:=\emptyset$; \textit{sucesor} cuando existe otro ordinal $\beta$ de modo que $\alpha = \beta \cup \{\beta\}$ (en cuyo caso se denota $\alpha=\beta+1$); y \textit{límite} en caso no ocurra ninguna de las dos anteriores. El primer ordinal límite se denotará por $\omega$. Es un hecho que $\omega$ es el conjunto de todos los números naturales.
    
    \begin{teorema}[Inducción transfinita]\index[alph]{Teorema!de Inducción transfinita}\index[alph]{transfinita!Inducción}
        Si $\varphi(x)$ es una fórmula de la teoría de conjuntos y:
        \begin{enumerate}
            \item $\varphi(0)$ se satisface.
            \item Para cada ordinal $\alpha$, la satisfacción de $\varphi(\alpha)$ implica la satisfacción de $\varphi(\alpha+1)$.
            \item Para cada ordinal límite $\gamma$, la satisfacción de $\forall \alpha \in \gamma (\varphi(\alpha))$ implica la satisfacción de $\varphi(\gamma)$.
        \end{enumerate}
        entonces, para cualquier ordinal $\alpha$, se satisface $\varphi(\alpha)$.

        Se obtiene la misma conclusión sustituyendo las condiciones (i)-(iii) por el enunciado: Para todo ordinal $\gamma$, la satisfacción de $\forall \alpha \in \gamma (\varphi(\alpha))$ implica la satisfacción de $\varphi(\gamma)$.
    \end{teorema}

    \index[alph]{funcional}\index[sym]{$F:\mathcal{C} \to \mathcal{C}'$}
    Dadas clases $\mathcal{C}=\{x \tq \varphi(x)\}$ y $\mathcal{C}'=\{x \tq \varphi'(x)\}$, se dice que un \textit{funcional de} $\mathcal{C}$ en $V$ es una clase $F$ de pares ordenados; a saber $F=\{(x,y) \tq \varphi(x) \land \psi(x,y)\}$, de forma que $\forall x (\varphi(x)\to \exists! y (\varphi'(y) \land \psi(x,y))) $. En cuyo caso, se denota $F:\mathcal{C} \to \mathcal{C}'$, y, para cada $x$ en $\mathcal{C}$, se denota por $F(x)$ al único $y$ en $\mathcal{C}'$ tal que $\psi(x,y)$. Siendo claro además que, si $A$ es un conjunto cualquiera, $F[A]=\{F(a) \tq a \in A\}$.

    \begin{teorema}[Recursión transfinita]\index[alph]{Teorema!de Recursión transfinita}\index[alph]{transfinita!Recursión}
        Para cualesquiera funcionales $F,G:\mathcal{V} \to \mathcal{V}$ y todo conjunto $A$, existe un único funcional $G:\OR \to \mathcal{V}$ de manera que:
        \begin{enumerate}
            \item $G(0)=A$.
            \item Para cada ordinal $\alpha$, $G(\alpha+1)=F(G(\alpha))$.
            \item Para cada ordinal límite $\gamma$, $G(\gamma)=H(G[\alpha])$.
        \end{enumerate}

        Además, existe un único funcional $K:\OR \to \mathcal{V}$ de manera que para todo ordinal $\alpha$ se satisface:
        \[ K(\alpha)=F(K[\alpha]) \]
    \end{teorema}

    Los teoremas anteriores se restringen a cualquier otro ordinal, consiguiéndose así las versiones clásicas para los teoremas de inducción y recursión (cada uno de ellos con dos versiones) para $\omega$ (o cualquier otro odrinal $\alpha$). Siendo tales restricciones las justificaciones rigurosas para ciertas técnicas y construcciones de las que se echa mano en este trabajo (véase \textbf{tal tal tal}).
    \index[alph]{ordinal!suma}\index[alph]{ordinal!producto}\index[alph]{ordinal!exponenciación}\index[sym]{$\alpha + \beta$}\index[sym]{$\alpha \cdot \beta$}\index[sym]{$\alpha ^ \beta$}
    Haciendo uso del Teorema de Recursión Transfinita, se pueden definir las operaciones binarias entre ordinales: $\alpha + \beta$, $\alpha \cdot \beta$ y $\alpha ^ \beta$, respectivamente. En caso se lleguen a utilizar durante la presente tesis, se indicará que tales símbolos corresponden a artimética es ordinal (para evitar confusión con la aritmética cardinal) y seguirá la definición expuesta en \cite[p.XXX]{amorIntermedio}.

    \begin{teorema}[de enumeración]\label{teo-enumeracion}\index[alph]{Teorema!de enumeración}\index[alph]{enumeración!Teorema de}
        Para cualquier buen orden $(P,<)$ existe un único ordinal $\alpha$ para el cual $(P,<) \cong (\alpha, \in)$.
    \end{teorema}

    \index[alph]{cardinalidad}\index[sym]{$|X|$}\index[alph]{conjunto!finito}\index[alph]{conjunto!infinito}\index[alph]{conjunto!numerable}\index[alph]{conjunto!a lo más numerable}\index[alph]{conjunto!no numerable}\index[alph]{conjunto!más que numerable}
    Tomando en cuenta que; bajo $\Ac$, cualquier conjunto admite un buen orden \cite[Teo.~5.1, p.~48]{jechSet}, se desprende de lo anterior que todo conjunto $X$ es biyectable con algún ordinal, al mínimo de tales ordinales se le denomina \textit{cardinalidad de} $X$ y se denota por $|X|$. Se conviene además que $X$ es: \textit{finito} si existe $n \in \omega$ tal que $|X|=n$; \textit{infinito} si $|X|\geq \omega$; \textit{numerable} si $|X|=\omega$; \textit{a lo más numerable} si $|X|\leq \omega$; y, \textit{más que numerable} (indistintamente, \textit{no numerable}) si $|X|>\omega$.
    
    \index[alph]{cardinal}\index[sym]{$\CAR$}
    Cualquier ordinal $\kappa$ que sea la cardinalidad de un conjunto tiene la virtud de no ser biyectable con ningun ordinal anterior a él, a estos ordinales se les llama \textit{cardinales}. Los cardinales se suelen denotar por letras griegas intermedias: $\kappa$, $\lambda$, $\mu$, etcétera. Se seguirá tal convención y además se denotará por $\CAR$ a la clase de cardinales mayores o iguales a $\omega$. Es un hecho que la intersección de una familia de cardinales, es un cardinal. En consecuencia, cualquier clase no vacía de cardinales tiene mínimo; y, cualquier conjunto de cardinales, supremo.

    \index[alph]{cardinal!suma}\index[alph]{cardinal!producto}\index[alph]{cardinal!exponenciación}\index[sym]{$\kappa + \lambda$}\index[sym]{$\kappa \cdot \lambda$}\index[sym]{$\kappa ^ \lambda$}\index[alph]{cardinal!suma general}\index[alph]{cardinal!producto general}\index[sym]{$\sum_{\alpha \in I} \kappa_\alpha$}\index[sym]{$\prod_{\alpha \in I} \kappa_\alpha$}
    Dados dos cardinales $\kappa$ y $\lambda$, se definen: $\kappa + \lambda:=|\kappa \times \{0\} \cup \lambda \times \{1\}|$, $\kappa \cdot \lambda:= |\kappa \times \lambda|$ y $\kappa ^ \lambda:= |\{f \tq f:\lambda \to \kappa\}|$; siendo las versiones generales de las dos primeras operaciones:
    \[ \sum_{\alpha \in I} \kappa_\alpha := \left|\bigcup_{\alpha \in I} (\kappa_\alpha \times \{\alpha\})\right| \quad \text{y} \quad \prod_{\alpha \in I} \kappa_\alpha := \left|\prod_{\alpha \in I} \kappa_\alpha\right| \]
    (cuando $\{\kappa_\alpha \tq \alpha \in I\}$ es un conjunto no vacío de cardinales).

    Se dará por sentado que el lector está familiarizado con la aritmética cardinal básica (véase \cite[Cap.~1, \S ~ 3]{jechSet}). Más allá de tal comportamiento elemental, se hace hincapié en los siguientes teoremas de suma relevancia para la aritmética cardinal:
    \begin{teorema}[suma y producto cardinal]\index[alph]{Teorema!de la suma cardinal}\index[alph]{Teorema!del producto cardinal}
        Si $\{\kappa_\alpha \tq \alpha \in I\}$ es conjunto no vacío de cardinales:
        \begin{enumerate}
            \item $\displaystyle \sum_{\alpha \in I} \kappa_\alpha = |I| \cdot \sup_{\alpha \in I} \kappa_\alpha$.
            \item Si ningun $\kappa_\alpha$ es $0$ y para cualesquiera $\alpha,\beta \in I$, $\alpha \leq \beta$ implica $\kappa_\alpha \leq \kappa_\beta$, entonces: $\displaystyle \prod_{\alpha \in I} \kappa_\alpha = \Big( \sup_{\alpha \in I} \kappa_\alpha \Big)^{|I|}$.
        \end{enumerate}
    \end{teorema}

    \begin{teorema}[Lema de König]\index[alph]{Lema!de König}\index[alph]{König!Lema de}\index[alph]{Teorema!de Cantor}\index[alph]{Cantor!Teorema de}
        Sean $\{\kappa_\alpha \tq \alpha \in I\}$ y $\{\lambda_\alpha \tq \alpha \in I\}$ conjuntos no vacíos de cardinales de modo que para todo $\alpha \in I$ se satisface $\kappa_\alpha < \lambda_\alpha$. Entonces:
        \[ \sum_{\alpha \in I} \kappa_\alpha < \prod_{\alpha \in I} \kappa_\alpha \]

        Particularmente, $\kappa = \sum_{\alpha \in \kappa} 1 < \prod_{\alpha \in \kappa} 2 = 2^\kappa$ (\textit{Teorema de Cantor}).
    \end{teorema}

    Del Lema anterior se desprende que si $\kappa \in \CAR$, existe $\lambda \in \CAR$ con $\kappa < \lambda$. Luego, se puede ordenar la clase $\CAR$ como:
    \begin{definicion}\index[sym]{$\aleph_\alpha$}\index[sym]{$\omega_\alpha$}
        Se define recursivamente; para cualquier ordinal $\alpha$, el número $\aleph_\alpha$, de la siguiente manera:
        \begin{enumerate}
            \item $\aleph_0:=\omega$.
            \item Para cada ordinal $\alpha$, $\aleph_{\alpha+1}:=\min\{ \lambda \in \CAR \tq \aleph_\alpha < \lambda \}$.
            \item Para cada ordinal límite $\gamma$, $\aleph_\gamma:=\sup_{\alpha < \gamma} \aleph_\alpha$.
        \end{enumerate}

        Además, para cada ordinal $\alpha$, se denota $\omega_\alpha:=\aleph_\alpha$.
    \end{definicion}
    
    \index[sym]{$[X]^\kappa$}\index[sym]{$[X]^{<\kappa}$}\index[sym]{$[X]^{\leq \kappa}$}\index[sym]{$[X]^{> \kappa}$}\index[sym]{$[X]^{\geq \kappa}$}\index[sym]{$X^\kappa$}\index[sym]{$X^{<\kappa}$}
    Siempre que $X$ sea un conjunto y $\kappa$ un cardinal, se escribirá por $[X]^\kappa$ al conjunto de todos los subconjuntos de $X$ de cardinalidad $\kappa$; $[X]^{<\kappa}$ al conjunto de todos los subconjuntos de $X$ de cardinalidad estrictamente menor que $\kappa$; definéndose análogamente a los conjuntos $[X]^{\leq \kappa}$, $[X]^{>\kappa}$ y $[X]^{\geq \kappa}$. Además, en caso no se confunda con la notación de aritmética cardinal, $X^\kappa$ será el conjunto de funciones de $\kappa$ en $X$; y, $X^{<\kappa}$ el conjunto de funciones de funciones $f:\alpha \to X$ (con $\alpha < \kappa$).
    
    Es un hecho que si $X$ es infinito, entonces $|[X]^\kappa|=|X|^\kappa$ y $|[X]^{<\omega}|=|X|$; además, $|X^\mu|=|X|^\mu$ y $|X^{<\omega}|=|X|$.

    \subsection{Árboles}

    \index[alph]{árbol}\index[alph]{árbol!orden de un elemento de un}\index{árbol!altura de un}\index[alph]{árbol!rama de un}\index{árbol!$\alpha$-ésimo nivel de un}\index[sym]{$<_x$}\index[sym]{$o(x)$}\index[sym]{$h(T,\leq)$}
    Un \textit{árbol} es un orden parcial $(T,\leq)$ (denotado simplemente por $T$ si no hay lugar a ambigüedades) tal que para cualquier $x \in T$, el conjunto $<_x:=<^{-1}[\{x\}]=\{y \in T \tq y < x\}$ es un buen orden. Dado el \autoref{teo-enumeracion}, para cada $x \in T$ existe un único ordinal, denotado $o(x)$ para el cual $(<_x,<) \cong (o(x),\in)$. Tal ordinal $o(x)$ es nombrado el \textit{orden de} $x$ \textit{en el árbol} $T$. La \textit{altura} de $T$ es el ordinal $h(T,\leq):=\sup\{o(x) +1 \tq x \in T\}$. Para cada ordinal $\alpha$ se define el $\alpha$\textit{-ésimo nivel} de $(T,\leq)$ como el conjunto $T_\alpha := \{ x \in T \tq o(x) = \alpha \}$. Y, finalmente, un subconjunto $R \subseteq T$ se dice que es \textit{rama} si y sólo si es una $(T,\leq)$-cadena $\subseteq$-maximal (del conjunto de $(T,\leq)$-cadenas).

    \index{árbol!de ramas de $2^\omega$}
    Dentro de la basta variedad de árboles, será de especial interés el \textit{árbol de ramas de} $2^\omega=\{f \tq f:\omega \to 2\}$; esto es, el conjunto $2^{<\omega}$ ordenado por contención. Tal árbol es numerable, todos sus elementos tienen orden finito y su altura es exactamente $\omega$.
    
    \label{arbol-2Ramas}
    En efecto, si $f \upharpoonright n \in 2^{<\omega}$, entonces $(n,\in) \cong (f, \subsetneq_f)$ debido al isomorfismo de orden $n \to \subsetneq_f$, dado por $n \mapsto f \upharpoonright n$. Por lo tanto $T$ es un árbol, y el orden de cada $f \in T$ es su dominio; como $2^{<\omega}$ contiene a todas las funciones de naturales en $2$, se sigue que la altura de $T$ es $\omega = \sup\{ n+1 \tq n \in \omega \}$.
		
	Además $T$ es numerable, ya que:
	$$ \omega \leq |2^{<\omega}| = \Big| \bigcup_{n \in \omega} 2^n \Big| \leq \sum_{n \in \omega} |2^n| = \omega $$

    Lo cual demuestra lo que se requería respecto al árbol $(2^{<\omega},\subseteq)$.

    \section{Topología}

    \subsection{Convenios generales y propiedades topológicas}
    \index[alph]{topología}\index[alph]{espacio!topológico}\index[alph]{espacio}\index[alph]{conjunto!cerrado}\index[alph]{conjunto!abierto}
    Una \textit{topología} para un conjunto $X$ es un conjunto $\tau \subseteq \ms{P}(X)$ que tiene por elementos a $\emptyset$ a $X$; es cerrado bajo uniones (arbitrarias); y, cerrado bajo intersecciones finitas. El par $(X,\tau)$ (con frecuencia confundido con su conjunto subyacente, $X$) se denomina \textit{espacio topológico} (o simplemente \textit{espacio}). Los elementos de $\tau$ se denominan \textit{abiertos} (\textit{de $X$}) y sus complementos respecto a $X$, \textit{cerrados} (\textit{de $X$}).
    
    \index[alph]{función!continua}\index[alph]{función!homeomorfismo}
    Dados dos espacios $X$ y $Y$, se dice que una función $f:X \to Y$ es \textit{continua} si para cada $U \subseteq Y$ abierto en $Y$, se tiene que $f^{-1}[U] \subseteq X$ es abierto en $X$. Un \textit{homemorfismo entre $X$ y $Y$} es una función continua $f:X \to Y$, biyectiva, cuya inversa $f^{-1}:Y \to X$ es también continua. Cuando exista un homeomorfismo entre $X$ y $Y$, esto se denotará $X \cong Y$.

    \index[alph]{subespacio}\index[alph]{topología!de subespacio}\index[alph]{encaje}
    Dados un espacio $(X,\tau)$ y $A \subseteq X$ se define la \textit{topología de subespacio} (\textit{de $A$ respecto $X$}) como la colección $\tau_A := \{ U \cap A \tq U \in \tau \}$ (que, claramente, es topología para $A$). Cuando $(X,\eta),(Y,\tau)$ sean espacios topológicos, se dice que una función $f:X \to Y$ es un \textit{encaje} si y sólo si $f$ es un homeomorfismo entre $(X,\eta)$ y $(f[X],\tau_{f[X]})$. En caso ocurra lo último, se convendrá que $X$ es un \textit{subespacio} de $Y$ (o bien, que $X$ \textit{se encaja en} $Y$) y, ocasionalmente, esto se denotará $X \hookrightarrow Y$. En este contexto, la notación ``$A \subseteq X$'' significará que $A$ está contenido en $X$ como conjunto y que $A \hookrightarrow X$ por medio del encaje $A \to X$ dado por $a \mapsto a$.

    \index[alph]{base}\index[alph]{subbase}
    Una \textit{base} para un espacio topológico $(X,\tau)$ es una coleccion $\mathcal{B} \subseteq \tau$ de forma que para cualquier abierto $U$ de $X$ y cada $x \in U$ existe cierto $B \in \mathcal{B}$ de forma que $x \in B \subseteq U$. 

    %\begin{teorema}\label{teo-subBase}\index[alph]{topología!generada}
    %    Para cualquier conjunto $X$ y cualquier $\mathcal{S} \subseteq \ms{P}(X)$, con $X \subseteq \midcup \mathcal{S}$, existe una topología $\tau$ para $X$ de forma que $\mathcal{S} \subseteq \tau$; y, para cualquier topología $\eta$ de $X$, si $\mathcal{S} \subseteq \eta$, entonces $\tau \subseteq \eta$.

    %   Tal topología $\tau$ se denomina \textbf{topología generada por} $\mathcal{S}$.
    %\end{teorema}
    
    \index[alph]{base!local}\index[alph]{base!de vecindades}\index[alph]{vecindad}
    Si $x \in X$, una \textit{vecindad de} $x$ (\textit{en} $X$) es un subconjunto $V \subseteq X$ de modo que existe un abierto $U$ de $X$ tal que $x \in U \subseteq V$. Además, se convendrá que una coleccion $\mathcal{B}_x \subseteq \ms{P}(X)$ es una \textit{base local} (\textit{de vecindades}, respectivamente) de $x$ en $X$ si y sólo si para cada elemento de $\mathcal{B}_x$ es una vecindad abierta (vecindad, respectivamente) de $x$; y, para todo abierto $U$ de $X$ con $x \in U$, existe $B \in \mathcal{B}_x$ de forma que $x \in B \subseteq U$.

    \index[alph]{operador!interior}\index[alph]{operador!clausura}\index[alph]{operador!exterior}\index[alph]{operador!frontera}\index[alph]{operador!derivado}\index[sym]{$\inte(A)$}\index[sym]{$\cla(A)$}\index[sym]{$\ext(A)$}\index[sym]{$\fron(A)$}\index[sym]{$\der(A)$}\index[alph]{punto!aislado}\index[alph]{punto!de acumulación}\index[alph]{conjunto!denso}
    Para cada $A \subseteq X$ se denotarán por $\inte(A),\cla(A),\ext(A),\fron(A),\der(A)$ a los \textit{operadores}: interior, clausura, exterior, frontera, y derivado de $A$, respectivamente. Sus definiciones se pueden consultar en \cite[Cap.~2]{fidelElementos}. Los elementos de $\der(A)$ se denominan \textit{puntos de acumulación de} $A$; y, los elementos en $A \setminus \der(A)$ se llaman \textit{puntos aislados de} $A$. Un subconjunto $D \subseteq X$ se dice \textit{denso} (\textit{en} $X$) si y sólo si $\cla(D)=X$.

    \index[alph]{producto!topológico}\index[alph]{producto!de Tychonoff}\index[alph]{suma topológica}
    Dado un conjunto no vacío de espacios topológicos $\{X_\alpha \tq \alpha \in \kappa \}$, se denotarán por $ \prod_{\alpha \in \kappa} X_\alpha$ y $ \coprod_{\alpha \in \kappa} X_\alpha$ a su \textit{producto topológico} (o, \textit{de Tychonoff}) y \textit{suma topológica}, respectivamete; siguiéndo las definiciones de estos espacios acorde al estándar, expuesto en textos como \cite{fidelElementos,munkresTopology}, entre otros. Al momento de trabajar con productos topológicos, será usual, para cada $\alpha \in \kappa$ denotar por $\pi_\alpha$ a la $\alpha$-ésima proyección cartesiana ($ \prod_{\beta \in \kappa} X_\beta \to X_\alpha$ dada por $f \mapsto f(\alpha)$).
    
    \index[alph]{propiedad!topológica}\index[alph]{propiedad!hereditaria}\index[alph]{propiedad!débilmente hereditaria}\index[alph]{propiedad!factorizable}\index[alph]{propiedad!productiva}\index[alph]{propiedad!finitamente productiva}
    Una propiedad $\varphi(X)$ (pensada como fórmula de la teoría de conjuntos) es: \textit{topológica} si es invariante bajo homeomorfismos; esto es, si $(X,\tau)$ y $(Y,\eta)$ son homeomorfos, entonces $\varphi(X)$ se satisface únicamente cuando $\varphi(Y)$ se satisface; \textit{hereditaria} (\textit{débilmente hereditaria}, respectivamente) cuando $\varphi(X)$ implica que para cualquier subespacio (subespacio cerrado, respectivamente) $A$ de $X$, $\varphi(A)$ se satisface; \textit{factorizable} si para cualquier conjunto no vacío de espacios topológicos $\{X_\alpha \tq \alpha \in \kappa \}$ se tiene que, si $\varphi(\prod_{\alpha \in \kappa} X_\alpha)$ se cumple, entonces $\forall \alpha \in \kappa (\varphi(X_\alpha))$ se satisface; \textit{productiva} (\textit{finitamente productiva}, respectivamente) si para cualquier cardinal $\kappa$ (natural $\kappa \in \omega$, respectivamente) no cero y familia $\{X_\alpha \tq \alpha \in \kappa \}$ de espacios, la satisfacción de $\forall \alpha \in \kappa (\varphi(X_\alpha))$ implica la satisfacción de $\varphi(\prod_{\alpha \in \kappa} X_\alpha)$. Además, si un espacio $X$ es tal que todos sus subespacios tienen una propiedad (a saber, $P$), $X$ se denomina \textit{hereditariamente $P$}.

    Las siguientes propiedades topológicas serán utilizadas a lo largo del texto. Un espacio $X$ se dice: \textit{Primero Numerable} (o $1\AN$) si cada uno de sus puntos admite una base local (equivalentemente, de vecindades) a lo más numerable; \textit{Segundo Numerable} (o $2\AN$) si admite una base a lo más numerable; \textit{Separable} si tiene un subconjunto denso y a lo más numerable; $\T_0$ si para cualesquiera $x,y \in X$ distintos existe un abierto $U$ de forma que $U \cap \{x,y\} \in \{\{x\},\{y\}\}$; $\T_1$ si para cada $x \in X$ el conjunto $\{x\}$ es cerrado, $T_2$ (o \textit{de Hausdorff}) si para cualesquiera $x,y \in X$ distintos existen abiertos ajenos $U,V$ tales que $x \in U$ y $y \in V$; \textit{regular} si para cualquier cerrado $F \subseteq X$ y cualquier $x \in X \setminus F$ existen abiertos $U,V$ ajenos de modo que $F \subseteq U$ y $x \in V$; $\T_3$ si es regular y $\T_1$; \textit{completamente regular} si para cualquier cerrado $F$ y punto $x \in X \setminus F$ existe una función continua $f:X \to \mathbb{R}$ de modo que $f(x)=0$ y $f[F] \subseteq \{1\}$; $\T_{3 \frac{1}{2}}$ (o \textit{de Tychonoff}) si es completamente regular y $\T_1$; \textit{normal} si para cualesquiera cerrados $F,G$ ajenos, existen abiertos ajenos $U,V$ de modo que $F \subseteq U$ y $G \subseteq U$; $\T_4$ si es normal y $T_1$.
    












    \newpage
    \section{Pruebas de Consistencia Relativa}
    \subsection{Preludio de Lógica}
    \subsection{Axioma de Martin}

    \index[alph]{orden!\textit{c.c.c.}}
    Un conjunto parcialmente ordenado $(P,\leq)$ es \textit{c.c.c.} (o bien, cuenta con la \textit{propiedad de anticadena contable}) si y sólo si cualquier $(P,\leq)$-anticadena es a lo más numerable.

    \index[alph]{filtro}\index[alph]{ideal}\index[alph]{filtro!propio}\index[alph]{ideal!propio}
    Un \textit{filtro} de $(P,\leq)$ es un subconjunto $F \subseteq P$ no vacío, cerrado por arriba (es decir, si $x \in F$ y $y \geq x$, entonces $y \in F$) y de elementos compatibles en $F$ (es decir, para cualesquiera $x,y \in F$ existe $r \in F$ de modo que $r \leq x$ y $r \leq y$). La noción de \textit{ideal} es dual a la de filtro; y, un filtro (o ideal) es \textit{propio} si y sólo si es distinto de $P$.

    \begin{observacion}\label{obs-filtr-potencia}
        Sea $X$ es conjunto, entonces $F \subseteq \ms{P}(X)$ es filtro (ideal) de $(\ms{P}(X),\subseteq)$ si y sólo si $F$ es no vacío, cerrado bajo superconjuntos (subconjuntos) y bajo intersecciones (uniones) dos a dos.
    \end{observacion}

    \index[alph]{subconjunto!denso (de un orden parcial)}\index[alph]{subconjunto!denso bajo $p$ (de un orden parcial)}
    Se conviene que un subconjunto $D \subseteq P$ es: \textit{denso} si y sólo si para cualquier $x \in P$ existe un elemento $d \in D$ de modo que $d \leq x$; \textit{denso bajo $p \in P$} cuando para cada $x \leq p$ existe $d \in D$ de modo que $d \leq x$.

    \index[alph]{filtro!genérico}\index[alph]{filtro!$\ms{D}$-genérico}
    Dada una colección $\ms{D} \subseteq \ms{P}(P)$ de subconjuntos densos de $(P,\leq)$, se dice que un filtro $G$ de $(P,\leq)$ es \textit{$\ms{D}$-genérico} si es propio y tiene intersección no vacía con cada elemento de $\ms{D}$. Un filtro $G$ es \textit{genérico} si es $\ms{D}$-genérico, donde $\ms{D}$ es la colección de todos los subconjuntos densos de $(P,\leq)$.

    El Axioma de Martin\footnote{que surgió como fruto del estudio de la \textit{Hipótesis de Souslin} (véase la discusión correspondiente en \cite{kunenSet})} se formula de la siguiente manera:
    \begin{definicion}\label{def-AxMartin}\index[alph]{Axioma!de Martin}\index[alph]{Martin!Axioma de}\index[sym]{$\Ma(\kappa)$}\index[sym]{$\Ma$}
        Para cada cardinal infinito $\kappa$, $\Ma(\kappa)$ es el enunciado: ``Para todo conjunto parcialmente ordenado $(P,\leq)$ \textit{c.c.c.} y cada colección $\ms{D}$ de conjuntos densos de $(P,\leq)$, con $|\ms{D}|\leq \kappa$, existe un filtro $\ms{D}$-genérico''.

        El enunciado $\Ma$ se definee como: ``Para cada cardinal infinito $\kappa<\mathfrak{c}$ se satisface $\Ma(\kappa)$''.
    \end{definicion}

    \index[sym]{$\mathfrak{m}$}
    Es un reesultado estándar y bien conocido que; en $\zfc$, $\Ma(\omega)$ es verdadero y $\Ma(\mathfrak{c})$ es falso; en consecuencia, $\Ma$ se suele utilizar junto con la negación de la hipótesis del continuo (para no obtener resultados siempre vacuos). Además, a razón de ello, está bien definido:
    $$ \mathfrak{m}:=\min\{ \kappa \geq \omega \tq \lnot \Ma(\kappa) \} $$

    Claramente $\aleph_1 \leq \mathfrak{m} \leq \mathfrak{c}$.

    \subsection{Forcing}
        \chapter{Familias casi ajenas}
	\emph{\small Las familias casi ajenas son objetos fascinantes en la teoría de conjuntos; y como se verá a lo largo de esta tesis, también en la topología. Entre los pioneros de su estudio destacan grandes figuras como Hausdorff, Sierpiński, Erdős y Rado.}
	
	\emph{\small El presente capítulo tiene como meta presentar las familias casi ajenas y exponer sus propiedades más inmediatas; los métodos más típicos para su construirlas; y finalmente, un estudio  básico sobre su combinatoria. En este última parte se abordarán resultados típicos; los Lemas de Dočkálková y Solovay, el Teorema de Simón y la existencia de las familias de Luzin.}

	\section{Observaciones inmediatas}

	\begin{definicion}\label{def-casi-ajena}\index[alph]{casi!ajeno}\index[alph]{casi!ajena sobre $N$, familia}\index[alph]{familia!casi ajena}\index[alph]{casi!ajena, familia}\index[alph]{familia!casi ajena sobre $N$}\index[sym]{$\Ad(N)$}
		Dado un conjunto numerable $N$, una \textbf{familia casi ajena sobre $N$} es un subconjunto $\ms{A}\subseteq[N]^\omega$ tal que cualesquiera dos elementos distintos de $\ms{A}$ son casi ajenos. Se denotará:
		$$ \Ad(N):=\{ \ms{A} \subseteq [N]^\omega \tq \ms{A} \text{ es familia casi ajena sobre } N \} $$
		y el término ``\textbf{familia casi ajena}'' (o simplemente ``\textbf{familia}'') hará referencia a una familia casi ajena sobre $\omega$.
	\end{definicion}

	El concepto previo es fácilmente generalizable, el lector puede indagar al respecto en \cite[Def.~9.20, p.~118]{jechSet}. Sin embargo, la teoría resultante del estudio de las familias casi ajenas (definidas como en \ref{def-casi-ajena}) tiene un gran valor por sí misma.

	\begin{observacion}
		Si $N$ es un conjunto numerable:
		\begin{enumerate}
			\item Cualquier subconjunto de una familia casi ajena sobre $N$ es también una familia casi ajena sobre $N$.
			\item Si $\ms{A} \subseteq [N]^\omega$ está enumerado como $\ms{A} =\{ a_\alpha \tq \alpha \in I \}$; para mostrar que $\ms{A}$ es familia casi ajena biyectable con $I$, bastará verificar si $\alpha \neq \beta$, entonces $a_\alpha \cap a_\beta$ es finito.
			\item Toda familia de subconjuntos infinitos de $N$, ajenos por pares, es casi ajena sobre $N$; en particular, $\left[ [N]^\omega \right]^{\leq 1} \subseteq \Ad(N)$.
		\end{enumerate}
	\end{observacion}
	
	Es claro que toda familia casi ajena tiene tamaño menor o igual a $\mathfrak{c}$; así que en virtud de la observacion previa, de existir alguna con tamaño exactamente el continuo, se garantizaría la existencia de familias ajenas de cualquier tamaño inferior a éste.

	\begin{ejemplo}
		\label{ej-ADfacil}
		Las colecciones $\{\omega\}$, $\{ \{ 2n \tq n \in \omega \}, \{ 2n+1 \tq n \in \omega \} \}$ y $\{ \{ p^n \tq n \in \omega \setminus \{0\} \} \tq p \text{ es primo} \}$ son familias casi ajenas sobre $\omega$.
	\end{ejemplo}
	
	Resulta no muy difícil verificar que las primeras dos familias del ejemplo anterior son ``grandes'', en el siguiente sentido:
	
	\begin{definicion}
		Sea $N$ conjunto numerable. Una familia casi ajena $\ms{A}$ sobre $N$ se dice \textbf{familia maximal en }$N$\index[alph]{familia!casi ajena sobre $N$!maximal en $N$} si y sólo si es un elemento $\subseteq$-maximal del conjunto $\Ad(N)$. Se denotará:\index[sym]{$\Mad(N)$} 
		$$ \Mad(N) = \{ \ms{A} \in \Ad(N) \tq \ms{A} \text{ es maximal en } N \} $$
		Cuando no haya riesgo de ambigüedad, el término \textbf{familia maximal}\index[alph]{familia!casi ajena!maximal} hará referencia a una familia maximal en $\omega$.
	\end{definicion}

	\begin{observacion}
		Sean $N$ un conjunto numerable. Una familia $\ms{A} \in \Ad(N) $ es maximal en $N$ si y sólo si se cumple cualquiera de las siguientes condiciones equivalentes:
		\begin{enumerate}
			\item Para toda $\ms{B} \in \Ad(N)$, si $\ms{A} \subseteq \ms{B}$, entonces $\ms{A} = \ms{B}$
			\item Para cualquier $\ms{B} \subseteq [N]^\omega$, si $\ms{A} \subsetneq \ms{B}$, entonces $\ms{B} \notin \Ad(N)$.
			\item Para cada $B \in [N]^\omega$ existe $A \in \ms{A}$ tal que $A \cap B$ es infinito.
		\end{enumerate}
	\end{observacion}

	Se advierte que las familias sobre $\omega$ parecerán deslucir a las construidas sobre otros conjuntos numerables; pero al no ser el estudio sobre éstas últimas nulo, es menester considerar las propiedades que son transferibles entre estas dos clases de objetos.

	\begin{definicion}\label{def-Biyecs-h}\index[sym]{$\Phi_h$}
		Sean $N,M$ conjuntos numerables y $h:N \to M$ cualquier biyección. Se define $\Phi_h : \ms{P}(\ms{P}(N)) \to \ms{P}(\ms{P}(M))$ como:
		$$ \Phi_h (\ms{A}) = \{ h[A] \tq A \in \ms{A} \} $$
	\end{definicion}

	En términos de lo recién enunciado, se remarca que al ser $h$ biyección, $\Phi_h$ será una biyección. Siendo claro además, que ésta respeta todas las virtudes conjuntistas.
	
	\begin{proposicion}\label{prop-ADbiyec}
		Sean $N,M$ son numerables y $h:N \to M$ una biyección cualquiera. Entonces:
		\begin{enumerate}
			\item $|\ms{A}|=|\Phi_h(\ms{A})|$.
			\item $\Phi_h(\ms{A} \cap \ms{B}) = \Phi_h(\ms{A}) \cap \Phi_h(\ms{B})$.
			\item $\Phi_h(\ms{A} \cup \ms{B}) = \Phi_h(\ms{A}) \cup \Phi_h(\ms{B})$.
			\item $\ms{A} \subsetneq \ms{B}$ ocurre si y sólo si $\Phi_h(\ms{A}) \subsetneq \Phi_h(\ms{B})$.
			\item $\Phi_h[\Ad(N)]=\Ad(M)$.
			\item $\Phi_h[\Mad(N)]=\Mad(M)$
		\end{enumerate}
	\end{proposicion}
	\begin{proof}
		Se mostrarán únicamente (v) y (vi). En ambos basta probar la contención directa, pues al ser $h$ biyección, $\Phi_h^{-1} = \Phi_{h^{-1}}$.
		
		(v) Si $\ms{A} \in \Ad(N)$, entonces $\ms{A} \subseteq [N]^\omega$ y así $\Phi_h(\ms{A}) \subseteq [M]^\omega$. Ahora, si $h[A],h[B] \in \Phi_h(\ms{A})$ son distintos, es necesario que $A \neq B$ y por ello $h[A] \cap h[B]=h[A \cap B]=^* \emptyset $, mostrando que $\Phi_h(\ms{A}) \in \Ad(M)$.
	
		(vi) Si $\ms{A} \in \Mad(\ms{A})$ y $B \subseteq M$ es infinito, entonces $h^{-1}[B] \subseteq N$ es infinito y existe $A \in \ms{A}$ tal que $A \cap h^{-1}[B]$ es infinito. Al ser $h$ biyección, $h[A \cap h^{-1}[B]]=h[A] \cap B$ es infinito, por ende $\Phi_h(\ms{A}) \in \Mad(M)$.
	\end{proof}
	
	Se consolida la usanzal; a partir de este momento, de hacer hincapié sobre cuáles propiedades u objetos basados en las familias casi ajenas se preservan bajo las biyecciones $\Psi_h$. 
	
	Una aplicación superflua del Corolario anterior es el nacimiento de un método cómodo para generar familias casi ajenas; en especial infinitas.
	
	\begin{ejemplo}
		\label{ej-Bandas}
		Claramente $\ms{A}=\{ \{n\} \times \omega \tq n \in \omega \} \in \Ad(\omega \times \omega)$. Así que si $h:\omega \times \omega \to \omega$ es biyección, entonces $\Psi_h(\ms{A}) \in \ms{A}$ es una familia casi ajena en $\omega$. Más aún, tal familia es del mismo tamaño que $\ms{A}$ (todo gracias a \ref{prop-ADbiyec})
	\end{ejemplo}
	
	A continuación se comenzarán a examinar las propiedades de las familias casi ajenas maximales; se tiene la intención de responder a las preguntas que surgen naturalmente como: ¿puede haber familias casi ajenas más que numerables?, o, ¿existen familias maximales infinitas?
	
	\begin{lema}\label{lem-MADnecesarioUnion}
		Si $\ms{A}$ es familia casi ajena maximal, entonces $\omega \subseteq^* \midcup \ms{A}$.
	\end{lema}
	
	\begin{proof}
		Por contrapuesta, supóngase que $\omega \not\subseteq^* \midcup \ms{A}$, es decir que el conjunto $B:=\omega \setminus \midcup \ms{A}$ es infinito. Si $A \in \ms{A}$, entonces $A \subseteq \midcup \ms{A}$ y así, $A \cap B \subseteq A \setminus \midcup \ms{A} \subseteq A \setminus A = \emptyset$. Por lo que $B \in [\omega]^\omega$ es casi ajeno con cada elemento de $\ms{A}$, mostrando que $\ms{A}$ no es maximal. 
	\end{proof}
	
	El recíproco del Lema previo falla para familias infinitas (véase la familia $\ms{B}$ del \autoref{ej-Bandas}); y de hecho, no se cuenta un resultado ``amigable'' para determinar cuándo estas resultan ser maximales (véanse \ref{prop-CaracMADIdeal} y \ref{prop-CaracMADPositiv}). En contraparte a esto, se deduce rápidamente la siguiente caracterización para la maximalidad de las familias casi ajenas finitas.
	
	\begin{corolario}\label{cor-MADnecesarioUnion}
		Sea $\ms{A}$ una familia casi ajena finita. Entonces $\ms{A}$ es maximal si y sólo si $\omega \subseteq^* \midcup \ms{A}$.
	\end{corolario}
	
	\begin{proof}
		Por el Lema previo, basta demostrar la necesidad.
		
		Supóngase $\omega \subseteq^* \midcup \ms{A}$ y nótese que si $B \in [\omega]^\omega$, entonces $B \subseteq^*\midcup \ms{A}$ y con ello $\emptyset \neq^* B  \subseteq^* \midcap \ms{A} = \midcup\{B \cap A \tq A \in \ms{A}\}$. Como la última es una unión finita, $B$ debe tener intersección finita con algún elemento de $\ms{A}$.
		
		Esto es, todos los subconjuntos de $\omega$ casi ajenos con $\ms{A}$ son finitos, por lo que $\ms{A}$ debe ser maximal. 
	\end{proof}

	El posterior resultado puede ser visto como un símil al aclamado Teorema del Ultrafiltro (todo filtro se extiende a un filtro maximal) o cualquier resultado afín en el que típicamente se haga uso de formas $\Ac$ relacionadas con órdenaes parciales
	
	\begin{lema}\label{lem-MADs}
		Toda familia casi ajena está contenida en una familia maximal.
	\end{lema}
	
	\begin{proof}
		Sean $\ms{A} \in \Ad(\omega)$ y $X$ el conjunto de todos las familias casi ajenas que contienen a $\ms{A}$. Como $(X,\subseteq)$ es un conjunto parcialmente ordenado y no vacío, por el Principio de Maximalidad de Hausdorff ($\Ac$), existe $Y \subseteq X$, una cadena $\subseteq$-maximal de $(X,\subseteq)$.
	
		Defínase $\ms{B}:=\midcup Y$, como $Y \subseteq \ms{P}([\omega]^\omega)$, entonces $\ms{B} \subseteq [\omega]^{\omega}$. Además, si $C,D \in \ms{B}$, existen $\ms{C},\ms{D} \in Y \subseteq \Ad(\omega)$ con $C \in \ms{C}$ y $D\in \ms{D}$. Puesto que $Y$ es cadena de $(X,\subseteq)$, sin pérdida de generalidad, $C,D \in \ms{D} \supseteq \ms{C}$; y con ello, $C \cap D$ es finito. Por lo que $\ms{B} \in \Ad(\omega)$.
		
		Finalmente, si $\ms{B}' \in \Ad(\omega)$ y $\ms{B} \subsetneq \ms{B}'$, entonces $Y \cup \{\ms{B}'\}$ es una cadena en $(X,\subseteq)$ con $Y \subsetneq Y\cup\{\ms{B}'\}$, lo que contradice la $\subseteq$-maximalidad de $Y$. Por lo tanto, $\ms{B}\in \Mad(\omega)$ y $\ms{A} \subseteq \ms{B}$.
	\end{proof}
	
	%\begin{observacion}
	%	Se destaca de la prueba anterior el siguiente hecho general. Si $X \subseteq \Ad(\omega)$ y $(X,\subseteq)$ es orden total, $\midcup X \in \Ad(\omega)$.
	%\end{observacion}
	
	Si bien las familias maximales finitas existen y su obtención resulta simple (\autoref{cor-MADnecesarioUnion}), la siguiente proposición es testigo de que construir una familia maximal infinita requiere de un nivel superior de creatividad. Pese a no haber un acuerdo general, el resultado se le atribuye a Wacław Sierpinski, pues éste se desprende de \cite[Teo.~2, p.~458]{SierpinskiCardinal}.
	
	\begin{lema}\label{prop-MADnoNum}
		Ninguna familia casi ajena numerable es maximal.
	\end{lema}
	
	\begin{proof}
		Sea $\ms{A}$ una familia casi ajena numerable indexada como $\ms{A}=\{A_n \tq n \in \omega\}$. Si $n \in \omega$ es cualquiera, $A_n \cap \midcup \{A_m \tq m< n\}$ es finito pues $A_n$ es casi ajeno con cada $A_m$ (si $m<n$). Así que por ser $A_n$ infinito, el conjunto $A_n \setminus \midcup \{A_m \tq m<n\} = A_n \setminus \big( A_n \cap \midcup \{A_m \tq m<n\} \big)$ es infinito, particularmente no no vacío.
	
		Sea $f:\omega \to \omega$ definida por $f(n) = \min\{A_n \setminus \midcup \{A_m \tq m<n\}\}$ para cada $n$. Así, $f$ es inyectiva, pues si $m<n$, entonces $f(n) \notin A_m$ y $f(m) \in A_m$, de donde $f(n) \neq f(m)$. Además, es claro que para cada $n \in \omega$, se tiene $A_n \cap \ima(f) = \{f(n)\}$. Entonces $\ima(f) \subseteq \omega$ es infinito y casi ajeno con cada elemento de $\ms{A}$. 
	\end{proof}
	
	Naturalmente, se conjetura la existencia de las familias maximales infinitas, el Axioma de Elección y su aplicación \ref{lem-MADs}, brinda la respuesta; misma que, dada su naturaleza no constructiva, podría ser considerada tan insatisfactoria como destacable.
	
	\begin{observacion}\label{obs-ExisteNoNumMAD}
		Existe una familia maximal más que numerable. 
		
		Efectivamente, considérese cualquier familia $\ms{A} \in \Ad(\omega)$ numerable. Por el \autoref{lem-MADs}, existe una familia maximal $\ms{B} \supseteq \ms{A}$. Así $\ms{B}$ es infinita y es numerable, dada la Proposición anterior.
	\end{observacion}

	\section{Familias casi ajenas de tamaño \texorpdfstring{$\mathfrak{c}$}{c}}

	La presente sección tiene por meta exhibir dos de los métodos más típicos para la construcción de familias casi ajenas infinitas. El primero de ellos, se basa en las sucesiones convergentes de espacios topológicos de Hausdorff, primero numerables.
	
	\begin{lema}
		Sean $X$ un espacio topológico $\T_1$, de Fréchet, $A\subseteq X$ denso en $X$ y $A \subseteq D \setminus X$. Para cada $x \in A$ existe una sucesión en $D$, inyectiva y convergente a $x$.
	\end{lema}
	\begin{proof}
		Sea $x\in A$ arbitrario. Como $D$ es denso en $X$ y éste es de Fréchet, $x \in \cla(D)=\scl(D)$ y se puede fijar una sucesión en $D$ convergente a $x$. Puesto que el espacio $X$ es $\T_1$, tal sucesión debe ser infinita; y como es convergente, sin pérdida de generalidad, inyectiva.
	\end{proof}

	%Agregando el Axioma de separación de Hausdorff, se deriva una forma muy conveniente de obtener familias casi ajenas.
	
	\begin{proposicion}\label{prop-famSucesiones}
		Sean $X$ un espacio topológico más que numerable, de Hausdorff, de Fréchet. Si $D$ es un denso numerable de $X$, para cada $A \subseteq X \setminus D$ existe una familia casi ajena sobre $D$ biyectable con $A$.
	\end{proposicion}
	
	\begin{proof}
		Fíjese $D\subseteq X$ un denso numerable de $X$. Usando el Lema previo, para cada $x \in A$ fíjese ($\Ac$) $A_x \subseteq D$ numerable de modo tal que $A_x \to x$. Defínase el conjunto $\ms{A}_{D,A}:=\{ A_x \subseteq D \tq x \in A \}$, nótese que $\ms{A}_{D,A} \subseteq [D]^\omega$ y $|\ms{A}_{D,A}|=|A|$.

		Sean $x,y \in A$ con $x \neq y$, por ser $X$ de Hausdorff, hay abiertos ajenos $U,V$ tales que $x \in U$ y $y \in V$. Seguido de que $A_x \to x$ y $A_y \to y$, se tiene $A_x \subseteq^* U$ y $A_y \subseteq ^* V$, y en consecuencia $A_x \cap A_y \subseteq^* U \cap V = \emptyset$. Lo cual prueba que $\ms{A}_{D,A} \in \Ad(D)$.
	\end{proof}
	
	\begin{definicion}\label{def-FamSucesiones}\index[alph]{familia!de!sucesiones en $D$ convergentes a $A$}\index[sym]{$\ms{A}_{D,A}$}
		Sean $X$ un espacio topológico de Hausdorff, de Fréchet, $D \subseteq X$ denso numerable y $A \subseteq X \setminus D$.
		
		La familia $\ms{A}_{D,A}:=\{ A_x \subseteq D \tq x \in A \}$; construida como en la demostración anterior, se denomina \textbf{familia de sucesiones en $D$ convergentes a $A$}.
	\end{definicion}

	Como la recta real $\mathbb{R}$ es de Hausdorff, de Fréchet (por ser $1\AN$) y $\mathbb{Q} \subseteq \mathbb{R}$ es un subespacio denso numerable; de lo previamente establecido se obtiene que $\ms{A}_{\mathbb{Q},\mathbb{R}\setminus \mathbb{Q}}$ es una familia casi ajena sobre $\mathbb{Q}$ de tamaño $\mathfrak{c}=|\mathbb{R} \setminus \mathbb{Q}|$.
	
	La próxima estrategia de construcción se basa en considerar ciertas ramas del árbol $2^{<\omega}$.

	\begin{lema}
		Sean $(T,\leq)$ un árbol y $S \subseteq T$ cualquier rama. Si $x \in S$, entonces $S \subseteq <_{x_0}$.
	\end{lema}
	\begin{proof}
		Sean $x \in S$ y $y \in <_{x}$ cualesquiera. Si $s \in S$, como $S$ es cadena, se tiene que $x \leq s$ o $s < x$. En el primer caso, $y<x\leq s$ y $y$ es comparable con $s$. En el segundo caso $y,s \in <_{x}$; y como $(<_{x},\leq)$ es buen orden (por ser $(T,\leq)$ un árbol), $y$ y $s$ son comparables.

		Por tanto, $S \cup \{y\}$ es una cadena; y seguido de que $S$ es rama, $y \in S$, lo cual demuestra la contención deseada.
	\end{proof}

	\begin{proposicion}
		Sean $(T,\leq)$ un árbol numerable de altura $\omega$ y $\ms{A} \subseteq \ms{P}(T)$ un conjunto de ramas numerables de $(T,\leq)$. Entonces $\ms{A}$ es una familia casi ajena sobre $T$.
	\end{proposicion}
	
	\begin{proof}
		Nótese que $\ms{A} \subseteq [T]^\omega$. Sean $R,S \in \ms{A}$ distintos, entonces $R \cap S$ es vacío y finito, o se puede fijar cierto $x_0 \in R \cap S$; en cuyo caso, de la proposición previa se desprende que $R \cap S \subseteq <_{x_0}$.
		
		Como $T$ tiene altura $\omega$, el orden de $x_0$ es natural, consecuentemente se tiene que $R \cap S$ es finito. 
	\end{proof}

	Un ejemplo canónico de árbol numerable de altura $\omega$ es $2^{<\omega}$ (véase \pageref{arbol-2Ramas}); considerar la siguiente clase de familias en él desenvocará en resultados súmamente notables (como se puede ver en la \autoref{Sec-PDM}).

	\begin{proposicion}
		Sea $T$ el árbol $(2^{<\omega},\subseteq)$ y para cada $f \in 2^\omega$ denótese $A_f:=\{ f \upharpoonright n \tq n \in \omega \} \subseteq 2^{<\omega}$; entonces:
		\begin{enumerate}
			\item Cada $A_f$ es una rama de $T$.
			\item Si $f\neq g$, entonces $A_f \neq A_g$.
			\item Para cada $X \subseteq 2^\omega$, el conjunto $\ms{A}_X := \subseteq \{A_f \tq f \in 2^\omega\}$ es una familia casi ajena sobre $2^{<\omega}$ biyectable con $X$.
		\end{enumerate}
	\end{proposicion}
	\begin{proof}
		Basta ver (i) y (ii), así, (iii) se sigue del Lema previo.

		(i) Sea $f \in 2^\omega$, inmediatamente, $A_f$ es cadena de $T$. Supóngase que $S \subseteq 2^{<\omega}$ es una rama de $T$ tal que $A_f \subseteq S$ y sea $g \in S$. Si $\dom(g)=n$, dado que $S$ es cadena de $T$, $f \upharpoonright n \subseteq g$ o $g \subseteq f \upharpoonright n$. Cualquiera de los casos anteriores implican que $f \upharpoonright n = g$ ya que $\dom(g)=\dom(f \upharpoonright n)$, así que $g \in A_f$ y $A_f = S$.

		(ii) si $f \neq g$, entonces existe $m\in \omega$ tal que $f(m) \neq g(m)$. Así, se obtiene que $f \upharpoonright m+1 \neq g \upharpoonright m+1$ y $f \upharpoonright m+1 \in R_f \setminus R_g$.
	\end{proof}

	\begin{definicion}\label{def-FamRamas}\index[sym]{$\ms{A}_X$}\index[alph]{familia!de!ramas de $X$ en $2^\omega$}
		Para cada $X \subseteq 2^\omega$ defínase $\ms{A}_X:=\{A_f \tq f \in X\}$ como en la proposición previa.
		
		Esta familia será nombrada la \textbf{familia de las ramas de $X$ en $2^{<\omega}$}.
	\end{definicion}

	En paralelo a lo comentado después de \ref{def-FamSucesiones}, también se puede concluir vía la construcción recién expuesta (y el \autoref{lem-MADs}) lo siguiente.

	\begin{corolario}\label{cor-famGrandes}
		Existe una familia maximal de cardinalidad $\mathfrak{c}$.
		
		Además, para cualquier cardinal $\lambda \leq \mathfrak{c}$ existe una familia casi ajena de cardinalidad $\lambda$.
	\end{corolario}

	Se concluirá esta sección comentando cosas en relación a la pregunta obvia: ¿existen familias maximales de cualquier cardinalidad entre $\aleph_1$ y $\mathfrak{c}$?
	
	\begin{definicion}
		Se define el \textbf{cardinal de casi ajenidad} como:\index[alph]{cardinal! de casi ajenidad}\index[alph]{casi!ajenidad, cardinal de} \index[sym]{$\mathfrak{a}$}
		$$ \mathfrak{a}:=\min\{ \kappa \geq \omega \tq \text{Existe una familia maximal de cardinalidad } \kappa \} $$
	\end{definicion}
	
	Debido a \ref{prop-MADnoNum}, se tiene $\aleph_1 \leq \mathfrak{a} \leq \mathfrak{c}$ y claramente bajo $\HC$ se debe satisfacer $\mathfrak{a}=\mathfrak{c}$; luego, es consistente con $\zfc$ que $\mathfrak{a}=\mathfrak{c}$. Comentar que la teoría en relación al cardinal $\mathfrak{a}$ (así como de otros cadrinales importantes) es incríblemente basta y existen resultados de consistencia como el siguiente.
	
	\begin{teorema}\label{teo-stafa}
		Si $\kappa$ es cualquier cardinal regular con $\aleph_1 \leq \kappa \leq \mathfrak{c}$, es consistente con $\zfc$ que $\mathfrak{a}=\kappa$.
	\end{teorema}

	El Teorema recién enunciado consecuencia de \cite[Teo.~5.1, p.~127]{kunenHandbook}; y, pese a que su demostración es ajena a los proposítos de la presente disertación, conviene remarcar que se harán más comentarios respecto al enunciado $\mathfrak{a}=\mathfrak{c}$ posteriormente (véase \ref{cor-MaSimple}).

	\section{El ideal generado y su comportamiento}
	\label{Sec-IdealGenerado}
	\index[alph]{ideal generado por $\ms{A}$}\index[sym]{$\ms{I}_N(\ms{A})$}\index[alph]{parte!positiva de $\ms{A}$}\index[sym]{$\ms{I}_N^+(\ms{A})$}\index[sym]{$\ms{I}(\ms{A})$}\index[sym]{$\ms{I}^+(\ms{A})$}
	\begin{definicion}\label{def-ideal}
		Si $N$ es numerable y $\ms{A} \in \Ad(N)$:
		\begin{enumerate}[i)]
			\item El \textbf{ideal generado por $\ms{A}$} es el conjunto:
			$$ \ms{I}_N(\ms{A}) := \{ B \subseteq N \tq \exists H \in [\ms{A}]^{<\omega} \: ( B \subseteq^* \midcup H ) \} $$
			\item La \textbf{parte positiva de $\ms{A}$} es $ \ms{I}_N^+(\ms{A}) := \ms{P}(N) \setminus \ms{I}_N(\ms{A})$.
		\end{enumerate}
		Si $N=\omega$, se escribirá únicamente $\ms{I}(\ms{A})$ ($\ms{I}^+(\ms{A})$, respectivamente).
	\end{definicion}

	El objeto introducido previamente es de vital importancia para el estuido de la combinatoria de las familias casi ajenas. Como se había advertido, resulta necesario realizar la siguiente observación con el propósito de no perder generalidad con los resultados mostrados durante esta sección.
	
	\begin{proposicion}\label{prop-IdealBiyec}
		Sean $N,M$ conjuntos numerables y $h:N \to M$ biyectiva. Si $\ms{A} \in \Ad(N)$, entonces $\Phi_h (\ms{I}_N (\ms{A})) = \ms{I}_M (\Phi_h( \ms{A} )) $.
	\end{proposicion}
	
	\begin{proof}
		Como $\Phi_h^{-1} = \Phi_{h^{-1}}$, basta probar una contención de la igualdad deseada. Sea $Y \in \ms{I}_N (\ms{A})$ cualquiera, entonces existe $H \subseteq \ms{A}$ finito tal que $Y \subseteq^* \midcup H$, esto es, $Y \setminus \midcup H$ es finito. Como $h$ es biyectiva, se obtiene que $ h \big[ Y \setminus \midcup H \big] = h[Y] \setminus h\big[ \midcup H \big] = h[Y] \setminus \midcup \Phi_h(H)$ es finito.
	
		Luego $h[Y] \subseteq^* \midcup \Phi_h(H) $ y $\Phi_h(H) \subseteq \Phi_h(\ms{A})$ es finito, mostrando de esta forma que $h[Y] \in \ms{I}_M (\Phi_h( \ms{A} ))$. 
	\end{proof}

	Resulta sencillo constatar que el objeto definido en \ref{def-ideal} es; como su nombre indica, un ideal (no necesariamente propio) sobre $\ms{P}(\omega)$. Además, se destacan las siguientes dos observaciones.
	
	\begin{observacion}\label{obs-IdealPrevia}
		Si $\ms{A}$ es familia casi ajena, entonces:
		\begin{enumerate}[i)]
			\item Cualquier subconjunto finito de $\omega$, así como cualquier elemento de $\ms{A}$, es elemento de $\ms{I}(\ms{A})$. Por lo que se dan las contenciones $\emptyset \subsetneq [\omega]^{<\omega} \cup \ms{A} \subseteq \ms{I}(\ms{A})$.
			\item Si $\ms{B} \in \Ad(\omega)$ y $\ms{A} \subseteq \ms{B}$, entonces $\ms{I}(\ms{A}) \subseteq \ms{I}(\ms{B})$.
		\end{enumerate}
	\end{observacion}
	
	Cada vez que $\ms{A}$ sea una familia casi ajena maximal y finita, en virtud del \autoref{lem-MADnecesarioUnion} se tendrá que $\omega \in \ms{I}(\ms{A})$, pues $\ms{A} \subseteq \ms{A}$ es finito y $\omega \subseteq^* \midcup \ms{A}$. El recíproco de esto también es cierto.
	
	\begin{proposicion}
		Sea $\ms{A}$ familia casi ajena. Si $\omega \in \ms{I}(\ms{A})$, entonces $\ms{A}$ es finita y maximal.
	\end{proposicion}
	 
	\begin{proof}
		Supóngase que $\omega \in \ms{I}(\ms{A})$, entonces existe $H \subseteq \ms{A}$ finito con $\omega \subseteq^* \midcup H \subseteq \midcup \ms{A}$. Por \ref{cor-MADnecesarioUnion}, basta ver que $\ms{A}$ es finita.

		Por ser $\ms{A}$ casi ajena, cada $B \in \ms{A} \setminus H$ es infinito y casi ajeno con cada elemento de $H$, pero esto implica $ B = B \cap \omega \subseteq^* \midcup \{B \cap h \tq h \in H \} =^* \emptyset $ (por ser $H$ un conjunto finito), lo cual es imposible. Así $\ms{A} \subseteq H$ es finita.
	\end{proof} 
	
	\begin{corolario}\label{cor-IdealPropioCaract}
		Sean $N$ un conjunto numerable y $\ms{A}$ cualquier familia casi ajena sobre $N$. Las siguientes condiciones son equivalentes:
		\begin{enumerate}[i)]
			\item $\ms{A}$ es infinita o no maximal en $N$.
			\item $\ms{I}_N(\ms{A})$ es ideal propio en $(\ms{P}(N),\subseteq)$, es decir, $N \notin \ms{I}_N(\ms{A})$.
		\end{enumerate}
	\end{corolario}
	
	Con relativa frecuencia aparecerán familias que, pese a no ser maximales, satisfacen la condición (ii) de lo subsecuente; ésta puede ser tomada como un debilitamiento de la maximalidad.
	
	\begin{definicion}\label{def-MaxEnAlguna}\index[alph]{traza de $\ms{A}$ en $X$}\index[sym]{$\ms{A} \upharpoonright X$}\index[alph]{familia!maximal en alguna parte}\index[alph]{familia!maximal en ninguna parte}
		Si $N$ es numerable y $\ms{A}$ familia casi ajena sobre $N$:
		\begin{enumerate}[i)]
			\item Dado $X \subseteq N$ infinito, la \textbf{traza de $\ms{A}$ en $X$} se define como:
			$$ \ms{A} \upharpoonright X := \{ A \cap X \in [X]^\omega \tq A \in \ms{A} \} $$
			\item $\ms{A}$ es \textbf{maximal en alguna parte} si y sólo si existe $X \in \ms{I}_N^+(\ms{A})$ tal que la familia $\ms{A} \upharpoonright X$ es maximal en $X$.
			\item $\ms{A}$ es \textbf{maximal en ninguna parte} si y sólo si no es maximal en alguna parte. 
		\end{enumerate}
	\end{definicion}
	
	Sin causa de asombro, los conceptos recién establecidos son respetados por las biyecciones $\Phi_h$.
	
	\begin{proposicion}
		Si $N,M$ son conjuntos numerables y $h:N \to M$ es biyección, entonces para cada $\ms{A} \in \Ad(N)$:
		\begin{enumerate}[i)]
			\item Para cada $X \in [N]^\omega$ se cumple $\Phi_h(\ms{A} \upharpoonright X)=\Phi_h(\ms{A}) \upharpoonright h[X]$.
			\item $\ms{A}$ es maximal en alguna parte si y sólo si $\Phi_h(\ms{A})$ es maximal en alguna parte.
		\end{enumerate}
	\end{proposicion}
	
	\begin{proof}
		(i) Nótese que por ser $h$ biyección:
		\begin{align*}
		\Phi_h(\ms{A}) \upharpoonright h[X] & = \{ B \cap h[X] \in \big[ h[X] \big]^\omega \tq B \in \Phi_h(\ms{A}) \} \\
		& = \{ h[A] \cap h[X] \in \big[ h[X] \big]^\omega \tq A \in \ms{A} \} \\
		& = \{ h[A \cap X] \in \big[ h[X] \big]^\omega \tq A \in \ms{A} \} \\
		& = \Phi_h (\ms{A} \upharpoonright X)
		\end{align*}	
		(ii) Como $\Phi_h ^{-1} = \Phi_{h^{-1}}$, basta probar la suficiencia. Supóngase que $\ms{A}$ es maximal en alguna parte, entonces existe $X \in \ms{I}^+(\ms{A})$ tal que $\ms{A} \upharpoonright X$ es maximal en $X$. Dada la igualdad de \ref{prop-IdealBiyec}, $h[X] \in \ms{I}^+(\Phi_h(\ms{A}))$.
		
		Además, dado que $g:= h \upharpoonright X : X \to h[X]$ es biyección y se tiene que $\Phi_h(\ms{A}) \upharpoonright h[X]=\Phi_h (\ms{A} \upharpoonright X)=\Phi_g (\ms{A} \upharpoonright X)$, se desprende del \autoref{prop-ADbiyec} que $\Phi_h(\ms{A}) \upharpoonright h[X]$ es maximal en $h[X]$. 
	\end{proof}
	
	Es adecuado señalar la siguiente serie de observaciones; que aunque técnicas, permitirán manejar con soltura tanto las trazas de familias casi ajenas, como sus ideales generados.
	
	\begin{proposicion}\label{prop-TrazaBasicos}
		Sean $\ms{A},\ms{B}$ familias casi ajenas y $X,Y \in [\omega]^\omega$ cualesquiera, entonces:
		\begin{enumerate}[i)]
			\item Si $\ms{A} \subseteq \ms{B}$, entonces $\ms{A} \upharpoonright X \subseteq \ms{B} \upharpoonright X$.
			\item Se da la igualdad $(\ms{A} \upharpoonright Y) \upharpoonright X = \ms{A} \upharpoonright (Y \cap X)$.
			\item Si $X \subseteq Y$, entonces $\ms{I}_X(\ms{A} \upharpoonright X) \subseteq \ms{I}_Y(\ms{A} \upharpoonright Y)$.
		\end{enumerate}
	\end{proposicion}
	\begin{proof}
		El punto (i) es claro.
	
		(ii) Si $(A \cap Y) \cap X \in (\ms{A} \upharpoonright Y) \upharpoonright X$ es cualquier elemento, entonces $A \in \ms{A}$ y $(A \cap Y) \cap X = A \cap (Y \cap X)$ es infinito, de donde $(A \cap Y) \cap X \in \ms{A} \upharpoonright (Y \cap X)$. Recíprocamente, si $A \cap (Y \cap X) \in \ms{A} \upharpoonright (Y \cap X)$, entonces $A \in \ms{A}$ y $A \cap (Y \cap X)$ es infinito y como $A \cap (Y \cap X) \subseteq A \cap Y$, entonces $A \cap Y$ es infinito, en consecuencia $A \cap (Y \cap X) = (A \cap Y) \cap X \in (\ms{A} \upharpoonright Y) \upharpoonright X$.

		(iii) Supóngase que $X \subseteq Y$ y sea $B \in \ms{I}_X(\ms{A} \upharpoonright X)$. Entonces $B \subseteq X \subseteq Y$ y existe $H \subseteq \ms{A} \upharpoonright X$ finito tal que $B \subseteq^* \midcup H$. Cada $A \cap X \in H$ es infinito, luego $A \cap Y$ es infinito y así $H \subseteq J:=\{A \cap Y \tq A \cap X \in H\}$. De manera que $B \subseteq^* \midcup J$ y por lo tanto $B \in \ms{I}_Y(\ms{A} \upharpoonright Y)$. 
	\end{proof}
	
	%\subsection{Caracterización de la maximalidad de una familia}\label{subsec-Ideal}
	
	Se darán a continuación una serie de observaciones que conectan la maximalidad de una familia con sus trazas y su ideal generado.
	
	\begin{proposicion}\label{prop-CaracMADIdeal}
		Sea $\ms{A} \in \Ad(\omega)$. Entonces $\ms{A}$ es maximal si y sólo si para cada $X \in \ms{I}(\ms{A})$, $\ms{A} \upharpoonright X$ es finita y maximal en $X$.
	\end{proposicion}
	
	\begin{proof}
		Supóngase que $\ms{A}$ es maximal y sea $X \in \ms{I}(\ms{A})$. Entonces, existe $H \subseteq \ms{A}$ finito tal que $X \subseteq^* \midcup H$. Luego $X \subseteq^* C \cap \midcup H$ y como $H$ es un conjunto finito:
		\begin{align*}
			X \subseteq^* \midcap\{ A \cap X \in [\omega]^\omega \tq A \in H \} \cup \midcup\{ A \cap X \in [\omega]^{<\omega} \tq A \in H \} \\
			=^* \midcap\{ A \cap X \in \ms{A} \upharpoonright X \tq A \in H \}
		\end{align*}
		mostrando que $X \in \ms(I)_X(\ms{A}\upharpoonright X)$, y por \ref{cor-IdealPropioCaract}, $\ms{A} \upharpoonright X$ es finita y maximal en $X$.

		Para el recíproco procédase por contrapuesta. Si $\ms{A}$ no es maximal, se sigue del \autoref{cor-IdealPropioCaract} que $\omega \in \ms{I}(\ms{A})$. Así que $\ms{A} = \ms{A} \upharpoonright \omega$ no es familia maximal en $\omega$. 
	\end{proof}

	Si $\ms{A}$ es maximal, cada subconjunto infinito de $\omega$ tiene intersección infinita con al menos un elemento de $\ms{A}$. En comparativa, cada conjunto en la parte positiva de $\ms{A}$ tiene un comportamiento más fuerte, más allá de complirse:
	\[ \{X \in [\omega]^{\omega} \tq \forall A \in \ms{A} \: (A \cap X =^* \emptyset) \} \subseteq \ms{I}^+(\ms{A}) \]
	ocurre la siguiente caracterización:

	\begin{proposicion}\label{prop-CaracMADPositiv}
		Sean $\ms{A}$ una familia casi ajena. Entonces $\ms{A}$ es maximal si y sólo si para cada $X \in \ms{I}^+(\ms{A})$ la familia $\ms{A} \upharpoonright X$ es infinita.
	\end{proposicion}
	
	\begin{proof}
		Prócedase a probar la suficiencia por contrapuesta. Supóngase que existe $X \in \ms{I}^+(\ms{A})$ tal que la familia $\ms{A} \upharpoonright X$ es finita, entonces el conjunto $H:=\{ A \in \ms{A} \tq A \cap X \neq^* \emptyset \}$ es finito. Como $X \notin \ms{I}(\ms{A})$, el conjunto $B:=X \setminus \midcup H \subseteq \omega$ es infinito. Si $A \in \ms{A}$ es cualquiera, $A \cap B$ no puede ser infinito, sino $A \cap X$ es infinito, $A \in H$ y en consecuencia $A \cap B = (A \cap X) \setminus \midcup H \subseteq (A \cap X) \setminus A = \emptyset$, lo cual no tiene sentido. Así que $B$ es casi ajeno con cada elemento de $\ms{A}$ y $\ms{A}$ no es maximal.
	
		Para la necesidad, procédase de nuevo por contrapuesta. Si $\ms{A}$ no es maximal, existe $B \in [\omega]^\omega$ casi ajeno con cada elemento de $\ms{A}$. Nótese que entonces $\ms{A} \upharpoonright B = \emptyset$ es finita. Además $B \notin \ms{I}(\ms{A})$, pues de no ocurrir esto, existe $H \subseteq \ms{A}$ finito tal que $B \subseteq \midcup H$. Pero como $H$ es finito, $B \subseteq^* B \cap \midcup H = \midcup \{ h \cap B \tq h \in H \} =^* \emptyset$, lo cual es imposible. Por lo tanto $B \in \ms{I}^+(\ms{A})$ y $\ms{A} \upharpoonright B$ es finita. 
	\end{proof}
	
	Y a consecuencia de \ref{prop-CaracMADIdeal} y \ref{prop-CaracMADPositiv} se obtiene:
	
	\begin{corolario}\label{cor-MADPositivCarac}
		Sean $N$ un conjunto numerable y $\ms{A}$ una familia casi ajena en $N$. Entonces $\ms{A}$ es maximal si y sólo si se da la igualdad:
		$$ \ms{I}_N^+(\ms{A})=\{ X \in [N]^\omega \tq \ms{A} \upharpoonright X \neq^* \emptyset\} $$
		Si $\ms{A}$ no es maximal, la contención directa de la igualdad anterior falla.
	\end{corolario}
	
	\section{Resultados en combinatoria infinita}

	\subsection{Teorema de Simon}\label{subsec-Simon}
	\label{Sec-TeoSimon}

	La siguiente observación, y el subsecuente Lema, configuran la antesala para enunciar uno de los tres resultados más importantes que figuran en esta sección.
	
	\begin{observacion}
		Sea $(X_n)_{n\in \omega} \subseteq [\omega]^\omega$ una sucesión decreciente respecto $\subseteq$, entonces existe $Y \in [X_0]^\omega$ tal que si $k \in \omega$, se da $Y \subseteq^* X_k$.
	
		Si $n \in \omega$, $\{y_m \tq m<n\} \subseteq \omega$ tiene exactamente $n$ elementos y para cada $m<n$ se tiene $y_m \in X_m$; entonces $X_n \setminus \{y_m \tq m<n\}$ es infinito y se puede fijar $y_n \in X_n \setminus \{y_m \tq m<n\}$.
		
		Así, $Y:=\{y_n \tq n \in \omega\} \subseteq X_0$ es infinito y si $k \in \omega$, entonces $Y \setminus X_k \subseteq \{y_n \tq n<k\}=^* \emptyset$, esto es, $Y \subseteq^* X_k$.
	\end{observacion}
	
	\begin{lema}[Dočkálková]\index[alph]{Dočkálková!Lema de}\index[alph]{Lema!de Dočkálková}\label{lem-PositivCadenaDecreciente}
		Sean $\ms{A} \in \Mad(\omega)$ y $(X_n)_{n\in\omega} \subseteq \ms{I}^+(\ms{A})$ decreciente respecto $\subseteq$. Entonces existe $Y \in \ms{I}^+(\ms{A})$ tal que si $n\in \omega$, entonces $Y \subseteq^* X_n$.
	\end{lema}
	
	\begin{proof}
		Se construirán por recursión, dos conjuntos enumerados inyectivamente; $\{Y_n \tq n\in \omega\} \subseteq [\omega]^\omega$ y $\{A_n \tq n \in \omega\} \subseteq \ms{A}$, tales que para cada $n \in \omega$ se cumple que $Y_n \subseteq X_n$; y, para cada $k\in \omega$, $Y_n \subseteq^* X_k$; y, $Y_n \cap A_n$ es infinito.
	
		Sea $n \in \omega$ y supóngase que los conjuntos $\{Y_m \tq m<n\} \subseteq [\omega]^\omega$ y $\{A_m \tq m<n\} \subseteq \ms{A}$ tienen exactamente $n$ elementos y son tales que si $m<n$, se satisface: $Y_m \subseteq X_m$; si $k \in \omega$, entonces $Y_m \subseteq^* X_k$; y, $Y_m \cap A_m$ es infinito.
		
		A consecuencia de que el conjunto $\{A_m \tq m<n\} \subseteq \ms{A}$ es finito, resulta que $B:=\midcup \{A_m \tq m<n\} \in \ms{I}^+(\ms{A})$ y debido a ello:
		$$ (X_{n+k}\setminus B)_{k \in \omega} \subseteq \ms{I}^+(\ms{A}) \subseteq [\omega]^\omega $$
		es una sucesión $\subseteq$-decreciente. Utilizando la Observación previa, fíjese (con $\Ac$) un conjunto infinito $Y_n \subseteq X_n$ tal que para cada $k \in \omega$ ocurre $Y_n \subseteq^*X_{n+k} \setminus B$ y nótese que por ser $(X_n)_{n \in \omega}$ sucesión $\subseteq$-decreciente, $Y \subseteq^* X_k$.
		
		Como $Y_n \subseteq X_n \setminus B \subseteq \omega$ es infinito y $\ms{A}$ es maximal, se puede fijar (de nuevo, con $\Ac$) un elemento $A_n \in \ms{A}$ tal que $Y_n \cap A_n$ es infinito. Nótese que de la Definición de $B$ y de $Y_n \cap B = \emptyset$, se desprende tanto que $Y_n \notin \{Y_m \tq m<n\}$, como que $A_n \notin \{A_m \tq m<n\}$; pues para cada $m<n$ se tiene que $Y_m \cap B$ es infinito, a razón de que $A_m \subseteq B$ y de que $A_m \cap Y_m$ es infinito, finalizando la construcción recursiva.
		
		Sea $Y:=\midcup\{Y_n \tq n\in \omega\}$ y obsérvese que $\ms{A} \upharpoonright Y$ es infinita, pues cada $A_n$ tiene intersección finita con $Y$. Luego, seguido de la \autoref{prop-CaracMADPositiv} y de que $\ms{A}$ es maximal, se obtiene $Y \in \ms{I}^+(\ms{A})$. Por otro lado, si $n \in \omega$ es cualquiera, entonces:
		\begin{align*}
		Y \setminus X_n & = \bigcup_{m<n} (Y_m \setminus X_n) \cup \bigcup_{m \geq n} (Y_m \setminus X_n) = \bigcup_{m<n} (Y_m \setminus X_n) =^* \emptyset
		\end{align*}
		pues si $m<n$ entonces $Y_m \subseteq^* X_n$, y si $m\geq n$, entonces se dan las contenciones $Y_m \subseteq X_m \subseteq X_n$. Finalizando la prueba. 
	\end{proof}
	
	El siguiente resultado fue demostrado en 1980 por Petr Simon \cite[p.~751]{SimonFrechet} y tiene consecuencias importantes en topología general (véase el \autoref{cor-FrechNoProd}).
	
	\begin{teorema}[Simon]\index[alph]{Simon!Teorema de}\index[alph]{Teorema!de Simon}\label{Teo-Simon}
		Para toda familia maximal e infinita $\ms{A}$ existe un elemento $X \in \ms{I}^+(\ms{A})$ tal que la familia $\ms{A} \upharpoonright X$ es maximal en $X$ y además es unión ajena de dos familias maximales en ninguna parte.
	\end{teorema}
	
	\begin{proof}
		Por contradicción, supóngase que $\ms{A}$ es una familia maximal infinita la cual, sin perder generalidad, la podemos suponer definida sobre $\omega$, tal que para cada $X \in \ms{I}^+(\ms{A})$ o bien $\ms{A} \upharpoonright X$ no es maximal en $X$, o bien, si $\ms{A} \upharpoonright X$ es unión ajena de $\ms{B}$ y $\ms{C}$, entonces $\ms{B}$ o $\ms{C}$ es maximal en alguna parte.
	
		Como $|\ms{A}| \leq \mathfrak{c}$, existe $F \subseteq 2^\omega$ tal que $\ms{A}$ se puede enumerar inyectivamente como $\ms{A}=\{A_f \tq f \in F\}$. Dados $n \in \omega$ y $k \in 2$, defínase:
		$$ \ms{A}(n,k)=\{A_f \in \ms{A} \tq f(n)=k \} $$
		y nótese que para todo $m$ natural, $\ms{A}$ unión ajena de $\ms{A}(m,0)$ y $\ms{A}(m,1)$.
		
		Constrúyanse; por recursión en $\omega$, las sucesiones $(X_n)_{n\in\omega} \subseteq \ms{I}^+(\ms{A})$ y $(k_n)_{n\in \omega} \subseteq 2$, tales que si $n \in \omega$, se da $X_{n+1} \in \ms{I}_{X_n}^+ ( \ms{A}(0,k_n) \upharpoonright X_n )$ y $\ms{A}(n,k_n) \upharpoonright X_{n+1} \in \Mad(X_{n+1})$.
		
		Como $\ms{A}$ es familia maximal e infinita, defínase $X_0:=\omega \in \ms{I}^+(\ms{A})$ (véase \ref{cor-IdealPropioCaract}); así que $\ms{A} = \ms{A} \upharpoonright X_0$ es maximal sobre $X_0$. Como $\ms{A}$ es unión ajena de $\ms{A}(0,0)$ y $\ms{A}(0,1)$, se sigue de la hipótesis la existencia de un elemento $k_0 \in 2$ tal que $\ms{A}(0,k_0)$ es maximal en ninguna parte. Con $\Ac$, fíjese un elemento $X_1 \in \ms{I}^+(\ms{A}(0,k_0)) = \ms{I}_{X_0}^+(\ms{A}(0,k_0) \upharpoonright X_0)$ de forma tal que $(\ms{A}(0,k_0) \upharpoonright X_0) \upharpoonright X_1 = \ms{A}(0,k_0) \upharpoonright X_1$ sea maximal en $X_1$. Como $X_1 \notin \ms{I}_{X_0}(\ms{A}(0,k_0) \upharpoonright X_0)$ y $X_1 \subseteq X_0$, se desprende de \autoref{prop-TrazaBasicos} que $X_1 \in \ms{I}_{X_1}^+ (\ms{A}(0,k_0) \upharpoonright X_1)$, por lo que $\ms{A}(0,k_0) \upharpoonright X_1$ es infinita, en virtud de su maximalidad y del \autoref{cor-IdealPropioCaract}. Así, $\ms{A} \upharpoonright X_1$ es infinita, lo cual implica que $X_1 \in \ms{I}^+(\ms{A})$, pues $\ms{A}$ es maximal (véase \ref{cor-MADPositivCarac}).
		
		Supóngase ahora que $n \in \omega$ y que $X_n, X_{n+1} \in \ms{I}^+(\ms{A})$ y $k_n \in 2$ son tales que $X_{n+1} \in \ms{I}_{X_n}^+ ( \ms{A}(n,k_n) \upharpoonright X_n )$ de modo que $\ms{A}(n,k_n) \upharpoonright X_{n+1}$ es maximal en $X_{n+1}$. Como $\ms{A}$ unión ajena de $\ms{A}(n+1,0)$ y $\ms{A}(n+1,1)$, $\ms{A} \upharpoonright X_{n+1}$ es unión ajena de $\ms{A}(n+1,0) \upharpoonright X_{n+1}$ y $\ms{A}(n+1,1) \upharpoonright X_{n+1}$. De nuevo, con $\Ac$ fíjense $k_{n+1}\in 2$ y $X_{n+2} \in \ms{I}_{X_{n+1}}^+ ( \ms{A}(n+1,k_{n+1}) \upharpoonright X_{n+1} )$ de modo tal que $\ms{A}(n+1,k_{n+1}) \upharpoonright X_{n+2} \in \Mad(X_{n+2})$. Al igual que antes, se obtiene de \ref{cor-IdealPropioCaract} y \ref{cor-MADPositivCarac}, que $X_{n+2} \in \ms{I}^+(\ms{A})$, lo que finaliza la construcción recursiva.
		
		Por construcción, $(X_n)_{n\in \omega} \subseteq \ms{I}^+(\ms{A})$ es $\subseteq$-decreciente, por lo que del Lema de Dočkálková, existe un conjunto $Y \in \ms{I}^+(\ms{A})$ tal que para cada $n \in \omega$ se cumple $Y \subseteq^* X_n$. Puesto que $Y \in \ms{I}^+(\ms{A})$ y $\ms{A}$ es maximal, de \ref{cor-MADPositivCarac} se tiene que $\ms{A} \upharpoonright Y$ es infinita, y con ello, existe $g \in F \setminus \{(k_n)_{n \in \omega}\} \subseteq 2^\omega$ tal que $A_g \cap Y$ es infinito. Siendo $k$ distinta de $g$, hay un natural $m$ tal que $k_m \neq g(m)$.
		
		Como $Y \subseteq^* X_{m+1}$, entonces $Y \setminus X_{m+1}$ es finito. Luego, derivado de que $Y \cap A_g$ es infinito, se obtiene que $A_g \cap X_{m+1} \subseteq X_{m+1}$ es infinito. Así, por la maximalidad de $\ms{A}(m,k_m) \upharpoonright X_{m+1}$ en $X_{m+1}$ se obtiene un $A_f \in \ms{A}(m,k_m)$ tal que $(A_g \cap X_{m+1}) \cap (A_f \cap X_{m+1})$ es infinito. Sin embargo, lo anterior conduce a una contradicción, pues $f(m)=k_m \neq g(m)$ implica que $f \neq g$, y esto a su vez, que $A_f \cap A_g$ es finito por ser $\ms{A}$ familia casi ajena. 
	\end{proof} 
		
	\begin{corolario}
		Existe una familia maximal de tamaño $\mathfrak{c}$ que es unión ajena de dos familias maximales en ninguna parte.
	\end{corolario}

	\subsection{Grietas y familias de Luzin}
	\label{Sec-Luzin}

	\begin{definicion}\label{Def-particionador}\label{def-grieta}\index[alph]{particionador}\index[alph]{grieta}\index[alph]{grieta!separada}\index[alph]{grieta!contenida en una familia}\index[alph]{familia!que contiene a una grieta}
		Sea $N$ un conjunto numerable y $\ms{A},\ms{B} \in \Ad(N)$. 
		\begin{enumerate}
			\item El par $(\ms{A},\ms{B})$ es una \textbf{grieta} si y solamente si $\ms{A} \cap \ms{B}=\emptyset$ y $\ms{A} \cup \ms{B} \in \Ad(N)$. Se suele decir que $(\ms{A},\ms{B})$ \textbf{está contenida} en $\ms{A} \cup \ms{B}$, o que $\ms{A} \cup \ms{B}$ \textbf{contiene a} $(\ms{A},\ms{B})$.
			\item Un subconjunto $D \subseteq N$ es \textbf{particionador} de $\ms{A}$ y $\ms{B}$ si y sólo si para cada $A \in \ms{A}$ y $B \in \ms{B}$ se tiene $A \subseteq^* D$ y $B \cap D =^* \emptyset$.
			\item Una grieta $(\ms{A},\ms{B})$ está \textbf{separada} si y sólo si existe un particionador de $\ms{A}$ y $\ms{B}$.
		\end{enumerate}
	\end{definicion}
	
	En términos de la definición anterior, no resulta complicado notar que $D$ es particionador de $\ms{A}$ y $\ms{B}$ si y sólo si $N \setminus D$ es particionador de $\ms{B}$ y $\ms{A}$. Además $\ms{A}=\{X \in \ms{A} \cup \ms{B} \tq X \subseteq ^* D\}$ y $\ms{B}=\{X \in \ms{A} \cup \ms{B} \tq X \cap D =^* \emptyset\}$.

	Como ha resultado ser rutina a lo largo de todo el capítulo, se deberá hacer hincapié en el comportamiento de las grietas respecto a las biyecciones $\Phi_h$. La demostración de este hecho resulta estándar.

	\begin{proposicion}\label{prop-grietasBiyec}
		Sean $N$ y $M$ conjuntos numerables. Para toda biyección $h:N \to M$ y toda grieta $(\ms{A},\ms{B})$ en $N$ se cumple:
		\begin{enumerate}
			\item $(\Phi_h(\ms{A}),\Phi_h(\ms{B}))$ es grieta en $M$.
			\item $(\ms{A},\ms{B})$ está separada si y sólo si $(\Phi_h(\ms{A}),\Phi_h(\ms{B}))$ está separada.
		\end{enumerate}
	\end{proposicion}

	En virtud de lo anterior, cada vez que $(\ms{A},\ms{B})$ sea grieta; y salvo que se diga lo contrario, se dará por sentado que $\ms{A},\ms{B} \in \Ad(\omega)$

	\begin{observacion}\label{obs-GrietasSimple}
		Para cualesquiera grietas $(\ms{A},\ms{B})$ y $(\ms{A}',\ms{B}')$:
		\begin{enumerate}
			\item $\ms{A} \subseteq \ms{I}^+(\ms{B})$ y $\ms{B} \subseteq \ms{I}^+(\ms{A})$ (seguido de \autoref{obs-IdealPrevia}).
			\item $(\ms{A},\ms{B})$ está separada si y sólo si $(\ms{B},\ms{A})$ está separada.
			\item Si $(\ms{A}',\ms{B}')$ está separada, entonces $(\ms{A},\ms{B})$ está separada.
		\end{enumerate}
	\end{observacion}

	A continuación se dan dos hechos básicos sobre la separación de grietas; y sin bien se podrían exponer las correspondientes demostraciones disponiendo solamente de la utilería dada hasta el momento, se dejarán a modo de corolario de la teoría resultante de los $\Psi$-espacios (véase \ref{col-tra-interrelacion}).

	\begin{ejemplo}\label{ej-interrelacion}
		Sea $\ms{C}\in \Ad(\omega)$, entonces:
        \begin{enumerate}
            \item Si $|\ms{C}|\leq \aleph_0$, entonces cualquier grieta contenida en $\ms{C}$ está separada.
            \item Si $\ms{C}$ es inifnita y, $|\ms{C}|=\mathfrak{c}$ o $\ms{C}\in \Mad(\omega)$; entonces $\ms{C}$ contiene una grieta que no está separada.
        \end{enumerate}
	\end{ejemplo}

	El siguiente tipo de familias poseen virtudes que las convierten en objetos canónicos dentro de la teoria  de conjuntos.

	\begin{definicion}\label{def-LuzinFam}\index[alph]{familia!de Luzin}\index[alph]{Luzin!familia de}
		Una \textbf{familia de Luzin} es una familia casi ajena $\ms{A}=\{A_\alpha \tq \alpha \in \omega_1 \}$ de manera que para cada $\alpha \in \omega_1$ y $n \in \omega$, el conjunto $ \{ \beta<\alpha \tq A_\alpha \cap A_\beta \subseteq n \} $ es finito.
	\end{definicion}

	La idea detrás de que $\ms{A}=\{A_\alpha \tq \alpha \in \omega_1 \}$ sea de Luzin es que; fijando $\alpha \in \omega_1$, para cada $D \subseteq \alpha$ infinito, $A_\alpha \cap \midcup\{A_\beta \tq \beta \in D\}$ es infinito. Esto se debe a que si $n \in \omega$, entonces $D \setminus \{ \beta<\alpha \tq A_\alpha \cap A_\beta \subseteq n \} $ es infinito, particularmente no vacío.

	\begin{proposicion}\label{pro-LuzinExisten}
		Toda familia casi ajena numerable se extiende a una familia de Luzin. Particularmente, existe una familia de Luzin.
	\end{proposicion}
	\begin{proof}
		Sea $\ms{B}=\{A_n \tq n \in \omega\}$ cualquier familia casi ajena numerable y nóntese que claramente para cualesquiera $m,n \in \omega$, el conjunto $ \{ k<m \tq A_m \cap A_k \subseteq n \} $ es finito.
	
		Por recursión sobre $\omega_1 \setminus \omega$, sea $\gamma \in \omega_1 \setminus \omega$ cualquiera y supóngase $\{A_\alpha \tq \alpha \in \gamma\}$ es una familia casi ajena tal que, si $\alpha<\gamma$ y $n \in \omega$, el conjunto $ \{ \beta<\alpha \tq A_\alpha \cap A_\beta \subseteq n \} $ es finito.

		Como $\gamma \in \omega \setminus \omega_1$, $\gamma$ es numerable y se puede enumerar $\{A_\alpha \tq \alpha \in \gamma\}$ como $\{B_n \tq n\in \omega\}$. Por ser tal, una familia casi ajena, cada conjunto $C_n:=B_n \setminus \midcup\{ B_j \tq j<n \}$ es infinito (corrobórese ésto en la demostración de \ref{prop-MADnoNum}). Para cada $n \in \omega$ fíjese $a_n \in [C_n]^n$ y defínase:
		$$ A_\gamma:=\midcup\{a_m \tq m \in \omega\} $$
		
		Nótese que si $n \neq m$, entonces $a_n \cap a_m = \emptyset$. De este modo, si $n \in \omega$ es cualquiera, resulta que $A_\gamma \cap B_n = a_n \cap B_n = a_n$ es finito. Más aún, como $a_n$ tiene exatamente $n$ elementos, $n \leq \max(A_n)$; y consecuentemente, si $m \in \omega$ y $A_\gamma \cap B_n \subseteq m$, entonces $n \leq m$.
		
		Lo anterior prueba, no sólo que $\{A_\alpha \tq \alpha \ \leq \gamma\}$ es familia casi ajena, sino que para cualesquiera $\alpha\leq \gamma$ y $n \in \omega$, el conjunto $ \{ \beta<\alpha \tq A_\alpha \cap A_\beta \subseteq n \} $ es finito. Lo cual finaliza la construcción por recursión de los conjuntos $A_\alpha$ (con $\omega \leq \alpha<2$); es claro que para $\ms{A}:=\{A_\alpha \tq \alpha \in \omega_1 \}$ es una familia Luzin que extiende a $\ms{B}$.
	\end{proof}

	\index[alph]{familia!inseparable}\label{def-FamInseparable}
	Cualquier familia de Luizn cumplirá que ninguna grieta formada por sus subconjuntos más que numerables está separada; es decir, es una \textit{familia inseparable} (véase \cite[\S~ 3.2]{hruAlmost}).

	Obsérvese que si $\ms{B}$ y $\ms{C}$ son familias casi ajenas de modo que $\midcup \ms{B} \cap \midcup \ms{C}$ es finito, entonces $\midcup \ms{B}$ es separador de $\ms{B}$ y $\ms{C}$, así $(\ms{B},\ms{C})$ está separada. Pese a no ocurrir el recíproco de lo anterior, se configura la siguiente caracterización.

	\begin{lema}
		Sea $\ms{A}\in \Ad(\omega)$, entonces $\ms{A}$ es inseparable si y sólo si para cualesquiera $\ms{B},\ms{C} \in [\ms{A}]^{\omega_1}$ ajenos, $\midcup \ms{B} \cap \midcup \ms{C}$ es infinito.
	\end{lema}
	\begin{proof}
		Por la discusión previa, basta sólo probar la necesidad. 
		
		Por contrapuesta, supóngase que $\ms{B},\ms{C} \in [\ms{A}]^{\omega_1}$ son tales que existe $D \subseteq \omega$, particionador de $\ms{B}$ y $\ms{C}$. Entonces las asignaciones $\ms{B} \to \omega$ y $\ms{C} \to \omega$; dadas por $b \mapsto \max(b \setminus D)$ y $c \mapsto \max(c \cap D)$ están bien definidas. Pero en vista de que $|\ms{B}|=|\ms{C}|=\omega_1$, éstas no pueden ser inyectivas, y existen $m,n \in \omega$ de modo que $\ms{B}':=\{b \in \ms{B} \tq b \setminus D \subseteq m\}$ y $\ms{C}':=\{b \in \ms{C} \tq c \cap D \subseteq n\}$ tienen tamaño $\omega_1$.
		
		Además $\midcup \ms{B}' \setminus D \subseteq m =^* \emptyset$ y $\midcup \ms{C}' \cap D \subseteq n =^* \emptyset$, por lo que $\midcap \ms{B} \subseteq^* D$, $\midcap \ms{C} \subseteq^* \omega \setminus D$, y así, $\midcup \ms{B}' \cap \midcup \ms{C}' \subseteq^* D \cap (\omega \setminus D) = \emptyset$.
	\end{proof}

	\begin{proposicion}\label{prop-LuzinSeparadas}
		Cualquier familia Luzin es inseparable.
	\end{proposicion}
	\begin{proof}
		Sean $\ms{A}=\{A_\alpha \tq \alpha \in \omega_1\}$ cualquier familia de Luzin y $\ms{B}=\{A_\alpha \tq \alpha \in B\},\ms{C}=\{A_\alpha \tq \alpha \in C\} \subseteq \ms{A}$ no numerables y ajenos. Como $C$ es infinito, existe $\alpha \in \omega_1$ de manera que $C \cap \alpha$ es infinito. Nótese que $B$ es cofinal en $\omega_1$, por ser $\omega_1$ regular; así que sin pérdida de generalidad, supóngase $\alpha \in B$.
		
		En virtud de los comentarios posteriores a la \autoref{def-LuzinFam}, se tiene que $A_\alpha \cap \midcup \{A_\beta \tq \beta \in C \cap \alpha \}$ es infinito, demostrando que $\midcup \ms{B} \cap \midcup \ms{C}$ es infinito. Se concluye del lema previo que $\ms{A}$ es inseparable.
	\end{proof}

	\subsection{Lema de Solovay}

	\begin{definicion}\label{def-ordenBasado} \index[alph]{orden!basado en $\ms{A}$}\index[sym]{$\mathbb{P}_\ms{A}$}\index[sym]{$\leq_\ms{A}$}
		Sea $\ms{A}$ una familia casi ajena. El par ordenado $\mathbb{P}_\ms{A}:=([\omega]^{<\omega} \times [\ms{A}]^{<\omega}, \leq_\ms{A})$; donde $(p,P) \leq_\ms{A} (h,H)$ si y sólo si $h \subseteq p$, $H \subseteq P$ y $p \setminus h \subseteq \omega \setminus \midcup H$, se denomina \textbf{orden basado en} $\ms{A}$. Cuando el contexto sea claro, se escribirá $\leq$ en vez de $\leq_\ms{A}$.
	\end{definicion}

	\begin{proposicion}
		Si $\ms{A} \in \Ad(\omega)$, entonces $\mathbb{P}_\ms{A}$ es un conjunto parcialmente ordenado.
	\end{proposicion}

	\begin{proof}
		Claramente el orden basado en $\ms{A}$; $\leq$, es una relación reflexiva y antisimétrica. Supóngase que $(p,P) \leq (h,H)$ y $(h,H) \leq (k,K)$. Dada \ref{def-ordenBasado}, $k \subseteq h \subseteq p$ y $K \subseteq H \subseteq P$; en consecuencia $k \subseteq p$ y $K \subseteq P$.
		
		Y como además $p \setminus h \subseteq \omega \setminus \midcup H$ y $h \setminus k \subseteq \omega \setminus \midcup K$, resulta que:
		$$ p \setminus k \subseteq (h \setminus k) \cup (p \setminus h) \subseteq \big( \omega \setminus \midcup K \big) \cup \big( \omega \setminus \midcup H \big) $$
		mostrando $p \setminus k \subseteq \omega \setminus \midcup K$, por lo que $\leq$ es transitva. 
	\end{proof}

	En términos informales, $(p,P) \leq (h,H)$ significa que ``$h$ se extiende a $p$ y $H$ a $P$''. Conforme $H \subseteq \ms{A}$ crece, se aproxima a $\ms{A}$. Dado que, conforme $h$ crece, éste se acerca a un subconjunto casi ajeno con $\midcup H$; eventualmente, se formará un subconjunto casi ajeno con $\midcup \ms{A}$.

	\begin{consideracion}\index[sym]{$D_a$ (si $a \in \ms{A}$)}\index[sym]{$D_G$ (si $\mathcal{G} \subseteq \mathbb{P}_\ms{A}$)}
		En lo que resta de la subsección:
		\begin{enumerate}
			\item Para cada $a \in \ms{A}$, $ D_a:=\{ (p,P) \in \mathbb{P}_\ms{A} \tq a \in P\} $.
			\item Si $\mathcal{G} \subseteq \mathbb{P}_\ms{A}$, $ D_\mathcal{G}:=\midcup\{ h \subseteq \omega \tq \exists H \in [\omega]^{\omega} \: \big( (h,H) \in \mathcal{G} \big) \} $.
		\end{enumerate}
	\end{consideracion}

	\begin{lema}\label{lem-DgMagia}
		Sean $\ms{A}$ una familia casi ajena, y $\mathcal{G}$ un filtro de $\mathbb{P}_\ms{A}$, entonces para cada $a \in \ms{A}$:
		\begin{enumerate}
			\item $D_a$ es denso en $\mathbb{P}_\ms{A}$.
			\item Si $\mathcal{G} \cap D_a \neq \emptyset$; entonces, $D_\mathcal{G} \cap a$ es finito.
		\end{enumerate}
	\end{lema}

	\begin{proof}
		(i) Si $a \in \ms{A}$ y $(p,P) \in \mathbb{P}_\ms{A}$ son elementos arbitrarios, entonces $(p,P\cup\{a\}) \in D_a$ y además es inmediato a la \autoref{def-ordenBasado} que $(p,P\cup\{a\}) \leq (p,P)$.

		(ii) Supóngase que $(p,P) \in \mathcal{G} \cap D_a$ y sea $x \in D_\mathcal{G} \cap a$ cualquier elemento. Por definción de $D_\mathcal{G}$, existe $(h,H) \in \mathcal{G}$ de modo que $x \in h$. Y por ser $\mathcal{G}$ filtro, $(k,K) \leq (p,P),(h,H)$ para cierto $(k,K) \in \mathcal{G}$. De esto, particularmente se obtiene que $h \subseteq k$, $k \setminus p \subseteq \omega \setminus \midcup P$.

		Ahora, como $a \in P$ (pues $(p,P)\in D_a$), se tiene que $x \in \midcap P$. Además, $x \in h \subseteq k$, así que $x \in k \cap \midcup P$, lo cual obliga a que $x \in p$. Por tanto $D_\mathcal{G} \cap a \subseteq p =^* \emptyset$. 
	\end{proof}

	\begin{corolario}\label{cor-SolovayDebil}
		Sean $\ms{A} \in \Ad(\omega)$ y $\ms{D}:=\{D_a \tq a \in \ms{A}\}$. Si existe un filtro $\ms{D}$-genérico, $\ms{A}$ no es maximal.
	\end{corolario}

	Debido a lo recién mostrado, de tener $\mathbb{P}_\ms{A}$ la \textit{c.c.c.} (ver \textbf{AAA}), se satisfaría que $\Ma(|\ms{A}|)$ implica $\ms{A} \notin \Mad(\omega)$.

	Y en efecto, si $\mathcal{A} \subseteq \mathbb{P}_\ms{A}$ es anticadena y $(p,P),(h,H) \in \mathcal{A}$, se tiene $p\neq h$; sino $(p,P \cup H) \leq (p,P),(h,H)$ y $\mathcal{A}$ dejaría de ser anticadena. En consecuencia $|\mathcal{A}|\leq|[\omega]^{<\omega}|=\aleph_0$ y $\mathbb{P}_\ms{A}$ tiene la \textit{c.c.c.}

	\begin{corolario}\label{cor-MaSimple}
		Si $\kappa$ un carinal con $\omega \leq \kappa <\mathfrak{c}$; bajo $\Ma(\kappa)$, se tiene $\Mad(\omega) \subseteq \left[[\omega]^\omega\right]^{>\kappa}$; y por ello $\mathfrak{a}>\kappa$.
		Consecuentemente:
		\begin{enumerate}
			\item $\zfc \vdash \mathfrak{m} \leq \mathfrak{a}$ (recuérese \textbf{Def m}).
			\item $\zfc + \Ma \vdash \mathfrak{a}=\mathfrak{c}$.
			\item $\mathfrak{a} = \mathfrak{c}$ es estrictamente más debil que $\HC$.
		\end{enumerate}
	\end{corolario}

	\begin{proof}
		Únicamente falta verificar (iii). Basta tener en cuenta que $\Ma + \lnot \HC$ es consistente con $\zfc$ (consúltese \cite[p.~279-281]{kunenSet}); así que de (iii), se obtiene que $\zfc + \Ma + \lnot \HC \vdash \mathfrak{a}=\mathfrak{c}$. Por ende, $\zfc + \mathfrak{a}=\mathfrak{c} \not\vdash \HC$. 
	\end{proof}

	El \autoref{cor-SolovayDebil} es una inmediatez, dada toda su discusión previa. Una versión bastante más fortalecida de éste, es el siguiente resultado mostrado por Robert Solovay.

	\begin{lema}[Solovay]\label{lem-Solovay}\index[alph]{Lema!de Solovay}\index[alph]{Solovay!Lema de}
		Sea $\kappa$ un cardinal de modo que $\omega \leq \kappa < \mathfrak{c}$. Bajo $\Ma(\kappa)$; para toda grieta $(\ms{A},\ms{B})$; con $|\ms{A}|,|\ms{B}|\leq \kappa$, existe $D\subseteq \ms{A}$ tal que para cada $A \in \ms{A}$ y $b \in \ms{B}$, $a \cap D=^*\emptyset$ y $b \cap D\neq ^*\emptyset$.
	\end{lema}

	\begin{proof}
		Supóngase $\Ma(\kappa)$ y sea $(\ms{A},\ms{B})$ una grieta de forma que $|\ms{A}|,|\ms{B}|\leq \kappa$. Para cualesquiera $b \in \ms{B}$ y $n \in \omega$, defínase el conjunto $D(b,n):=\{ (h,H) \in \mathbb{P}_\ms{A} \tq h \cap b \not\subseteq n \}$.

		Cada $D(b,n)$ es denso en $\mathbb{P}_\ms{A}$. Sea $(p,P) \in \mathbb{P}_\ms{A}$ cualquiera; por \ref{obs-GrietasSimple}, $b \notin \ms{I}(\ms{A}) $, luego $b \setminus \midcup P$ es infinito. Por ello, existe $m \in \omega$ de modo que $n+1 \in m$ y $m \in b \setminus \midcup P \subseteq \omega \setminus \midcup P$; así, $p \cup \{m\}$ es finito, $(p \cup \{m\}, P) \in D(b,n)$ y $(p \cup \{m\}, P) \leq_\ms{A} (p,P)$.

		Sea $\ms{D}=\{ D(b,n) \tq (b,n) \in \ms{B} \times \omega \} \cup \{D_a \tq a \in \ms{A} \}$ y obsérvese que $\ms{D}$ es una familia de densos de $\mathbb{P}_\ms{A}$ de cardinalidad menor o igual a $\kappa$. Como $\mathbb{P}_\ms{A}$ es \textit{c.c.c.}, de $\Ma(\kappa)$ se desprende la existencia de un filtro $\mathcal{G}$ en $\mathbb{P}_\ms{A}$, $\ms{D}$-genérico. Se afirma que $D_\mathcal{G}$ es el conjunto buscado.

		En efecto, por \ref{lem-DgMagia} se tiene que para cada $a \in \ms{A}$, el conjunto $D_\mathcal{G} \cap a$ es finito. Ahora, si $b \in \ms{B}$ es cualquiera, para cada $n \in \omega$ existe $(k,K) \in \mathcal{G} \cap D(b,n)$; y en consecuencia $D_\mathcal{G} \cap b \not \subseteq n$ (pues $h \cap b \not \subseteq n$). Por lo que el conujunto $D_\mathcal{G} \cap b$ no puede ser finito. 
	\end{proof}

	Sería deseable que la conclusión del Lema de Solova fuese que la grieta $(\ms{A},\ms{B})$ está separada; sin embargo tal formulación desenvoca en un resultado falso.
	
	Bajo $\Ma + \lnot\HC$, la existencia de una familia de Luzin (probada en \ref{prop-LuzinSeparadas}) sería testigo de tal falsedad. Se puede decir que si una grieta $(\ms{A},\ms{B})$ satisface la conclusión de \ref{lem-Solovay}, entonces está ``débilmente separada'' (véase \cite[\S~ 3.2]{hruAlmost}), si $D$ es como en la conclusión de \ref{lem-Solovay}.


	\begin{corolario}
		Sea $\kappa$ un carinal con $\omega \leq \kappa <\mathfrak{c}$, entonces bajo $\Ma(\kappa)$, se tiene $2^\kappa=\mathfrak{c}$.

		Consecuentemente, es consistente con $\zfc$ que $2^{\aleph_0}=2^{\aleph_1}$
	\end{corolario}

	\begin{proof}
		Sea $\kappa$ un carinal con $\omega \leq \kappa <\mathfrak{c}$ y supóngase $\Ma(\kappa)$. Tomando en cuenta \ref{cor-famGrandes}, fíjese una familia casi ajena $\ms{A}$ con $|\ms{A}|=\kappa$ y defínase $f:\ms{P}(\omega) \to \ms{P}(\ms{A})$ como $ f(X)=\{ b \in \ms{A} \tq b \cap X =^* \emptyset \} $.

		Si $\ms{B} \subseteq \ms{A}$ es cualquiera, entonces $|\ms{A} \setminus \ms{B}|, |\ms{B}| \leq \kappa$ y por el Lema de Solovay (\ref{lem-Solovay}), existe un particionador $D \subseteq \omega$ para $\ms{A} \setminus \ms{B}$ y $\ms{B}$, resultando en que $ f(D)=\{b \in \ms{A} \tq b \cap X =^* \emptyset \} = \ms{B} $. Luego $f$ es sobreyectiva y $\mathfrak{c} \geq 2^\kappa $. Como además $\kappa \geq \aleph_0$, etnonces $2^\kappa \geq 2^{\aleph_0}=\mathfrak{c}$.

		Para la segunda parte, $\lnot \HC + \Ma$ es consistente con $\zfc$. Y como $\lnot \HC$ y $\Ma$ implican $\Ma(\aleph_1)$ y $\omega \leq \aleph_1 < \mathfrak{c}$, se tiene por consiguiente que $\zfc+\lnot \HC + \Ma \vdash 2^{\aleph_0}=2^{\aleph_1}$.
	\end{proof}

	Como se podrá atestiguar posteriormente en la tesis, los enunciados $2^{\aleph_1} = 2^{\aleph_0}$ y su negación; $2^{\aleph_1} > 2^{\aleph_0}$ tienen efectos notables en la topología general. Especialmente, se mostrarán sus efectos sobre la Conjetura de Moore (\autoref{Sec-PDM}).
        \chapter{Espacios de Mrówka}\label{chap-mrowkas}
	\emph{\small Los $\Psi$-espacios; o espacios de Mrówka, cuentan con un lugar privilegiado en la topología de conjuntos; esto se debe a que son, entre otras cosas, espacios idóneos para la búsqueda de ejemplos. Esta virtud tiene por motivo las múltiples caracterizaciones que existen para sus propiedades topológicas.}
	
	\emph{\small La intención primordial del presente capítulo es presentar aspectos; en primer lugar, cubrir la definición de los $\Psi$-espacios y exhibir sus propiedades topológicas elementales; y en segundo lugar, dar una caracterización para los espacios de Mrówka en términos de propiedades topológicas, el hoy conocido como Teorema de Kannan y Rajagopalan.}

\section{\texorpdfstring{$\Psi$-espacios y caracterizaciones elementales}{Psi-espacios y caracterizaciones elementales}}
	
	Dada $\ms{A} \subseteq [\omega]^\omega$, se satisface que $\omega \cap \ms{A} = \emptyset$. Por cómo se define la topología de Isbell-Mrókwa (en un conjunto $N$, dada $\ms{A}\subseteq[N]^\omega$), es conveniente establecer lo siguiente.
	
	\begin{consideracion}\label{cons-ajenosNyA}
		A partir de ahora, siempre que $N$ sea un conjunto numerable y $\ms{A} \subseteq [N]^\omega$, se asumirá que $N \cap \ms{A} = \emptyset$.
	\end{consideracion}

	En el espacio (de ordinales) $X=\omega+1$, cada punto de $\omega$ es aislado, pero el punto $\omega \in X$; situado ``en la periferia'' de $X$, se mantiene cercano al subconjunto $\omega \subseteq X$ del espacio.

	Los $\Psi$-espacios tienen por conjunto subyacente a $\omega \cup \ms{A}$; y pueden ser vistos como una forma generalizada de $\omega+1$. Se configura su topología de modo que $\omega$ es una masa de puntos aislados, y cada punto $\omega \cup \ms{A}$ ``en la periferia del espacio'' permanece cercano al subconjunto $a \subseteq \omega \cup \ms{A}$ del espacio.

	\begin{proposicion}
		Sean $N$ un conjunto numerable y $\ms{A} \subseteq [N]^\omega$. La siguiente colección es una topología para $N \cup \ms{A}$.\index[sym]{$\tau_{N,\ms{A}}$}
	$$ \tau_{N,\ms{A}} := \{ U \subseteq N \cup \ms{A} \tq \forall x \in U \cap \ms{A} \: ( x \subseteq^* U ) \} $$
	\end{proposicion}
	\begin{proof} 
		Resulta evidente que $\emptyset, N \cup \ms{A} \in \tau_{N,\ms{A}}$. Ahora, dados $U,V \in \tau_{N,\ms{A}}$ y $x \in (U \cap V) \cap \ms{A}$ cualquiera, $x \subseteq^* U$ y $x \subseteq^* V$, de donde $x \subseteq^* U \cap V$ y $U \cap V \in \tau_{N,\ms{A}}$. Finalmente, dados $\mathcal{U}\subseteq \tau_{N,\ms{A}}$ y $x \in \midcup \mathcal{U} \cap \ms{A}$ arbitrarios, existe $U_0 \in \mathcal{U}$ con $x \in U_0$; así que $x \subseteq^* U_0$ y consecuentemente $x \setminus U_0$ es finito. Como $x \setminus \midcup \mathcal{U} \subseteq x \setminus U_0$, resulta que $x \subseteq^* \midcup \mathcal{U}$ y así $\midcup \mathcal{U} \in \tau_{N,\ms{A}}$.
	\end{proof}

	\begin{definicion}\label{Def-Mrowka}\index[alph]{topología!de Mrówka}\index[alph]{topología!de Isbell-Mrówka}\index[alph]{Mrówka! topología de}\index[alph]{Isbell-Mrówka!topología de}\index[sym]{$\tau_\ms{A}$}\index[alph]{$\Psi$-espacio}\index[alph]{espacio!,$\Psi$}\index[sym]{$\Psi_N(\ms{A})$}\index[sym]{$\Psi(\ms{A})$}	
		Sean $N$ un conjunto numerable y $\ms{A} \subseteq [N]^\omega$.
		\begin{enumerate}
			\item La colección $\tau_{N,\ms{A}}$ de la Proposición anterior es la \textbf{Topología de Mrówka (de Isbell-Mrókwa)} \textbf{generada por $\ms{A}$}.
			\item El \textbf{$\Psi$-espacio}\index[alph]{$\Psi$-espacio}\index[alph]{espacio!,$\Psi$} \textbf{generado por $\ms{A}$} se denota por $\Psi_N(\ms{A})$\index[sym]{$\Psi_N(\ms{A})$}, y consta del conjunto $N \cup \ms{A}$ dotado con su topología $\tau_{N,\ms{A}}$.
		\end{enumerate}
		
		 Si $N=\omega$, se denotarán $\tau_\ms{A}:=\tau_{N,\ms{A}}$ y $\Psi(\ms{A})=\Psi_N(\ms{A})$.
	\end{definicion}

	Previo a abordar otros temas, se mostrará por qué; a efectos topológicos, bastará considerar familias de subconjuntos de $\omega$.% para estudiar las propiedades topológicas de los $\Psi$-espacios.

	\begin{proposicion}\label{prop-MrowHomeoBiyec}
		Sean $N,M$ conjuntos numerables, $\ms{A} \subseteq [N]^\omega$ arbitraria y $h:N \to M$ una biyección. Entonces $\Psi_N(\ms{A}) \cong \Psi_M(\Phi_h(\ms{A}))$.
	\end{proposicion}
	\begin{proof} 
		Sea $f:\Psi_N(\ms{A}) \to \Psi_M(\Phi_h(\ms{A}))$ definida por medio de $f(x)=h(x)$ si $x \in N$ y $f(x)=h[x]$ si $x \in \ms{A}$. Nótese que; por la \autoref{cons-ajenosNyA}, $f$ es biyectiva. Además, por definición de $f$, y como $\Psi_h^{-1}=\Phi_{h^{-1}}$, basta verificar únicamente la continuidad de $f$.
	
		Sea $U$ abierto en $\Psi_M(\Phi_h(\ms{A}))$ y supóngase que $x \in f^{-1}[U] \cap \ms{A}$. Entonces $f(x) = h[x] \in U \cap \Phi_h(\ms{A})$. Como $U$ es abierto en $\Psi_M(\Phi_h(\ms{A}))$, entonces $f(x) \setminus U$ es finito. Así que $f^{-1}[f(x) \setminus U] = h^{-1}[h(x)] \setminus f^{-1}[U] = x \setminus f^{-1}[U]$ es finito y así $x \subseteq^* f^{-1}[U]$, probando $f^{-1}[U]$ es abierto en $\Psi_N(\ms{A})$.
	\end{proof} 


	\newpage
	
	La siguiente manera de describir la topología de Mrówka es la más cumún en la literatura (como ejemplo están \cite{hruMrowka} o \cite{hruAlmost}).

	\begin{proposicion}\label{prop-BaseLocMrowka}\index[sym]{$\mathcal{B}_x$}\index[alph]{base!local!estándar de $x$ en $\Psi_N(\ms{A})$}
		Sea $\ms{A} \subseteq [\omega]^\omega$, entonces:
		\begin{enumerate}[i)]
			\item Cada $B \subseteq \omega$ es abierto en $\Psi(\ms{A})$, en particular, cada $n \in \omega$ es punto aislado.
			\item Si $x \in \ms{A}$, el conjunto $\mathcal{B}_x:=\{ \{x\} \cup x \setminus F \tq F \in[x]^{<\omega} \}$ es base local de $x$ en $\Psi(\ms{A})$. $\mathcal{B}_x$ es la \textbf{base local estándar de $x$ en $\Psi_N(\ms{A})$}.
		\end{enumerate}
	\end{proposicion}
	\begin{proof} 
		(i) Si $B \subseteq \omega$, es vacuo que $B \in \tau_\ms{A}$, pues $B \cap \ms{A} = \emptyset$.
	
		(ii) Sea $x \in \ms{A}$, entonces $\mathcal{B}_x \subseteq \tau_\ms{A}$. En efecto, si $G \subseteq x$ es finito y $y \in \big( \{x\} \cup x \setminus G \big) \cap \ms{A}$, necesariamente $y=x$, de donde $y \subseteq^* \{x\} \cup x \setminus G$ pues $G$ es finito, así $\{x\} \cup x \setminus G \in \tau_{\ms{A}}$. Ahora, si $U \subseteq \Psi(\ms{A)}$ es abierto y $x \in U$, $F:= x \setminus U \subseteq x$ es finito y $x \in \{x\} \cup x \setminus F \subseteq U$.
	\end{proof} 
		
	\begin{corolario}\label{cor-NumAxMrowka}\index[sym]{$\mathcal{B}_\ms{A}$}\index[alph]{base!estándar de $\Psi_N(\ms{A})$}
		Si $N$ es numerable y $\ms{A} \subseteq [N]^\omega$, entonces:
		\begin{enumerate}[i)]
			\item $\Psi(\ms{A})$ es $1\AN$.
			\item $\mathcal{B}_{\ms{A}} := \midcup \{ \mathcal{B}_x \tq x \in \ms{A} \} \cup \big\{ \{n\} \tq n \in N \big\}$; denominado la \textbf{base estándar de} $\Psi_N(\ms{A})$, es una base de $\Psi_N(\ms{A})$ de tamaño $\aleph_0+|\ms{A}|$.
			\item $w(\Psi_N(\ms{A})) = \aleph_0+|\ms{A}|$. Por ello, $\Psi(\ms{A})$ es $2\AN$ si y sólo si $|\ms{A}|\leq \aleph_0$.
		\end{enumerate}
	\end{corolario}
	\begin{proof}
		(i), (ii) y $w(\Psi_N(\ms{A})) \leq \aleph_0+|\ms{A}|$ son claros.

		Para $\aleph_0+|\ms{A}| \leq w(\Psi_N(\ms{A}))$ basta observar que $\omega,\ms{A} \subseteq \Psi(\omega)$ son subespacios discretos de tamaño (y por tanto, peso) $\aleph_0$ y $|\ms{A}|$, respectivamente. Por consiguiente, el peso de $\Psi(\omega)$ debe ser mayor o igual que ambos.
	\end{proof}

	Si $\ms{A}\subseteq [\omega]^\omega$ y $X \subseteq \Psi(\ms{A})$, dado que cada punto de $\omega$
es aislado, se tiene que $\der(X) \subseteq \ms{A}$. Por otra parte, si $a \in \ms{A}$, la única forma de que cada $a \setminus F$ (con $F \in [a]^{<\omega}$) tenga intersección no vacía con $X$ es que $X \cap a$ sea infinito.

	Debido a \ref{prop-BaseLocMrowka}, la discusión recién dada prueba el primer inciso (y con ello todos los restantes) del siguiente útil Lema.

	\begin{lema}\label{lem-primerosSubs}
		Sea $\ms{A} \subseteq [\omega]^\omega$, entonces:
		\begin{enumerate}[i)]
			\item Si $X \subseteq \Psi(\ms{A})$ , entonces $ \der(X)=\{ y \in \ms{A} \tq X \cap y \neq^* \emptyset \} $.
			\item $\ms{A}=\der( \Psi(\ms{A}) )$ y $\omega$ es discreto, denso en $\Psi(\ms{A})$.
			\item Cada $B \subseteq \ms{A}$ es un subespacio cerrado y discreto de $\Psi(\ms{A})$.
			\item $B \subseteq \omega$ es cerrado en $\Psi(\ms{A})$ sólo si es casi ajeno con cada elemento de $\ms{A}$.
		\end{enumerate}
	\end{lema}
		
	\begin{proposicion}\label{prop-PsiSiempre}
		Todo $\Psi$-espacio es separable, primero numerable, $\T_1$, disperso y desarrollable.
	\end{proposicion}
	
	\begin{proof} 
		Sea $\ms{A} \subseteq [\omega]^\omega$ cualquiera. El $\Psi$-espacio generado por $\ms{A}$ es separable pues $\omega$ es denso en $\Psi(\ms{A})$ y numerable; además, éste espacio es primero numerable debido al Corolario \ref{cor-NumAxMrowka}.
	
		(Axioma $\T_1$) Si $x \in \ms{A}$, del \autoref{lem-primerosSubs} se desprende la igualdad $\der(\{x\})=\{y \in \ms{A} \tq \{x\} \cap y \neq^* \emptyset \}=\emptyset$, lo cual implica que $\{x\}$ es cerrado.
		
		(Dispersión) Supóngase que $X \subseteq \Psi(\ms{A})$ es cualquier subconjunto no vacío. Si $X \subseteq \ms{A}$, cualquier $x \in X$ es aislado en $X$, pues $X$ es discreto (véase \ref{lem-primerosSubs}). En caso contrario, existe un elemento $x \in X \cap \omega$ y $x$ es aislado en $X$, pues $\{x\}$ es abierto al ser $x$ elemento de $\omega$.
		
		(Desarrollabilidad) Defínase $\mathcal{U}_n:=\{ \{a\} \cup a \setminus n \tq a \in \ms{A} \} \cup \{ \{y\} \tq y \in \omega \}$ para cada $n \in \omega$. Resulta claro que cada colección $\mathcal{U}_n$ es cubierta abierta de $\Psi(\ms{A})$. Sean $x \in \Psi(\ms{A})$ y $U$ un abierto tal que $x \in U$.
		
		Si $x \in \omega$, entonces $\{x\} = \St(x, \mathcal{U}_{x+1})$; en efecto, sea $V \in \mathcal{U}_{x+1}$ con $x \in V$, entonces $V=\{x\}$; pues de lo contrario $V =\{a\} \cup a \setminus (x+1)$ para cierto $a \in \ms{A}$, implicando esto que $x \notin x+1$, lo cual es imposible ya que $x \in \omega$. Por tanto, $x \in \{x\} = \St(x,\mathcal{U}_{x+1}) \subseteq U$. 
		
		Si $x \in \ms{A}$, entonces $x \subseteq^* U$ y $x \setminus U \subseteq \omega$ es finito y por ello existe $n_0 \in \omega$ tal que $x \setminus U \subseteq n_0$. Como $\{x\} \cup x \setminus n_0 \in \mathcal{U}_{n_0}$ es el único abirto de $\mathcal{U}_{n_0}$ al cual $x$ pertenece, $x \in \{x\} \cup x \setminus n_0 = \St(x, \mathcal{U}_{n_0}) \subseteq U$.
		
		Así pues, para cada $n \in \omega$, la colección $\{\St(x, \mathcal{U}_n) \tq n \in \omega\}$ es base local de $x$. Así que $\{\mathcal{U}_n \tq n \in \omega \}$ es un desarrollo para $\Psi(\ms{A})$.
	\end{proof}
	
	Como se probó recién, todo $\Psi$-espacio es $\T_1$, sin embargo, cuando la familia $\ms{A} \subseteq [\omega]^\omega$ no es casi ajena, el espacio $\Psi(\ms{A})$ no satisface el axioma de separación $\T_2$. Por esta razón, en la literatura se suele dar la definición \ref{Def-Mrowka} partiendo directamente de una familia casi ajena (el lector podrá corroborar esto en textos como \cite{hruMrowka,hruAlmost,kannanHereditarily}).
	
	\begin{proposicion}\label{prop-tra-casiAjenidad}\index[trad]{cero-dimensionalidad de $\Psi(\ms{A})$}\index[trad]{propiedad de! Tychonoff en $\Psi(\ms{A})$}\index[trad]{propiedad de!Hausdorff en $\Psi(\ms{A})$}
		Para cualquier $\ms{A}\subseteq [\omega]^\omega$ son equivalentes las siguientes condiciones:
		\begin{enumerate}[i)]
			\item $\ms{A}$ es familia casi ajena.
			\item $\Psi(\ms{A})$ es cero-dimensional.
			\item $\Psi(\ms{A})$ es de Tychonoff.
			\item $\Psi(\ms{A})$ es de Hausdorff.
		\end{enumerate}
	\end{proposicion}
	
	\begin{proof} 
		(i) $\rightarrow$ (ii) Si $\ms{A}$ es familia casi ajena, como $\Psi(\ms{A})$ es $\T_1$, basta verificar que cada elemento de la base estándar $\ms{B}_\ms{A}$ (definida en el Corolario \ref{cor-NumAxMrowka}) es cerrado. En efecto, cada $\{n\}$ con $n \in \omega$ es cerrado pues $\Psi(\ms{A})$ es $\T_1$. Y dados $x \in \ms{A}$ y $F \subseteq x$ finto, haciendo uso de \ref{lem-primerosSubs} se tiene que por ser $\ms{A}$ familia casi ajena, $\der(\{x\} \cup x \setminus F)=\{x\} \subseteq \{x\} \cup x \setminus F$. Así que $\{x\} \cup x \setminus F$ es cerrado, mostrando que $\Psi(\ms{A})$ es cero-dimensional, pues es $\T_1$ y contiene una base de abiertos y cerrados (\textbf{resultado R)}.
	
		(ii) $\rightarrow$ (iii) $\rightarrow$ (iv) Si $\Psi(\ms{A})$ es cero-dimensional, al ser espacio $\T_1$, resulta que entonces es espacio de Tychonoff (\textbf{resultado R)}. Por su parte, si $\Psi(\ms{A})$ es de Tychonoff, entonces es de Hausdorff.
		
		(iv) $\rightarrow$ (i) Si $\Psi(\ms{A})$ es de Hausdorff y $x,y \in \ms{A}$ son distintos, existen abiertos ajenos $U,V \subseteq \Psi(\ms{A})$ abiertos tales que $x \in U$ y $y \in V$. De donde $x \subseteq^* U$, $y \subseteq^* V$ y por consiguiente $x \cap y \subseteq^* U \cap V = \emptyset$.
	\end{proof}
	
	La Proposición anterior es el motivo por el cual el presente trabajo se enfocará únicamente la siguiente clase de espacios:

	\begin{definicion}
		Un \textbf{espacio de Mrówka (o, de Isbell-Mrówka)}\index[alph]{espacio!de Mrówka}\index[alph]{Mrówka!espacio de}\index[alph]{espacio!de Isbell-Mrówka}\index[alph]{Isbell-Mrówka!espacio de} es un $\Psi$-espacio generado por una familia casi ajena.
	\end{definicion}

	\begin{corolario}\label{cor-MrwokaSiempre}
		Todo espacio de Mrókwa es separable, primero numerable, de Tychonoff, cero-dimensional, disperso y de Moore.
	\end{corolario}

	La siguiente es sólo una de las múltiples relaciones importantes que existen entre los espacios de Mrówka y el conjunto de Cantor. Su demostración se basa en un hecho conocido en topología general; todo espacio cero-dimensional de peso $\kappa$ se encaja en $2^\kappa$ (véase \cite[Teo.~8.5.11, p.~299]{fidelElementos}).

	\begin{corolario}\label{cor-EncajeMrowkaCantor}
		Todo espacio de Mrówka $\Psi(\ms{A})$ se encaja en $2^{\aleph_0+|\ms{A}|}$. Particularmente, si $|\ms{A}|\leq \aleph_0$, el espacio $\Psi(\ms{A})$ se encaja en $2^\omega$ y es metrizable.
	\end{corolario}

	\section{Compacidad y local compacidad}

	Como el lector puede advertir, cada vez surgen más traducciones con las cuales maniobrar al momento de estudar los $\Psi$-espacios. El ideal generado por cierta $\ms{A} \in \Ad(\omega)$ es clave para distinguir cuáles subespacios de $\Psi(\ms{A})$ son compactos, y cuales no. 
	
	\begin{proposicion}\label{prop-Kcaract}\index[trad]{compacidad de los subespacios de $\Psi(\ms{A})$}
		Sean $\ms{A}\in \Ad(\omega)$ y $K \subseteq \Psi(\ms{A})$. Entonces $K$ es compacto si y sólo si $K \cap \omega \subseteq^* \midcup (K \cap \ms{A})$ y $K \cap \ms{A}$ es finito.
	\end{proposicion}
	\begin{proof} 
		Supóngase que $K \subseteq \Psi(\ms{A})$ es subespacio compacto, como la colección $\mathcal{U}:=\{ \{n\} \tq n \in K \cap \omega \} \cup \{ \{x\} \cup x \tq x \in K \cap \ms{A} \}$ es cubierta abierta para $K$ en $\Psi(\ms{A})$, existen $F \subseteq K \cap \omega$ y $G \subseteq K \cap \ms{A}$ finitos tales que $\{ \{n\} \tq n \in F\} \cup \{ \{x\} \cup x \tq x \in G\}$ es subcubierta de $\mathcal{U}$. Luego, es necesario que $K \cap \ms{A} =G$, así que $K \cap \ms{A}$ es finito. Además $(K \cap \omega) \setminus \midcup G = K \setminus \midcup G \subseteq F$ es finito y con ello $K \cap \omega \subseteq^* \midcup (K \cap \ms{A})$.
		
		Conversamente, supóngase que $K \cap \omega \subseteq^* \midcup (K \cap \ms{A})$ y que $K \cap \ms{A}$ es finito. Resulta claro que; si $y \in \ms{A}$, entonces $\{y\} \cup y$ es un subespacio compacto de $\Psi(\ms{A})$; consecuentemente $L:=\midcup\{ \{y\} \cup y \tq y \in K \cap \ms{A} \}$ es un subespacio compacto de $\Psi(\ms{A})$.
		
		Nótese que $K \cap L$ es cerrado en $L$; pues $L \setminus K \subseteq \omega$; consecuentemente $K \setminus L$ es compacto. Como $K \setminus L = (K \cap \omega) \setminus \midcup (K \cap \ms{A})$ es finito por hipótesis, $K \setminus L$ es compacto. Así, $K=(K \setminus L) \cup (K \cap L)$ es unión de subsespacios compactos de $\Psi(\ms{A})$; por tanto, es compacto.
	\end{proof}

	Así, los los subespacios compactos de $\Psi(\ms{A})$ son únicamente aquellos de la forma $M \cup H$; donde $H \subseteq \ms{A}$ es finito y $M \subseteq^* \midcup H$. Esto es, si $\mathcal{K}$ el conjunto de los subespacios compactos de $\Psi(\ms{A})$:
	$$ \mathcal{K}=\bigcup_{H \in [\ms{A}]^{<\omega}} \{ F \cup M \cup H \tq (F,M) \in [\omega]^{<\omega} \times \ms{P}(H) \} $$

	Por ello $|\ms{A}| \cdot \aleph_0 \leq |\mathcal{K}| \leq \sum \{ (\aleph_0 \cdot \mathfrak{c} ) \tq H \in [\ms{A}]^{<\omega} \} \leq |\ms{A}| \cdot \mathfrak{c} \leq \mathfrak{c} $; asi que todo espacio de Mrówka tiene; a lo sumo, $\mathfrak{c}$ subespacios compactos.

	La discusión sobre cuántos subsespacios compactos \textit{importantes} (esto es, los que determinan el carácter topológico de su extensión unipuntual) tiene $\Psi(\ms{A})$ se retomará en la \autoref{Subsec-sucesiones-Franklin}.
	
	\begin{corolario}\label{cor-IdealCompactosCarac}
		Sean $\ms{A}$una familia casi ajena y $A \subseteq \omega$ cualquiera. Entonces son equivalentes las siguientes condiciones:
		\begin{enumerate}[i)]
			\item $A \in \ms{I}(\ms{A})$
			\item Existe $K \subseteq \Psi(\ms{A})$ compacto tal que $A \subseteq K$.
			\item Existe $K \subseteq \Psi(\ms{A})$ compacto tal que $A \subseteq^* K$.
		\end{enumerate}
	\end{corolario}
	
	\begin{proof} 
		(i) $\to$ (ii) Si $A \in \ms{I}(\ms{A})$, existe $H \subseteq \ms{A}$ finito tal que $A \subseteq^* \midcup H$. De la Proposición anterior se desprende que $K:=A \cup H$ es un subespacio compacto de $\Psi(\ms{A})$ tal que $A \subseteq K$.
		
		La implicación (ii) $\to$ (iii) es clara, procédase con la restante.
		
		(iii) $\to$ (i) Supóngase que $K$ es un subcespacio compacto de $\Psi(\ms{A})$ tal que $A \subseteq^* K$. Consecuentemente $A \setminus \midcup (K \cap \ms{A}) \subseteq^* A \setminus (K \cap \omega) = A \setminus K =^* \emptyset$, en virtud de la Proposición previa. Lo anterior; dado que $K \cap \ms{A}$ es finito, muestra que $A \in \ms{I}(\ms{A})$.
	\end{proof}
	
	\begin{proposicion}\label{prop-tra-compacidad}\index[trad]{compacidad de $\Psi(\ms{A})$}\index[trad]{compacidad numerable de $\Psi(\ms{A})$}
		Sea $\ms{A}\in \Ad(\omega)$, entonces son equivalentes:
		\begin{enumerate}[i)]
			\item $\Psi(\ms{A})$ es compacto.
			\item $\Psi(\ms{A})$ es numerablemente compacto.
			\item $\ms{A}$ es finita y maximal.
		\end{enumerate}
	\end{proposicion}
	
	\begin{proof} 
		La implicación (i) $\rightarrow$ (ii) es evidente.
		
		(ii) $\rightarrow$ (iii) Supóngase que $\Psi(\ms{A})$ es numerablemente compacto. Dado que $\ms{A} \subseteq \Psi(\ms{A})$ es subespacio cerrado y discreto de $\Psi(\ms{A})$ (véase \ref{lem-primerosSubs}), entonces $\ms{A}$ es numerablemente compacto y discreto; por ello, es finito. De esta forma, $\mathcal{U}:=\big\{ \{n\} \tq n\in \omega \big\} \cup \big\{ \{x\} \cup x \tq x \in \ms{A} \big\}$ es una cubierta numerable para $\Psi(\ms{A})$ y en consecuencia, existe $F \subseteq \omega$ finito de tal modo que la colección $\big \{ \{n\} \tq n \in F \big\} \cup \big\{ \{x\} \cup x \tq x \in \ms{A}\}$ es subcubierta de $\mathcal{U}$. Por ende $\omega \subseteq^* \midcup \ms{A}$, al ser $\ms{A}$ y $F$ finitos. Así, $\ms{A}$ es maximal en virtud del Corolario \ref{cor-MADnecesarioUnion}.
		
		(iii) $\rightarrow$ (i) Si $\ms{A}$ es finita y maximal, se desprende del Corolario \ref{cor-MADnecesarioUnion} que $\omega \subseteq^* \midcup \ms{A}$. Así, $\Psi(\ms{A}) \cap \omega \subseteq^* \midcup (\Psi(\ms{A}) \cap \ms{A} )$ y $\Psi(\ms{A}) \cap \ms{A} = \ms{A}$ es finito, siguiéndose dela \autoref{prop-Kcaract} la compacidad de $\Psi(\ms{A})$.
	\end{proof}
	
	La siguiente Proposición para nada carece de importancia, pues los espacios de Isbell-Mrówka son los únicos (dentro de cierta clase) con tal propiedad.
	
	\begin{proposicion}\label{prop-MrwokaHLC}
		Todo espacio de Mrówka es hereditariamente localmente compacto, y en consecuencia, es espacio de Baire.
	\end{proposicion}
	
	\begin{proof} 
		Supóngase que $\ms{A}\in \Ad(\omega)$ y sea $X \subseteq \Psi(\ms{A})$ cualquiera. Como $\Psi(\ms{A})$ es de Hausdorff (recuérdese \ref{cor-MrwokaSiempre}), $X$ es de Hausdorff y basta verificar que cada punto de $X$ tiene una vecindad en $X$ compacta.
	
		Sea $x \in X$ arbitrario. Si $x \in \omega$, entonces $\{x\}$ es vecindad compacta de $x$ en $X$. Ahora, si $x \in \ms{A}$, entonces $K:=X \cap (\{x\} \cup x) \subseteq \{x\} \cup x$ es vecindad de $x$ en $X$. Además $K$ es compacto, en virtud del \autoref{cor-IdealCompactosCarac}, pues $K \cap \ms{A} = \{x\}$ es finito y $K \cap \omega \subseteq x \subseteq^* \midcup \{x\} = \midcup (K \cap \ms{A})$. Así, $X$ es localmente compacto y $\Psi(\ms{A})$ hereditariamente localmente compacto. 
		
		Consecuentemente, $\Psi(\ms{A})$ es localmente compacto y de Hausdorff, siendo esto suficiente para ser de Baire \textbf{(Teorema de Categoría de Baire)}.
	\end{proof}
	
	\section{Metrizabilidad y Pseudocompacidad}

	El \autoref{cor-EncajeMrowkaCantor} evidencía que la numerabilidad de una familia casi ajena $\ms{A}$ es suficiente para concluir la metrizabilidad de su espacio de Mrówka asociado, no resulta dificil notar que el recíproco también ocurre (dados \ref{cor-NumAxMrowka} y que $\Psi(\ms{A})$ es separable); sin embargo, se tienen más equivalencias:
	
	\begin{proposicion}\label{prop-tra-numerable}\index[trad]{metrizabilidad de $\Psi(\ms{A})$}\index[trad]{segundo numerabilidad de $\Psi(\ms{A})$}\index[trad]{$\sigma$-compacidad de $\Psi(\ms{A})$}\index[trad]{propiedad de!Lindelöf en $\Psi(\ms{A})$}
		Sea $\ms{A}\in \Ad(\omega)$, entonces son equivalentes:
		\begin{enumerate}[i)]
			\item $\ms{A}$ es a lo más numerable
			\item $\Psi(\ms{A})$ es metrizable.
			\item $\Psi(\ms{A})$ es segundo numerable.
			\item $\Psi(\ms{A})$ es $\sigma$-compacto.
			\item $\Psi(\ms{A})$ es de Lindelöf.
		\end{enumerate}
	\end{proposicion}
	
	\begin{proof} 
		(i) $\rightarrow$ (ii) $\rightarrow$ (iii) Si $|\ms{A}| \leq \omega$, se obtiene de \ref{cor-EncajeMrowkaCantor} que $\Psi(\ms{A})$ es metrizable. Por otro lado, si $\Psi(\ms{A})$ es metrizable, al ser éste un espacio separable, se tiene garantizado que es $2\AN$ \textbf{(MtzEq)}.
			
		(iii) $\rightarrow$ (iv) $\to$ (v) Si $\Psi(\ms{A})$ es $2\AN$, entonces al localmente compacto, resulta que es $\sigma$-compacto \textbf{(Resultado R)}. Además; todo espacio $\sigma$-compacto, es también de Lindelöf. \textbf{(Resultado R)}
			
		(v) $\rightarrow$ (i) Por último, supóngase que $\Psi(\ms{A})$ es de Lindelöf y sea $\mathcal{B}_\ms{A}$ la base estándar de $\Psi(\ms{A})$ (definida en \ref{cor-NumAxMrowka}). Luego $\mathcal{B}_ \ms{A}$ es una cubierta abierta de $\Psi(\ms{A})$, y deben existir $\ms{A}' \subseteq \ms{A}$ y $N \subseteq \omega$ a lo más numerables tales que $\midcup \big\{ \big\{ \{x\} \cup x \setminus F \tq F \in [x]^{<\omega} \big\} \tq x \in \ms{A}' \big\} \cup \big\{ \{n\} \tq n \in N \big\}$ es subcubierta de $\mathcal{B}_ \ms{A}$. Resulta así que $\ms{A} \subseteq \ms{A}'$ y $|\ms{A}| \leq \aleph_0$.
	\end{proof}
	
	Como fue mostrado en \ref{prop-MADnoNum}, ninguna familia casi ajena numerable es maximal. Así que si $\ms{A}$ es una familia casi ajena numerable, por \ref{prop-tra-compacidad}, $\Psi(\ms{A})$ no es compacto. Consecuentemente (por \ref{prop-tra-numerable}), si $\ms{A}$ es numerable, $\Psi(\ms{A})$ es Lindelöf y $\sigma$-compacto, pero no compacto.
	
	\begin{observacion}
		Si $\Psi(\ms{A})$ es metrizable (o cualquiera de sus equivalentes planteados en \ref{prop-tra-numerable}) y no compacto, no necesariamente $\ms{A}$ es numerable. Esto responde al sencillo motivo de que $\ms{A}$ podría ser maximal o no; la \autoref{prop-tra-numerable} no toma en cuenta este aspecto.
	\end{observacion}

	La observación recíen hecha da constancia de que falta establecer una relación entre $\Psi(\ms{A})$ y la maximalidad de la familia $\ms{A}$. En la \autoref{Subsec-sucesiones-Franklin} se ahondará con mucha más profundidad en el estudio de las sucesiones convergentes; pero de momento, es necesario considerar el siguiente Lema, en orden de dar una caracterización completa para $\ms{A} \in \Mad(\omega)$.
	
	\begin{lema}\label{lem-convObvia}
		Sean $\ms{A} \in \Ad(\omega) $, $x \in \ms{A}$ y $B \subseteq [\omega]^\omega$ cualesquiera. Entonces $B \to x$ en $\Psi(\ms{A})$ si y sólo si $B \subseteq ^* x$.
	\end{lema}

	\begin{proof} 
		Supóngase que $B \to x$ en $\Psi(\ms{A})$. Entonces, como $x \cup \{x\}$ es un abierto de $\Psi(\ms{A})$ que contiene a $x$, se tiene que $B \subseteq^* x \cup \{x\}$, mostrando que $B \subseteq^* x$. Y recíprocamente, si $B \subseteq^* x$ y $U \subseteq \Psi(\ms{A})$ es cualquier abierto con $x \in U$, entonces $x \subseteq^* U$, y por tanto, $B \subseteq^* U$.
	\end{proof}
	
	\begin{proposicion}\label{prop-tra-pseudoCaract}\index[trad]{pseudocompacidad de $\Psi(\ms{A})$}
		Sea $\ms{A} \in \Ad(\omega)$, son equivalentes:
		\begin{enumerate}[i)]
			\item $\Psi(\ms{A})$ es pseudocompacto.
			\item $\ms{A}$ es maximal.
			\item Todo subespacio discreto, abierto y cerrado de $\Psi(\ms{A})$ es finito.
			\item Toda sucesión en $\omega$ tiene una subsucesión convergente.
		\end{enumerate}
	\end{proposicion}
	
	\begin{proof}
		(i) $\rightarrow$ (ii). Si $\ms{A}$ no es maximal, existe $B \subseteq \omega$ infinito y casi ajeno con cada elemento de $\ms{A}$. Por \ref{prop-BaseLocMrowka} y \ref{lem-primerosSubs}, $B$ es discreto, abierto y cerrado, y de \textbf{(Ree A)} se sigue que $\Psi(\ms{A})$ no es pseudocompacto.
		
		(ii) $\rightarrow$ (iii) Por contrapuesta, supóngase que $B \subseteq \Psi(\ms{A})$ es infinito, discreto, abierto y cerrado de $\Psi(\ms{A})$. Sin pérdida de generalidad $B \subseteq \omega$ (de lo contrario cada $a \in B \cap \ms{A}$ cumple que $a \cap B = B \cap (\{a\} \cup a) \subseteq \omega$ es infinito, cerrado, abierto y discreto). Luego, de \ref{lem-primerosSubs} se desprende que $B$ casi ajeno con cada elemento de $\ms{A}$.
	
		(iii) $\rightarrow$ (iv) Supóngase (iii) y sea $B \in [\omega]^\omega $. Así, $B$ es discreto, infinito y abierto. Por hipótesis, debe existir $x \in \der(B) \setminus B$ y por \ref{cor-MrwokaSiempre}, $x \in \ms{A}$ y $B \cap x$ es infinito. Siguiéndose del \autoref{lem-primerosSubs} que $B \cap x \to x$.
		
		(iv) $\rightarrow$ (i) Por contrapuesta, supóngase que $f:\Psi(\ms{A}) \to \mathbb{R}$ es continua y no acotada. Entonces; por densidad de $\omega$, para cada $n \in \omega$ se puede fijar $m_n \in \omega \cap f^{-1}[(n,\infty)]$. Así, $B=\{m_n \tq n \in \omega\}$ es infinito, y no admite subsucesiones convergentes en $\Psi(\ms{A})$, pues ningún $C \in [B]^\omega$ tiene imagen no acotada bajo $f$.
	\end{proof}
		
	Combinando \ref{prop-tra-compacidad}, \ref{prop-tra-numerable} y \ref{prop-tra-pseudoCaract} se obtienen ejemplos muy concretos. Por ejemplo, si un espacio de Mrówka $\Psi(\ms{A})$ no es pseudocompacto pero sí es metrizable, necesariamente $\ms{A}$ es numerable. Otro ejemplo responde con una negativa a lo que en su momento fue un problema popular: ¿la pseudocompacidad equivale a la compacidad numerable en espacios Tychonoff?, resultado se sabía cierto en la clase de espacios $\T_4$ (\textbf{Reee R}) y falso dentro de la clase de espacios que no son $\T_1$. En virtud de \ref{prop-tra-compacidad}, y considerando cualquier familia maximal infinita, se obtiene
	
	\begin{corolario}\label{cor-EjmPseudoNoNumC}
		Existe un espacio de Tychonoff, que es pseudocompacto pero no numerablemente compacto.
	\end{corolario}
	
	La siguiente es una caracterización conocida (véase \cite[p.~39,45]{GeorginaTesis}) y; entre tanto, desvela que el comportamiento súmamente organizado y \textit{amigable} de $\Psi(\ms{A})$ se rompe bruscamente cuando $\ms{A}$ deja de ser numerable. Por tal motivo, no suelen ser de tanto interés los $\Psi$-espacios generados por familias casi ajenas a lo más numerables.

	\begin{proposicion}\label{prop-alomasNumCaract}\index[trad]{ordenabilidad lineal de $\Psi(\ms{A})$}
		Sea $\ms{A}$ una familia casi ajena con cardinalidad $\kappa$, entonces\footnotemark se satisface:
		\begin{enumerate}[i)]
			\item Si $\kappa=0$, entonces $\Psi(\ms{A}) \cong \omega$.
			\item Si $\kappa \in \omega$ y $\ms{A}$ no es maximal, $\Psi(\ms{A}) \cong \omega \cdot (\kappa+1)$.
			\item Si $\kappa \in \omega$ y $\ms{A}$ es maximal, $\Psi(\ms{A}) \cong \omega \cdot (\kappa+1)+1$.
			\item Si $\kappa=\omega$, entonces $\Psi(\ms{A}) \cong \omega^2$.
			\item Si $\kappa>\omega$, entonces $\Psi(\ms{A})$ no homeomorfo a ningún espacio de ordinales; más aún, $\Psi(\ms{A})$ no es linealmente ordenable.
		\end{enumerate}
	\end{proposicion}

	\footnotetext{En los incisos (i)-(iv), los espacios homeomorfos a $\Psi(\ms{A})$ están escritos en aritmética ordinal y dotados de su topología de orden.}
	
	Se derivan conclusiones de interés moderado, como puede ser que $\omega^2$ (como producto ordinal) es el único espacio de Mrówka metrizable, no compacto. Una consecuencia \textit{curiosa} en relación a éste espacio; y que además, surge como fruto del Teorema principal de la \autoref{sec-KRTeo}, es el \autoref{cor-omegaCuadra}.

	La peculiaridad recién comentada, sugiere que todas las familias casi ajenas numerables son muy \textit{esencialmente iguales} (conviniendo que $\ms{A}$ y $\ms{B}$ son \textit{esencialmente iguales} cuando $\Psi(\ms{A})$ y $\Psi(\ms{B})$ son homeomorfos).

	\section{Teorema de Kannan y Rajagopalan}
	\label{sec-KRTeo}
	La meta primordial en lo que resta del capítulo será caracterizar aquellos espacios que son homeomorfos a algún espacio de Mrówka. Como fué mostrado en la Proposición \ref{prop-MrwokaHLC}, todos los espacios de Mrówka son hereditariamente localmente compactos, una propiedad cuanto menos peculiar. Tal propiedad será la que los caracterizará dentro de la clase de espacios infinitos, de Hausdorff y separables.	
	
	\begin{lema}\label{lem-TKR-Baire}
		Sea $X$ un espacio de Hausdorff y localmente compacto. Si $X$ contiene un denso $D$, abierto y a lo más numerable, entonces $N:=X \setminus \der(X) \subseteq X$ discreto y denso en $X$.
	\end{lema}
	
	\begin{proof}
		Claramente $N$ es discreto. Por el Teorema de Categoría De Baire (\textbf{TCB}), resulta que $X$ es un espacio de Baire.

		Ahora, si $x \in D$ es aislado en $D$, entonces $\{x\}=D \cap U$ para cierto abierto $U$ de $X$ y dado que $D$ es denso y $X$ es un espacio $\T_1$, es necesario que $U=\{x\}$. Lo anterior prueba que $X \setminus \der_D(D) \subseteq N$.

		Por otra parte, si $x \in \der_D(D) \subseteq \der(X)$, entonces $X \setminus \{x\}$ es abierto y denso en $X$. Luego $X \setminus \der_D(D)=\midcap\{ X \setminus \{x\} \tq x \in \der_D(D) \}$ es denso, debido a que $X$ es de Baire. Lo cual basta para mostrar que $N$ es denso.
	\end{proof}
	
	\begin{lema}\label{lem-TKR-DerX}
		Sean $X$ un espacio topológico y $N:=X \setminus \der(X)$. Las siguientes condiciones son equivalentes:
		\begin{enumerate}[i)]
			\item $N$ es denso y para cada $y \in \der(X)$, $N \cup \{y\}$ es abierto.
			\item $\der(X)$ es discreto.
		\end{enumerate}
	\end{lema}
	
	\begin{proof} 
		(i) $\rightarrow$ (ii) Supóngase (i) y sea $y \in \der(X)$ cualquier elemento. $N \cup \{y\}$ es abierto en $X$, en consecuencia $y \in U \subseteq N \cup \{y\}$, para cierto abierto $U$. Seguido de lo anterior, ${y} = U \setminus N = U \cap \der(X)$. Mostrando que $\der(X)$ es discreto.
		
		(ii) $\rightarrow$ (i) Supóngase que $\der(X)$ es discreto. Si $N$ no es denso, existen $x \in X$ y un abierto $U$ de modo tal que $x \in U \subseteq \der(X)$. Pero al ser $\der(X)$ discreto, $\{x\}=W \cap \der(X)$ para cierto abierto $W$, de donde $U \cap W = \{x\}$ y $x \in N$, esto es imposible. Así que $N$ es denso en $X$.
		
		Ahora, si $y \in \der(X)$ es arbitrario, existe un abierto $U$ de modo que se da $\{y\} = U \cap \der(X)$, pues $\der(X)$ es discreto. De lo anterior se obtiene que $N \cup \{y\} = (N \cup U) \cap (N \cup \der(X)) = N \cup U$ es abierto en $X$.
	\end{proof}
	
			
	La siguiente caracterización es debida a Varadachariar Kannan y a Minakshisundaram Rajagopalan, quienes en 1970 (consúltese \cite{kannanHereditarily}) dieron con el resultado.
	
	\begin{teorema}[Kannan, Rajagopalan]\label{teo-HLCCaract}\index[alph]{Kannan!Teorema de Rajagopalan y}\index[alph]{Rajagopalan!Teorema de Kannan y}\index[alph]{Teorema! de Kannan y Rajagopalan}\index[trad]{compacidad local hereditaria de cualquier espacio infinito, separable, de Hausdorff}
		Para cualquier espacio topológico $X$ infinito, de Hausdorff y separable son equivalentes:
		\begin{enumerate}[i)]
			\item $X$ es hereditariamente localmente compacto.
			\item $X$ es localmente compacto $\der(X)$ es discreto.
			\item $X$ es homeomorfo a un espacio de Mrówka.
		\end{enumerate}
	\end{teorema}
	
	\begin{proof} 
		Supóngase que $X$ es cualquier espacio infinito, de Hausdorff, separable y sea $N:=X \setminus \der(X)$.
	
	(i) $\rightarrow$ (ii) Supóngase que $X$ es hereditariamente localmente compacto. Por separabilidad de $X$, existe $D \subseteq X$ denso y a lo más numerable. Se sigue de la hipótesis que $D$ es localmente compacto y por ello, es abierto en su cerradura, $X$. Debido al \autoref{lem-TKR-Baire}, $N$ es denso en $X$.
	
	Por otro lado, si $y \in \der(X)$ es cualquiera, $N \cup \{y\} \subseteq X$ es localmente compacto, y por ende, es abierto en su cerradura. Pero $N$ es denso, así que $N \cup \{y\}$ es abierto en $X=\cla(N \cup \{y\})$. Por lo tanto, de \ref{lem-TKR-DerX} se obtiene que $\der(X)$ es discreto.
	
	(ii) $\rightarrow$ (iii) Supóngase que $X$ es localmente compacto y que $\der(X)$ es discreto. Por el Lema \ref{lem-TKR-DerX} resulta que $N$ es denso en $X$ y que $N \cup \{y\}$ es abierto siempre que $y \in \der(X)$. Por ser $X$ infinito y separable, se tiene que $N$ es numerable. Utilizando la compacidad local de $X$, para cada $x \in \der(X)$ fíjese (utilizando $\Ac$) una vecindad compacta $V_x$ de $x$ en $X$ contenida en $N \cup \{x\}$. Se afirma que $ \ms{A} = \{ V_x \setminus \{x\} \subseteq N \tq x \in \der(X) \} \in \Ad(N) $.
	
	En efecto, si $x \in \der(X)$ es cualquiera, entonces $V_x \setminus \{x\}$ no es finito. De lo contrario, $\{x\}= (N \cup \{x\}) \setminus (V_x \setminus \{x\})$ sería abierto en $X$ (que es espacio $\T_1$) y se contradiría que $x \in \der(X)$. Por tanto, $\ms{A} \subseteq [N]^\omega$. Además, si $x,y \in \der(X)$ son distintos, se tiene que $V_x \cap V_y \subseteq N$. Así $V_x \cap V_y$ es subespacio compacto del discreto $N$, lo cual obliga a que sea finito. Como consecuencia, $\ms{A}$ es familia casi ajena en $N$.
	
	Defínase $f:X \to \Psi_N(\ms{A})$ por medio de $f(n)=n$ si $n \in N$ y $f(x)=V_x$ si $x \in \der(X)$. Claramente $f$ es función biyectiva; además, como $N$ es el conjunto de puntos aislados de $X$, para verificar que $f$ es homeomorfismo basta verificar lo siguiente.	
	\begin{enumerate}[\hspace{1.5 cm}, listparindent=1.5em]
		\item \textit{Afirmación.} Un subconjunto $U \subseteq X$ es abierto si y sólo si para cada $x \in U \cap \der(X)$ se tiene $V_x \setminus \{x\} \subseteq^* U$.
		
		\item \textit{Demostración.} Sea $U \subseteq X$. Si $U$ es abierto y $x \in U \cap \der(X)$ es cualquiera, entonces $V_x \setminus U \subseteq N$ es cerrado en $X$, así en $V_x$ y como $V_x$ es compacto; $V_x \setminus U$ es subespacio compacto del discreto $N$, por tanto finito. Así que $V_x \setminus \{x\} \subseteq^* U$.
		
		Recíprocamente, supóngase que para cada $x \in U \cap \der(X)$ se tiene que $V_x \setminus \{x\} \subseteq^* U$, es decir, que $V_x \setminus U$ es finito. Sea $y \in U$ cualquiera, si $y \in N$ entonces $\{y\}$ es abierto en $X$ y $U$ es vecindad de $y$. Ahora, si $y \in \der(X)$ entonces $V_y \setminus U$ es finito y con ello $V_y \setminus (V_y \setminus U) \subseteq U$, de donde $U$ es vecindad de $y$ (usando que $X$ es espacio $\T_1$). Luego, $U$ es vecindad de todos sus puntos, y por tanto, es abierto. \hfill$\boxtimes$
	\end{enumerate}
	
	(iii) $\rightarrow$ (i) Si $X$ es homeomorfo a un espacio de Mrówka, las propiedades topológicas del último se satisfacen en $X$, siguéndose de \ref{prop-MrwokaHLC} que $X$ es hereditariamente localmente compacto.
	\end{proof}
	
	Del resultado anterior es casi inmediata la obtención de las siguientes condiciones equivalentes.
	
	\begin{corolario}\label{cor-HLCPseudoCaract}\index[trad]{compacidad local hereditaria y pseudocompacidad de cualquier espacio infinito, separable, de Hausdorff}
		Sea $X$ cualquier espacio infinito, de Hausdorff y separable. Entonces las siguientes condiciones son equivalentes:
		\begin{enumerate}[i)]
			\item $X$ es pseudocompacto y hereditariamente localmente compacto.
			\item $X$ es regular, $\der(X)$ es subespacio discreto de $X$ y cualquier subespacio discreto, abierto y cerrado a la vez en $X$ es finito.
			\item $X$ es homeomorfo a un espacio de Mrówka generado por una familia casi ajena maximal.
		\end{enumerate}
	\end{corolario}
	\begin{proof} 
		Por el Teorema de Kannan y Rajagopalan, lo demostrado en \ref{prop-tra-pseudoCaract} y como todo espacio de Mrówka es de Tychonoff (véase \ref{cor-MrwokaSiempre}); particularmente regular, bastará demostrar que si $X$ satisface (ii) entonces $X$ es localmente compacto. Supóngase (ii), claramente cada punto aislado de $X$ tiene una vecindad compacta en $X$.
	
		Sea $x \in \der(X)$ arbitrario, como $\der(X)$ es discreto, existe $U \subseteq X$ abierto con $\{x\} = U \cap \der(X)$. Por regularidad de $X$, fíjese un abierto $V$ tal que $x \in V \subseteq \cla(V) \subseteq U$ y nótese que entonces $\{x\}=\cla(V) \cap \der(X)$.
		
		Si $W$ es una vecindad abierta de $x$, entonces $\cla(V) \setminus W$ es discreto y abierto (por ser subespacio de $X \setminus \der(X)$) y cerrado (por ser intersección de cerrados). De (ii) se sigue la finitud de $\cla(V) \setminus W$, y de esto, la compacidad de $\cla(V)$, siendo tal subespacio, una vecindad compacta de $x$ en $X$.
	\end{proof}
	
	\begin{corolario}
		Sea $X$ un espacio topológico infinito, entonces $X$ es homeomorfo a un espacio de Mrówka si y sólo si es homeomorfo a un subespacio abierto de un espacio de Mrówka.
	\end{corolario}
	\begin{proof} 
		Basta probar la necesidad. Supóngase que $\ms{A}$ es una familia casi ajena y que $U \subseteq \Psi(\ms{A})$ es un abierto tal que $X \cong U$. Como $X$ es infinito, $U$ es infinito, además por ser $\Psi(\ms{A})$ de Hausdorff y hereditariamente localmente compacto, se tiene que $U$ es de Hausdorff y hereditariamente localmente compacto. Por último, como $\omega$ es denso en $\Psi(\ms{A})$ y $U$ es abierto en $\Psi(\ms{A})$, se tiene que $U \cap \omega$ es denso en $U$; así que $U$ es separable. De lo anterior $U$, y por tanto $X$, es homeomorfo a un espacio de Mrówka; a saber $\Psi_{U \cap \omega} (U \cap \ms{A})$.
	\end{proof}
	
	\begin{corolario}
		Sea $\{X_\alpha \tq \alpha \in \kappa \}$ una familia no vacía de espacios topológicos infinitos; sin pérdida de generalidad ajenos dos a dos, entonces son equivalentes:
		\begin{enumerate}[i)]
			\item $\displaystyle Y:=\coprod_{\alpha \in \kappa} X_\alpha$ es homeomorfo a un espacio de Mrówka.
			\item $\kappa$ es contable y cada $X_\alpha$ es homeomorfo a un espacio de Mrówka.
		\end{enumerate}
	\end{corolario}
	\begin{proof} 
		(i) $\to$ (ii) Supóngase que $Y$ es espacio de Mrówka. Como cada $X_\alpha \subseteq Y$ es infinito y abierto en $Y$, se sigue del Corolario anterior que $X_\alpha$ es de Mrówka. Por otro lado, si $\kappa$ fuese más que numerable, $Y$ no podía ser separable, pues es la suma de $\kappa$ espacios no vacíos; así que $\kappa$ es a lo más numerable.
	
		(ii) $\to$ (i) Supóngase que $\kappa$ es a lo más numerable y para cada $\alpha \in \kappa$, el espacio $X_\alpha$ es homeomorfo a un espacio de Mrówka. Entonces, del \autoref{sec-KRTeo}, cada $X_\alpha$ es (infinito) de Hausdorff, separable, localmente compacto y además el subespacio $\der_{X_\alpha}(X_\alpha)\subseteq X_\alpha$ es discreto.
		
		La suma de espacios de Hausdorff (localmente compactos, respectivamente) es de Hausdorff (localmente compacta, respectivamente); además, por ser cada $X_\alpha$ separable y $\kappa$ a lo más numerable, resulta que $Y$ es infinito, de Hausdorff, localmente compacto y separable.
		
		Sea $y \in \der_Y(Y)$ cualquiera, por definición de $Y$, para el único elemento $\alpha \in \kappa$ tal que $y \in X_\alpha$, se tiene $y \in \der_{X_\alpha}(X_\alpha)$. Y como tal subespacio de $X_\alpha$ es discreto, existe $V \subseteq X_\alpha$ abierto tal que $\{y\}=U \cap \der_{X_\alpha}(X_\alpha)$, pero $U$ es abierto también en $Y$ y además $\{y\}=U \cap \der_Y(Y)$. De lo contrario, existe $x \in V \cap \der_Y(Y) \setminus \{y\}$ y consecuentemente $x \notin \der_{X_\alpha}(X_\alpha)$, mostrando que $\{x\}$ es abierto en $X_\alpha$ y por tanto en $Y$, lo cual es absurdo dada la elección de $X$. Lo anterior prueba que $\der_Y(Y)$ es discreto, finalizando la prueba en virtud del \autoref{teo-HLCCaract}.
	\end{proof}

	Se explotará mucho la siguiente observación durante el subsecuente Corolario, pues nuevamente, se hará uso del inciso (ii) del \autoref{teo-HLCCaract}.
	\begin{observacion}
		Sea $X$ un espacio topológico, $\der(X)$ es discreto si y sólo si $\der^2(X):=\der(\der(X)) = \emptyset$.

		Efectivamente; como $X\setminus \der(X)$ es abierto, $\der(X)$ es discreto si y sólo si es discreto y cerrado. Esto último sucede únicamente cuando $\der_{\der(X)}(\der(X))=\der(X) \cap \der^2(X) =\emptyset$. Sin embargo, cualquier punto aislado en $X$, es aislado en $\der(X)$, así que $\der^2(X) \subseteq \der(X)$; por lo tanto, $\der(X)$ es discreto si y sólo si $\der^2(X)=\emptyset$.
	\end{observacion}
	
	\begin{lema}
		Sean $X$ y $Y$ espacios topológicos infinitos, entonces $X \times Y$ es homeomorfo a un espacio de Mrówka si y sólo si $X$ y $Y$ son de Mrówka y además $X \cong \omega$ o $Y \cong \omega$
	\end{lema}
	
	\begin{proof} 
		Obsérvese la igualdad:
	\begin{align*}
		\der^2_{X \times Y} (X \times Y) & = \der_{X \times Y} \Big( \der_X(X) \times \cla_Y(Y) \cup \cla_X(X) \times \der_Y(Y) \Big) \\
		& = \der_{X \times Y} \Big( \der_X(X) \times Y \cup X \times \der_Y(Y) \Big) \\
		& = \der_{X \times Y} \Big( \der_X(X) \times Y \Big) \cup \der_{X \times Y} \Big( X \times \der_Y(Y) \Big) \\
		& = \der_X(\der_X(X)) \times \cla_Y(Y) \cup \cla_X(\der_X(X)) \times \der_Y(Y) \: \cup \\
		& \cup \der_X(X) \times \cla_Y(\der_Y(Y)) \cup \cla_X(X) \times \der_Y(\der_Y(Y)) \\
		& = \der^2_X(X) \times Y \cup \der_X(X) \times \der_Y(Y) \cup X \times \der^2_Y(Y)
	\end{align*}
	
		Puesto que $X,Y \neq \emptyset$, resulta que $\der^2_{X \times Y} (X \times Y)$ es vacío si y sólo si $\der^2_X(X) = \der^2_Y(Y) = \der_X(X) \times \der_Y(Y) = \emptyset$. Esto es, el subespacio $\der_{X \times Y}(X \times Y) \subseteq X \times Y$ es discreto si y sólo si los subespacios $\der_X(X)$ de $X$ y $\der_Y(Y)$ de $Y$ son discretos y además $X$ es discreto o $Y$ es discreto.
		
		Como $X,Y$ son infinitos, $X \times Y$ es infinito, además las propiedades de separabilidad, axioma de separación de Hausdorff y local compacidad son propiedades finitamente productivas y finitamente factorizables. De esto último, lo comentado en el párrafo anterior, el hecho de que el único espacio de Mrówka discreto es $\omega$ y el inciso (ii) del \autoref{teo-HLCCaract}, se obtiene el resultado.
	\end{proof}
	
	\begin{corolario}
		Sea $\{X_\alpha \tq \alpha \in \kappa \}$ una familia no vacía de espacios topológicos infinitos; sin pérdida de generalidad ajenos dos a dos, entonces son equivalentes:
		\begin{enumerate}[i)]
			\item $\displaystyle Y:=\prod_{\alpha \in \kappa} X_\alpha$ es homeomorfo a un espacio de Mrówka.
			\item $\kappa$ es finito, cada $X_\alpha$ es homeomorfo a un espacio de Mrówka y existe $\beta_0 \in \kappa$ tal que si $\alpha \in \kappa \setminus \{\beta_0\}$, se tiene $X_\alpha \cong \omega$.
		\end{enumerate}
	\end{corolario}
	
	\begin{proof} 
		Sin perder generalidad, tómese $\kappa$ como un cardinal.
	
		(i) $\to$ (ii) Supóngase que $Y$ es homeomorfo a un espacio de Mrówka, entonces $Y$ es de Hausdorff, Separable y hereditariamente localmente compacto. Todas las propiedades anteriores son factorizables, así que por el por el Teorema de Kannan y Rajagopalan (\ref{teo-HLCCaract}), cada $X_\alpha$ es homeomorfo a un espacio de Mrówka.
		
		Ahora, por contradicción, supóngase $\kappa \geq \omega$. Entonces, existen $P,Q \subseteq \kappa$ ajenos e infinitos, de donde:
		$$ Y = \prod_{\alpha \in \kappa} X_\alpha \cong \prod_{\alpha \in P} X_\alpha \times \prod_{\alpha \in Q} X_\alpha $$
		siguiéndose del Lema previo que; sin pérdida de generalidad, $\prod_{\alpha \in P} X_\alpha \cong \omega$. Lo anterior conduce a un absurdo, pues como $P$ es infinito y cada $X_\alpha$ también, resulta que:
		$$ \Bigg| \prod_{\alpha \in P} X_\alpha \Bigg| = \prod_{\alpha \in P} |X_\alpha| \geq \prod_{\alpha \in P} \aleph_0 = \aleph_0^{|P|} \geq \aleph_0^{\aleph_0} > \aleph_0 $$
		imposibilitando que $\prod_{\alpha \in P} X_\alpha \cong \omega$ sea biyectable con $\omega$. Así, $\kappa < \omega$.

		Finalmente, si cada $X_\alpha$ es homeomorfo a $\omega$, o $\kappa=1$, (ii) se satisface. Supóngase pues que $\kappa \geq 2$ y que existe $\beta_0 \in \kappa$ con $X_{\beta_0} \not\cong \omega$. Dado que:
		$$ Y = \prod_{\alpha \in \kappa} X_\alpha \cong X_{\beta_0} \times \prod_{\alpha \in \kappa \setminus \{\beta_0\}} X_\alpha $$
		se sigue del Lema Previo que $\prod_{\alpha \in \kappa \setminus \{\beta_0\}} X_\alpha \cong \omega$. Siendo así, cada $X_\alpha$ (con $\alpha \in \kappa \setminus \{\beta_0\}$) infinito, numerable y discreto; esto es, homeomorfo a $\omega$.
		
		(ii) $\to$ (i) Supóngase que $\kappa$ es finito, que cada $X_\alpha$ es homeomorfo a un espacio de Mrówka y que $\beta_0 \in \kappa$ es un elemento tal que si $\alpha \in \kappa \setminus \{\beta_0\}$, entonces $X_\alpha \cong \omega$. Como $\kappa \setminus \{\beta\}$ es finito, entonces:
		$$ Y = \prod_{\alpha \in \kappa} X_\alpha \cong X_\beta \times \prod_{\alpha \in \kappa \setminus \{\beta\}} X_\alpha \cong X_\beta \times \prod_{\alpha \in \kappa \setminus \{\beta\}} \omega = X_\beta \times \omega $$
		y a consecuencia del Lema previo, $Y$ es de Mrówka.
	\end{proof}
	
	El siguiente Corolario del Teorema de Kannan y Rajagopalan (\ref{teo-HLCCaract}), es un resultado sencillo (y sumamente particular) de metrización.
	
	\begin{corolario}\label{cor-omegaCuadra}\index[trad]{separabilidad hereditaria de cualquier espacio infinito, separable, de Hausdorff, hereditariamente localmente compacto}
		Si $X$ es infinito, separable, de Hausdorff y hereditariamente localmente compacto. Entonces son equivalentes:
		\begin{enumerate}[i)]
			\item $X$ es hereditariamente separable.
			\item $X$ es metrizable.
		\end{enumerate}
	\end{corolario}
	
	\begin{proof} 
		Dado el \autoref{teo-HLCCaract} y la caracterización \ref{prop-tra-numerable}, basta ver que si $\ms{A}\in \Ad(\omega)$, entonces $\Psi(\ms{A})$ es hereditariamente separable si y sólo si $\ms{A}$ es a lo más numerable.
	
		Para la suficiencia procédase por contrapuesta suponiendo que $\ms{A}$ es más que numerable, entonces $\ms{A}$ es un subespacio de $\Psi(\ms{A})$ discreto y más que numerable, con lo que, no puede ser serparable. Para la necesidad, si $\ms{A}$ es a lo más numerable, cada subespacio de $\Psi(\ms{A})$ es a lo más numerable, y con ello, separable.
	\end{proof}
	
	La \textit{curiosidad} (comentada posteriormente a \ref{prop-alomasNumCaract}) en relación al espacio de ordinales $\omega^2$ tiene su justificación en el anterior Corolario.

	Se finalizará la sección; y con ello el actual capítulo, dando un Corolario importante en relación a las imagenes continuas de los espacios de Mrówka pseudocompactos.
	
	\begin{corolario}
		Sea $X$ infinito y de Hausdorff. Son equivalentes:
		\begin{enumerate}[i)]
			\item Existe un denso $D \subseteq X$ de $X$ numerable tal que cada sucesión en $D$ tiene una subsucesión convergente en $X$.
			\item $X$ es imagen continua de un espacio de Mrówka generado por una familia maximal.
		\end{enumerate}	
	\end{corolario}
	
	\begin{proof} 
		(i) $\rightarrow$ (ii) Supóngase (ii) y sea $S \subseteq \Ad(D)$ el conjunto de familias casi ajenas en $D$ tales que para cada $\ms{B} \in S$, cada elemento de $\ms{B}$ es imagen de una sucesión en $D$ convergente en $X$. Como $D$ es numerable, existe una biyección $f_0:\omega \to D$ biyectiva, misma que admite una subsucesión convergente, a saber $g_0:\omega \to D$ convergente en $X$. Se desprende que $\{\ima(g_0)\} \in S$ y por tanto $S$ es no vacío, siguiéndose de una aplicación del Principio de Maximalidad de Hausdorff (similar al utilizado en \ref{lem-MADs}) la existencia de una familia casi ajena en $D$, $\ms{A} \subseteq \midcup S$ tal que si $\ms{B} \in S$ y $\ms{A} \subseteq \ms{B}$, entonces $\ms{A} = \ms{B}$.
	\begin{enumerate}[\hspace{1.5 cm}, listparindent=1.5em]
		\item \textit{Afirmación.} $\ms{A}$ es familia casi ajena maximal sobre $D$.
		
		\item \textit{Demostración.} Obsérvese primero que si $A\in \ms{A}$, existe $\ms{C} \in S$ tal que $A \in \ms{C}$ tal que $A \in \ms{C}$; consecuentemente $A$ es imagen de una sucesión en $D$ convergente en $X$; es decir $A \in \ms{A}$. Ahora, si $B \subseteq D$ infinito, entonces existe una biyección $f:\omega \to B$ y, por hipótesis, existe $g:\omega \to B \subseteq D$ subsucesión de $f$, convergente en $X$ y con ello $\{\ima(g)\} \in S$.
		
		Por un lado, si $\ms{A} \cup \{\ima(g)\}$ no es casi ajena, existe $A \in \ms{A}$ de modo que $A \cap \ima(g)$ es infinito, y con ello $A \cap B$ es infinito. De otro modo, $\ms{A} \cup \{\ima(g)\} \in S$ y por la construcción de $A$ se tiene $\ms{A} \cup \{\ima(g)\} = \ms{A}$, siendo $A:=\ima(g) \in \ms{A}$ tal que $A \cap B$ es infinito (pues $g$ es subsucesión de $f$). Lo anterior prueba que $\ms{A}$ es maximal sobre $D$. \hfill $\boxtimes$
	\end{enumerate}
	
	Para cada $A \in \ms{A}$ fíjese (\Ac) una sucesión $f_A:\omega \to D$ convergente a $x_A$ en $X$ tal que $A=\ima(f_A)$. Nótese que, como $X$ es de Hausdorff tal elemento $x_A$ es el único al cual $f_A$ converge. Además, dado que los elementos de $\ms{A}$ son casi ajenos dos a dos, y de nuevo por ser $X$ de Hausdorff, cada vez que $A,B \in \ms{A}$ sean distintos, se tendrá que $f_A \neq f_B$ y $x_A \neq x_B$. Defínase la función $p:\Psi_D(\ms{A}) \to X$ como $p(d)=d$ si $d \in D$ y $p(A)=x_A$ si $A \in \ms{A}$, veamos que $p$ es continua y sobreyectiva.
	
	Sea $U \subseteq X$ abierto en $X$ y supóngase que $A \in p^{-1}[U] \cap \ms{A}$ es cualquiera, entonces $p(A)=x_A \in U$ y $A=\ima(f_A)$. Como $U$ es un abierto de $X$ y $f_A$ converge a $x_A$ en $X$, resulta que $\ima(f_A) \subseteq^* U$ y con ello $A \subseteq^* p^{-1}[U]$; así que $p^{-1}[U]$ es abierto en $\Psi_D(\ms{A})$, y por tanto $p$ es continua.
	
	Ahora, sea $x \in X \setminus D$ cualquier elemento. Por contradicción, supóngase que $x \notin \ima(p)$, entonces si $s:\omega \to D$ es cualquiera, $s$ no puede converger a $a$ en $X$; de lo contrario, existe $A \in \ms{A}$ tal que $A \cap \ima(s)$ es infinito y con ello $f_A$ converge a $x$ en $X$, con lo que $x=p(A)$. Sin embargo 
	$$ \text{AQUÍ ESTO YA NO SALE} $$

	(ii) $\to$ (i) SALE FÁCIL
	\end{proof}
	% Supóngase que $\ms{A}$ es una familia maximal tal que $X$ es imagen continua de $\Psi(\ms{A})$ por medio de $f:\Psi(\ms{A}) \to X$. Como $f$ es continua y sobreyectiva, $D:=f[\omega]$ es denso en $X$. Supóngase que $(a_n)_{n \in \omega} \subseteq D$ es cualquier sucesión. Para cada $n \in \omega$ fíjese $b_n \in \omega$ de modo tal que $f(b_n)=a_n$ (esto no requiere de elección debido al buen orden de $\omega$). Así, $(b_n)_{n \in \omega}$ es una sucesión en $\omega$ y como $\ms{A}$ es maximal, se sigue de la Proposición \ref{prop-tra-pseudoCaract}, que $(b_n)_{n \in \omega}$ contiene una subsucesión convergente en $\Psi(\ms{A})$. Así que por continuidad de $f$, la sucesión $(a_n)_{n \in \omega}$ admite también una subsucesión convergente en $X$.
	
	\begin{corolario}
		Todo espacio metrizable, separable y compacto es imagen continua de un espacio de Mrówka; en particular, el cubo de Hilbert $[0,1]^\omega$ y el conjunto de Cantor $2^\omega$.
	\end{corolario}
        \chapter{El compacto de Franklin}
	\emph{\small Se comenzará introduciendo los espacios conocidos como \textit{compactos de Franklin}, que no son más que la extensión unipuntual (de Alexxandroff) de los espacios de Mrówka, si estos son no compactos.}

	\emph{\small Se logrará caracterizar cuándo estos espacios satisfacen con la propiedad de Fréchet; objetivo que requerirá los conocimientos obtenidos en el primer capítulo de este trabajo de tesis y nociones básicas sobre espacios secuenciales y de Fréchet. Durante el proceso de tal caracterización, se resolverá de paso un problema que estuvo sin solución en $\zfc$ durante cierta parte del siglo pasado; la productividad finita de la propiedad de Fréchet.}

	\section{\texorpdfstring{Sucesiones en $\ms{F}(\ms{A})$}{Sucesiones en F(A)}}
	\label{Subsec-sucesiones-Franklin}
	
	\begin{definicion}\index[alph]{compacto!de Franklin}\index[alph]{Franklin! compacto de}\index[sym]{$\ms{F}(\ms{A})$}
		Sea $\ms{A}\subseteq [\omega]^\omega$ cualquiera. El \textbf{compacto de Franklin generado por $\ms{A}$} es la extensión unipuntual del $\Psi$-espacio generado por $\ms{A}$, se denota por $ \ms{F}(\ms{A}):=\Psi(\ms{A}) \cup \{\infty_\ms{A} \} $.
		
		Cuando el contexto así lo permita, se omitirá el subíndice ``$_\ms{A}$'' y se denotará el punto al infinito simplemente por $\infty$.
	\end{definicion}
	
	\index[alph]{familia!no compacta}
	Dado que un espacio topológico no compacto admite compactaciones Hausdorff únicamente cuando es Tychonoff y localmente compacto (véase \cite[p.~ 221]{fidelElementos}), el compacto de Franklin resulta ser la compactación de Alexandroff de $\Psi(\ms{A})$ únicamente cuando $\ms{A}$ sea una familia casi ajena, lo que garantiza que $\Psi(\ms{A})$ sea de Tychonoff; y sea \textit{no compacta} (es decir, que no sea simultanemente finita y maximal), lo cual obliga a que $\Psi(\ms{A})$ sea no compacto; a razón de la \autoref{prop-tra-compacidad}.

	Por ello; y salvo se diga lo contrario, se convendrá que $\ms{A}$ es una familia (casi ajena) no compacta. Y más allá de esto, en pos de aligerar la notación de las futuras pruebas del capítulo, es menester convenir: 

	\begin{consideracion}
		Durante esta sección:
		\begin{enumerate}
			\item Para cada subespacio compacto $K \subseteq \Psi(\ms{A})$, se denotará por $V(K)$ a la vecindad abierta de $\infty$: $\{\infty\} \cup \Psi(\ms{A}) \setminus K$ (como $\Psi(\ms{A})$ es Hausdorff, todos los abiertos al rededor de $\infty$ son de esta forma).
			\item Se utilizarán casi en exceso los resultados obtenidos en \ref{prop-Kcaract} y \ref{cor-IdealCompactosCarac}, así que no se referenciarán de ahora en más.
			\item Todas las convergencias y operadores que aparezcan sin subíndices se asumirán en $\ms{F}(\ms{A})$. En caso de aparecer éstos en otros espacios, esto se indicará en sus subíndices.
		\end{enumerate}
	\end{consideracion}

	Lo primero a observar es lo siguiente: dado que $\Psi(\ms{A})$ no es compacto, al ser un espacio de Tychonoff y localmente compacto, se tiene efectivamente que $\ms{F}(\ms{A})$ es de Hausdorff, compacto (en consecuencia, normal). Como es previsible, ciertas propiedades de $\Psi(\ms{A})$ influyen en la topología de $\ms{F}(\ms{A})$; como ejemplo inmediato, la separabilidad se preserva.

	\begin{observacion}
		Sea $\ms{A}$ una familia casi ajena. Entonces el espacio $\ms{F}(\ms{A})$ es de Hausdorff, compacto, normal, localmente compacto, separable y disperso.
	\end{observacion}

	Comparando con el Corolario \ref{cor-MrwokaSiempre} con las observaciones recién hechas, vale mencionar que existen propiedades $\ms{F}(\ms{A})$ que tienen una dependencia más compleja con $\Psi(\ms{A})$. Entre ellas, todas las que tengan relación a las sucesiones.

	\begin{proposicion}\label{prop-caracterFrechet}
		Sea $\ms{A}$ una familia no compacta. Entonces el carácter de $\infty$ en $\ms{F}(\ms{A})$ es exactamente $\aleph_0 + |\ms{A}|$.
	\end{proposicion}
	
	\begin{proof} Es evidente que $\aleph_0 \leq \chi(\infty,\ms{F}(\ms{A}))$. Ahora, sea $\mathcal{B}$ una base local de $\infty$ en $\ms{F}(\ms{A})$. Para cada $y \in \ms{A}$ fíjese un elemento $B_y \in \mathcal{B}$ de modo que $B_y \subseteq V(y \cup \{y\})$ (recuérdese que $y \cup \{y\}$ es compacto). Obsérvese que la asignación $y \mapsto B_y$ es inyectiva; pues, si $x,y \in \ms{A}$ son distintos, entonces $B_x \subseteq U(x \cup \{x\})$ y $B_y \subseteq U(y \cup \{y\})$, de donde $x \in B_y \setminus B_x$. Por lo tanto $ |\ms{A}| \leq |\mathcal{B}|$, y en consecuencia $\aleph_0 + |\ms{A}|\leq \chi(\infty,\ms{F}(\ms{A}))$.
	
	Para la desigualdad recíproca defínase:
	$$ \mathcal{B} = \big\{ V\big( h \cup (B \cup \midcup h) \big) \tq (h,B) \in [\ms{A}]^{<\omega} \times [\omega]^{<\omega} \big\} $$
	y nótese que $\mathcal{B}$ es un conjunto de vecindades de $\infty$ en $\ms{F}(\ms{A})$.
	
	Si $K$ es cualquier subespacio compacto de $\Psi(\ms{A})$, entonces los conjuntos $K \cap \ms{A} \subseteq \ms{A}$ y $G:=(K \cap \omega) \setminus \midcup (K \cap \ms{A}) \subseteq \omega$ son finitos; consecuentemente $V\big( (K \cap \ms{A}) \cup \big( G \cup \midcup (K \cap \ms{A}) \big) \big) \subseteq U $. Mostrando que que $\mathcal{B}$ es base local para $\infty$ en $\ms{F}(\ms{A})$; y además $ |\mathcal{B}| \leq | [\ms{A}]^{<\omega} \times [\omega]^{<\omega} | \leq |\ms{A}| \cdot \aleph_0 = \aleph_0 + |\ms{A}| $. Lo anterior prueba que $\chi(\infty,\ms{F}(\ms{A})) \leq \aleph_0 + |\ms{A}|$.
	\end{proof}

	El siguiente Corolario se puede enriquecer con \ref{prop-tra-numerable}.

	\begin{corolario}\index[trad]{primero numerabilidad de $\ms{F}(\ms{A})$}
		Para toda familia no compacta $\ms{A}$, el espacio $\ms{F}(\ms{A})$ es primero numerable si y sólo si $|\ms{A}| \leq \aleph_0$.
	\end{corolario}

	El próximo Lema es clave por varios motivos; entre ellos, responde a una pregunta que sugiere la discusión previa a la \autoref{prop-CaracMADPositiv} ¿qué diferencía a los subconjuntos de $\omega$ casi ajenos con cada elemento de una familia casi ajena con aquellos en su parte positiva?.

	\begin{lema}\label{lem-convClave}
		Sean $\ms{A}\in \Ad(\omega)$ y $B \subseteq \Psi(\ms{A})$, entonces:
		\begin{enumerate}[i)]
			\item Si $B$ es numerable, $B \to \infty$ si y sólo si $B \subseteq^* \ms{A}$, o $B \cap \omega$ es infinito y casi ajeno con cada elemento de $\ms{A}$.
			\item $\infty \in \cla(B)$ si y sólo si $B \cap \ms{A}$ es infinito, o $B \cap \omega \in \ms{I}^+(\ms{A})$.
		\end{enumerate}
	\end{lema}

	\begin{proof} 
		(i) Supóngase que $B$ es numerable. Pruébese la suficiencia por contradiccion; esto es, asúmase que $B \to \infty$ pero $B \not\subseteq^* \ms{A}$ ($B\cap \ms{A}$ es infinito) y que existe cierto $a \in \ms{A}$ cuya interseccion con $B \cap \omega$ es infinita. Como $B\to \infty$ y $a \cap B = a \cap (B \cap \omega) \subseteq B$ es infinito, ocurre que $a \cap B \to \infty$. Sin embargo, $a \cap B \subseteq a$ es infinito y a razón del Lema \ref{lem-convObvia} se tiene que $a \to a$ en $\Psi(\ms{A})$; así mismo en $\ms{F}(\ms{A})$. Lo anterior implica que $a \cap B \to a$ y $a \cap B \to \infty$, siendo esto un absurdo.

		Conversamente, si $B \not\to \infty$, existe un compacto $K \subseteq \Psi(\ms{A})$ de modo que $B \not\subseteq^* U_K$; esto es, $B \cap K$ es infinito. Como $K \cap \ms{A}$ es finito, lo anterior prueba que $B \cap \omega \subseteq (B \cap \omega) \cap K$ es infinito, y así mismo, $B \not\subseteq^* \ms{A}$. Además $C:=(B \cap \omega) \cap K$ es un elemento en $\ms{I}(\ms{A})$; a consecuencia de esto y lo comentado antes de \ref{prop-CaracMADPositiv}, existe cierto $a \in A$ con $C \cap a$ infinito; de donde, $B \cap a$ es infinito.
		
		(ii) Para la suficiencia, supóngase que $\infty \in \cla(B)$ y que $B \cap \ms{A}$ es finito. Resulta necesario que $\infty \in \cla(B \cap \omega)$ y con ello $B \cap \omega \not\to \ms{I}(\ms{A})$. En efecto; de lo contrario, $B \cap \omega$ sería compacto y por ello lo tanto $\infty \in \cla(B \cap \omega) \subseteq B$, lo cual es imposible pues $\infty \notin \Psi(\ms{A})$. Así que $B \cap \omega \in \ms{I}^+(\ms{A})$.

		Para la necesidad, sí $B \cap \ms{A}$ es infinito, existe $C \subseteq B \cap \ms{A}$ numerable y por el inciso anterior $C \to \infty$, de donde $\infty \in \cla(C) \subseteq \cla(B)$. Por otra parte, si $B \cap \omega \in \ms{I}^+(\ms{A})$ y $K \subseteq \Psi(\ms{A})$ es compacto, resulta que $B \cap \omega \not\subseteq K$ y con ello $B\cap U_K \neq \emptyset$, mostrando que $\infty \in \cla(B)$.
	\end{proof}

	Si $\ms{A}$ es una familia casi ajena maximal e infinita, entonces la condicion (i) del Teorema anterior implica que las únicas sucesiones convergentes a $\infty$ son únicamente las (infinitas) contenidas en $\ms{A}$; por ello:
	\begin{corolario}\label{cor-convMaximal}
		Supóngase que $\ms{A}$ es una familia casi ajena maximal e infinita, entonces para cada $X \in [\ms{F}(\ms{A})]^\omega$:
		\begin{enumerate}
			\item $X$ es convergente si y sólo si $X \subseteq^* \ms{A}$, o, para algun $a \in \ms{A}$ se tiene que $X \subseteq^* a$.
			\item Si $X \subseteq \omega$ y $x \in \omega \cup \{\infty\}$, entonces $X \not\to x$.
			\item Si $B \subseteq \omega \cap \scl(X)$, entonces $B \subseteq X$.
		\end{enumerate}
	\end{corolario}

	\begin{proof} 
		(i) Sea $X \in [\ms{F}(\ms{A})]^\omega$ cualquiera. El recíproco es inmediato a razón de \ref{lem-convObvia} y el Lema previo. Para la suficiencia asúmase que $X \to x$ en el compacto de Franklin. Si $x = \infty$, se sigue del lema anterior y la maximalidad de $\ms{A}$ que $X \subseteq^* \ms{A}$. En otro caso, se puede suponer sin pérdida de generalidad que $X \subseteq \Psi(\ms{A})$, siguiéndose de \ref{lem-convObvia} que $X$ debe estar casi contenido en algún elemento de $\ms{A}$.

		(ii) y (iii) Se desprenden inmediatamente del inciso (i) y de que cada punto en $\omega$ es aislado (por lo que las sucesiones convergentes a puntos de $\omega$ son eventualmente constantes).
	\end{proof} 

	\begin{corolario}
		Sea $\ms{A} \in \Mad(\omega)$ infinita, entonces el orden secuencial de $\ms{F}(\ms{A})$ es $2$.
	\end{corolario}

	\begin{proof} 
		Nótese que $\scl^2(\omega) \not\subseteq \scl(\omega)$. Efectivamente; dado el Corolario anterior, $\scl(\omega) = \omega \cup \ms{A}$. Más aún, como $\ms{A}$ es infinita, contiene cierto subconjunto numerable $B \subseteq \ms{A}$. Y se obtiene de \ref{lem-convClave} que $B \to \infty$, esto muestra que $\infty \in \scl^2(\omega) \setminus \scl(\omega)$, por lo que $\Osq(\ms{F}(\ms{A})) \geq 2$.

		Ahora, sea $X \subseteq \ms{F}(\ms{A})$ cualquiera y supóngase que $x \in \scl^3(X)$. Como $\Psi(\ms{A})$ tiene orden secuencial $1$ (pues es $1\AN$, consecuentemente de Fréchet), es requisito que $x=\infty$. Así, existe $A \subseteq \scl^2(X)$ numerable tal que $A \to \infty$ en $\ms{F}(\ms{A})$. Por el \autoref{lem-convClave}, sin pérdida de generalidad, $A \subseteq \ms{A}$. Para cada $a \in A$ fíjese un conjunto numerable $B_a \subseteq \scl(X)$ de modo que $B_a \to a$. Se afirma que $\scl(X) \cap \ms{A}$ es infinito.

		Supóngase lo contrario, es decir, $\scl(X) \subseteq^* \omega$. Como cada $B_a \subseteq X$ es convergente, se puede asumir sin pérdedida de generalidad que $B_a \subseteq \omega$. En consecuencia de lo anterior, $B_a \subseteq \omega \cap \scl(X)$ y se obtiene del inciso (iii) del Corolario anterior que $B_a \subseteq X$. Así $A \subseteq \scl(X)$ ya que cada $B_a$ satisface $B_a \to a \in A$. Esto muestra que $x \in \scl(\scl(X))=\scl^2(X)$. Es decir, $\scl^3(X) \subseteq \scl^2(X)$ y $\Osq(\ms{F}(\ms{A})) \leq 2$.
	\end{proof}

	El posterior Teorema sigue la línea del Teorema de Kannan y Rajagopalan (\ref{teo-HLCCaract}), es una caracterización en propiedades topológicas de ciertos compactos de Franklin.
	\begin{teorema}\index[trad]{homeomorfismo con $\ms{F}(\ms{A})$ ($\ms{A}$ maximal)}
		Sea $X$ un espacio topológico infinito. $X$ es homeomorfo a un compacto de Franklin generado por una familia maximal infinita si y sólo si se satisfacen:
		\begin{enumerate}
			\item $X$ es compacto, de Hausdorff y separable, y
			\item Existe $x_0 \in X$ tal que $x_0$ es el único punto de acumulación de $\der(X)$, y, para cada $B \in [X \setminus \der(X)]^\omega$ se tiene $B \not\to x_0$.
		\end{enumerate}
		Claramente, en tal caso $x_0$ se identifica bajo algún homeomorfismo con el punto al infinito del compacto de Franklin.
	\end{teorema}

	\begin{proof} 
		Para la suficiencia basta suponer que $X=\ms{F}(\ms{A})$; con $\ms{A} \in \Mad(\omega)$ infinita, y probar (ii). Sea $x_0:=\infty$. Por ser $X$ la compactación unipuntual de $\Psi(\ms{A})$, se tiene que $\Psi(\ms{A})$ es un denso de $X$ y por ello $\der(X) = \{\infty\} \cup \der_{\Psi(\ms{A})}(\Psi(\ms{A}))$. De lo anterior y \ref{lem-primerosSubs} se tiene que $\der(X) = \{\infty\} \cup \ms{A}$; y además que $\ms{A}$ es discreto. En consecuencia, $\der_{\der(X)}(\der(X)) \subseteq \{\infty\}$ y la contención recíproca ocurre; pues cada subespacio compacto de $\Psi(\ms{A})$ tiene intersección finita; particularmente no vacía, con el conjunto (infinito) $\ms{A}$. Por lo que $\der(X)$ sólo se acumula en $x_0=\infty$. Ahora, si $B \in [X \setminus \der(X)]^\omega$, entonces $B \subseteq \omega$ y por la maximalidad de $\ms{A}$, se obtiene que $B \not\to \infty$ (utilizando el \autoref{cor-convMaximal}).

		Véase ahora la necesidad; esto es, supóngase que es compacto, de Hausdorff, separable y que $x_0$ actúa tal cual dicta (ii). Defínase $Y:=X\setminus \{x_0\}$, se mostrará primero que $Y \cong \Psi(\ms{A})$ para alguna familia maximal $\ms{A}$. Efectivamente, nótese que $Y$ es infinito, de Hausdorff y separable (ya que $Y$ es abierto al ser $\{x_0\}$ cerrado); así que haciendo uso del \autoref{cor-HLCPseudoCaract}, es suficiente mostrar los siguientes tres puntos:

		($Y$ es regular) Dado que $X$ es compacto, de Hausdorff es normal y particularmente, regular. Esto prueba que $Y \subseteq X$ es regular.

		($\der_Y(Y)$ es discreto) Efectivamente, si $y \in \der_Y(Y)$ es cualquiera, entonces $y \in Y$ es punto de acumulación de $X$. Como $y \neq x_0$ y $x_0$ es el único punto de acumulación de $\der_X(X)$, $\{y\}$ es abierto en $\der_X(X)$; y por tanto, $\{y\}$ es abierto en $\der_Y(Y)$. Mostrando que $\der_Y(Y)$ es discreto.

		(Si $B \subseteq Y$ es discreto, abierto y cerrado a la vez, entonces $B$ es finito) Supóngase que $B \subseteq Y$ es discreto, abierto y cerrado a la vez. Por ser $B$ discreto y abierto, se da $B \subseteq X \setminus \der(X)$. Ahora, si $B$ es infinito (sin pérdida de generalidad, numerable) se tiene de la hipótesis que $B \not\to x_0$; así, existe una vecindad de $x_0$; a saber U, de modo que $B \setminus U$ es infinito. Sin embargo, $B \cap U$ es cerrado en vista de que $B$ es cerrado; por ello, tal conjunto es cerrado, discreto e infinito en $X$; lo que contradice que $X$ sea compacto y $\T_1$. Por ello, es necesario que $B$ sea finito. Concluyéndose de \ref{cor-HLCPseudoCaract}, la existencia de una familia $\ms{A}\in\Mad(\omega)$ de modo que $Y \cong \Psi(\ms{A})$.

		Para finalizar, obsérvese que $\{x_0\}$ no es abierto en $X$, pues de lo contrario no podría ser punto de acumulación de ninguno de sus subespacios. Así, $Y$ es un subespacio denso de $X$ y como $X$ es de Hausdorff, compacto, con $X \setminus Y = \{x_0\}$, resulta que $X$ es la compactacion unipuntual de $Y \cong \Psi(\ms{A})$; esto es, $X \cong \ms{F}(\ms{A})$.
	\end{proof}

	\section{La propiedad de Fréchet}

	Continuando con los frutos del \autoref{lem-convClave}, se extrae el siguiente Corolario; este relaciona las propiedades de combinatoria de las familias casi ajenas con propiedades de convergencia.

	\begin{lema}\label{lem-TrazaMad}
		Si $\ms{A}$ es una familia no compacta, entonces para cada $X \subseteq \omega$ son equivalentes:
		\begin{enumerate}
			\item $\infty \in \scl(X)$.
			\item $\ms{A} \upharpoonright X \notin \Mad(X)$.
		\end{enumerate}
	\end{lema}

	\begin{proof} 
		(i) $\to$ (ii) Si $\infty \in \scl(X)$, entonces existe $B \subseteq X \subseteq \omega$ de modo que $B \to x$ y de acuerdo al \autoref{lem-convClave} se tiene garantizado que $B$ es casi ajeno con cada elemento de $\ms{A}$ (pues $B \cap \ms{A}$ es finito, por ser vacío). Entonces $B \in [X]^\omega$ es casi ajeno con cada elemento de $\ms{A} \upharpoonright X$; efectivamente, si $a \cap X \in \ms{A} \upharpoonright X$ es cualquiera, entonces $B \cap (X \cap a) = X \cap (a \cap B) \subseteq a \cap B =* \emptyset$. Mostrando que $\ms{A} \upharpoonright X$ no es maximal en $X$.

		(ii) $\to$ (i) Si $\ms{A} \upharpoonright X$ no es maximal en $X$, existe $B \subseteq X$ infinito y casi ajeno con cada elemento de $\ms{A} \upharpoonright X$. Nótese que entonces $B \cap X$ es casi ajeno con cada elemento de $\ms{A}$; y por lo tanto, $B \to \infty$ (por \ref{lem-convClave}). Por ello $\infty \in \scl(B) \subseteq \scl(X)$.
	\end{proof}

	De la \autoref{prop-CaracMADPositiv} se desprende fácilmente la contención:
	\[ \{X \in [\omega]^{\omega} \tq \forall A \in \ms{A} \: (A \cap X =^* \emptyset) \} \subseteq \ms{I}^+(\ms{A}) \]

	Resulta que la contención recíproca encapsula la conexión que existe entre la combinatoria de $\ms{A}$ y la propiedad de Fréchet de su compacto de Franklin asociado.

	\begin{corolario}\label{cor-TraFrechet}\index[trad]{propiedad de!Fréchet en $\ms{F}(\ms{A})$}
		Sea $\ms{A}$ una familia casi ajena no compacta. Son equivalentes:
		\begin{enumerate}
			\item $\ms{F}(\ms{A})$ es de Fréchet.
			\item $\{X \in [\omega]^{\omega} \tq \forall A \in \ms{A} \: (A \cap X =^* \emptyset) \} = \ms{I}^+(\ms{A})$.
			\item $\ms{A}$ es maximal en ninguna parte.
		\end{enumerate}
	\end{corolario}

	\begin{proof}
		(i) $\to$ (ii) Supóngase que $\ms{F}(\ms{A})$ es de Fréchet. Basta probar la contención recíproca de (ii). Y efectivamente, si $X \in \ms{I}^+(\ms{A})$, entonces $\infty \in \cla(X)$ debido \ref{lem-convClave}, pero como $\ms{F}(\ms{A})$ es de Fréchet, $\infty \in \scl(X)$; siguiéndose del mismo \autoref{lem-convClave}, que $X$ es casi ajeno con cada elemento de $\ms{A}$.

		(ii) $\to$ (iii) Supóngase (ii) y sea $X \in \ms{I}^+(\ms{A})$ cualquiera. Dada la hipótesis, $X$ es casi ajeno con cada elemento en $\ms{A}$, así que por \ref{lem-convClave}, $\infty \in \scl(X)$. Obteniéndose del \autoref{lem-TrazaMad}, que $\ms{A} \upharpoonright X \notin \Mad(X)$.

		(iii) $\to$ (i) Supóngase que $\ms{A}$ es maximal en niguna parte. Como $\Psi(\ms{A})$ es de Fréchet (por ser primero numerable) basta verificar la propiedad de Fréchet en $\infty \in \ms{F}(\ms{A})$. Sea $X \subseteq \ms{F}(\ms{A})$ de modo que $\infty \in \cla(X)$, entonces por \ref{lem-convClave}, $X \cap \ms{A}$ es infinito o $X \cap \omega \in \ms{I}^+(\ms{A})$. Si ocurre lo primero, sea $B \subseteq X \cap \ms{A}$ numerable y nótese que entonces $B \to \infty$, lo cual basta para mostrar que $\infty \in \scl(X)$. Si ocurre el segundo caso, de la hipótesis se obtiene $A \upharpoonright (X \cap \omega) \notin \Mad(X \cap \omega)$, probando que $\infty \in \scl(X \cap \omega) \subseteq \scl$ (en virtud \ref{lem-TrazaMad}). En ambos casos, $\infty \in \scl(X)$; y por tanto $\scl(X)=\cla(X)$.
	\end{proof}

	El Corolario anterior puede ser empleado para solucionar un problema clásico en topología general; determinar si el producto de dos espacios de Fréchet es de Fréchet. Los espacios de Mrówka dejan ver su ``maleabilidad'' al momento de generar contraejemplos a través de la subsecuente hilación de ideas.

	\begin{proposicion}
		Sea $\ms{A}$ una familia casi ajena, unión ajena de las familias no vacías $\ms{B}$ y $\ms{C}$. Si $\ms{A}$ es maximal en alguna parte, entonces $\ms{F}(\ms{B}) \times \ms{F}(\ms{C})$ no es de Fréchet.
	\end{proposicion}

	\begin{proof} 
		Supóngase que existe $X \in\ms{I}^+(\ms{A})$ de modo que $\ms{A} \upharpoonright X \in \Mad(X)$ y sea $B:=\{ (n,n) \given n \in X \}$.

		Como $X \in \ms{I}^+(\ms{A})$ y $\ms{B},\ms{C} \subseteq \ms{A}$, resulta que $X \in \ms{I}^+(\ms{B})$ y $X \in \ms{I}^+(\ms{C})$ (véase \ref{prop-TrazaBasicos}). Entonces se tiene que $\infty_\ms{B} \in \cla_{\ms{F}(\ms{B})}$ y $\infty_\ms{C} \in \cla_{\ms{F}(\ms{C})}$ a consecuencia del \autoref{lem-convClave}. De este modo:
		$$ (\infty_\ms{B},\infty_C) \in \cla_{\ms{F}(\ms{B}) \times \ms{F}(\ms{C})}(B) $$
		sin embargo $(\infty_\ms{B},\infty_\ms{C}) \notin \scl_{\ms{F}(\ms{B}) \times \ms{F}(\ms{C})}(B)$. Efectivamente, de lo contrario, existe $Y \in [X]^\omega$ de manera que $\{(n,n) \tq n \in Y\}$ converge a $ (\infty_\ms{B},\infty_C)$. De lo anterior, y la continuidad de las funciones proyección, se obtiene que $Y \to \infty_\ms{B}$ en $\ms{F}(\ms{B})$ y $Y \to \infty_\ms{C}$ en $\ms{F}(\ms{C})$. Sin embargo a consecuencia de ello; y por \ref{lem-convClave}, $Y \subseteq X$ es infinito y casi ajeno con cada elemento de $\ms{B}$ y $\ms{C}$; es decir, con cada elemento de $\ms{A}\ms{B} \cup \ms{C}$, siendo esto una contradicción a la maximalidad de $\ms{A} \upharpoonright X$ en $X$.

		Por lo tanto $(\infty_\ms{B},\infty_\ms{C}) \notin \scl_{\ms{F}(\ms{B}) \times \ms{F}(\ms{C})}(B)$ y el producto $\ms{F}(\ms{B}) \times \ms{F}(\ms{C})$ no tiene la propiedad de Fréchet.
	\end{proof}
	
	Combinando con el Teorema de Simon (\ref{Teo-Simon}), se tiene la siguiente fuente de contrajemplos: Cada vez que $\ms{A}$ sea una familia infinita y maximal (por ello, no compacta), se pueden dar dos familias $\ms{B} \subseteq \ms{C}$ no vacías, maximales en ninguna parte, de modo que $\ms{A}=\ms{B} \cup \ms{C}$. Y se desprende de la Proposición previa que $\ms{F}(\ms{B}) \times \ms{F}(\ms{C})$ no es de Fréchet; pues claramente $\ms{A}$ es maximal en alguna parte, ya que $\omega \in \ms{I}^+(\ms{A})$ (por \ref{cor-IdealPropioCaract}); y además, $\ms{F}(\ms{B})$ y $\ms{F}(\ms{C})$ son ambos de Fréchet (por \ref{cor-TraFrechet}). Esto implica:

	\begin{corolario}\label{cor-FrechNoProd}
		Existen dos espacios de Mrówka cuyas compactaciones unipuntuales son de Fréchet, pero su producto no

		En particular, la propiedad de Fréchet no es finitamente productiva; ni siquiera en la clase de espacios compactos, de Hausdorff.
	\end{corolario}

	Dado el Teorema de Simon (\ref{Teo-Simon}), toda familia maximal de tamaño $\kappa$ contiene una familia maximal en ninguna parte de cardinalidad, también $\kappa$. De la caracterización dada en \ref{cor-TraFrechet} y la \autoref{prop-caracterFrechet}, se obtiene:

	\begin{corolario}
		Si existe una familia maximal de tamaño $\kappa$, existe un espacio de Fréchet tal que uno de sus puntos tiene carácter $\kappa$.

		Particularmente, existe un espacio de Fréchet, que contiene un punto de carácter $\mathfrak{c}$.
	\end{corolario}
        \chapter{Normalidad en los espacios de Mrówka}

    \index[alph]{Conjetura!de Moore}\index[alph]{Conjetura!débil de Moore}\index[alph]{Moore!conjetura de}\index[alph]{Moore!conjetura débil de}\index[sym]{$\Pm$}\index[sym]{$\Pdm$}
    \emph{\small La hoy conocida como <<Conjetura de Moore>> (\,$\Pm$), establece que todo espacio de Moore normal es metrizable; se trata de un prblema lanzado a la comunidad matemática por Jones en 1933 que atiende a la cuestión ¿qué requiere un espacio de Moore para ser metrizable?. Este problema marcó un antes y un despues para la topología general, consolidándose como uno de los problemas (sino el que más) importantes en la topología y la teoría de conjuntos. $\Pm$ tiene, presumiblemente, una solución independiente a la axiomática $\zfc$ (\cite[p.~429-435]{nyikosMoore}).}
   
    \emph{\small En el año 1937 (véase \cite[Teo~5, p.~ 676]{jonesCM}), el propio Jones muestra la consistencia de la <<Conjetura Débil de Moore>> (\,$\Pdm$); esto es, cualquier espacio separable, normal y de Mooore, debe ser metrizable. Pero no sería sino hasta 1969 cuando Tall, en su tesis doctoral \cite{tallTesis}, logra establecer una equivalencia para $\Pdm$ en términos de la existencia ciertos espacios topológicos ($Q$-sets, \autoref{def-Qset}) no numerables; mismos para los cuales, Silver mostró consistente su existencia.}

    \emph{\small La meta de este capítulo será exponer las contribuciones de Jones, Silver y Tall; las cuales conjuntamente, permiten mostrar la independecia de $\Pdm$ de la axiomática usual de $\zfc$.}

    \section{Independencia de la Conjetura Débil de Moore}

    Por otra parte, todo espacio de Mrówka es de Moore y separable (\autoref{cor-MrwokaSiempre}); así que el enunciado $\Pdm$ (por consiguiente, $\Pm$) implica que ningún espacio $\Psi(\ms{A})$; con $\ms{A}$ una familia casi ajena más que numerable, puede ser normal. Lo que ataña a la presente sección; y claramente presenta una dificultad mayor, es mostrar el recíproco de la anterior implicación.

    Se comenzará exponiendo una condición necesaria que dicta ''dónde buscar'' espacios de Mrówka que sirvan de contrajemplo para $\Pdm$.
    
    \begin{proposicion}\label{pro-ObsNormalidad}
        Sea $\ms{A} \in \Ad(\omega)$, se cumple:
        \begin{enumerate}
            \item Si $|\ms{A}| \leq \aleph_0$, entonces $\Psi(\ms{A})$ es normal.
            \item Si $\ms{A}$ es infinita y $\Psi(\ms{A})$ es normal, $\ms{A}$ no es maximal y además $\aleph_1 \leq |\ms{A}|< \mathfrak{c}$.
        \end{enumerate}
    \end{proposicion}
    \begin{proof}
        (i) Cualquier espacio de Mrówka numerable es metrizable (por \ref{prop-tra-numerable}), particularmente normal (\textbf{Ree BSB}).

        (ii) Supóngase que $\ms{A}$ es infinita y que $\Psi(\ms{A})$ es normal. Si $\ms{A}$ fuera maximal, entonces por \ref{prop-tra-pseudoCaract} y \ref{prop-tra-compacidad}, se tiene que $\Psi(\ms{A})$ es pseudocompacto pero no numerablemente compacto. Lo cual (por el \textbf{RAKA}) imposibilita que $\Psi(\ms{A})$ sea normal. Por tanto, $\ms{A}$ no es maximal. Esto también implica que $\aleph_1 \leq |\ms{A}|$ (en virtud de la \autoref{prop-MADnoNum}).
        
        Finalmente, $|\ms{A}|=\mathfrak{c}$, debido a \ref{lem-primerosSubs}, $\ms{A}$ es un subespacio cerrado y discreto de $\Psi(\ms{A})$ de tamaño $\mathfrak{c}$. Así, de la separabilidad y normalidad de $\Psi(\ms{A})$ se desprende; por el Lema de Jones (léase \textbf{TAL}), que $2^\mathfrak{c} = 2^{|\ms{A}|} \leq 2^{\aleph_0} = \mathfrak{c}$, lo cual es imposible. Por lo tanto $|\ms{A}|< \mathfrak{c}$.
    \end{proof}

    \begin{corolario}\label{cor-HCnoMrowkasNormales}
        Bajo $\HC$; ningún espacio de Mrówka más que numerable, es normal.
    \end{corolario}

    Lo subsecuente caracteriza; en términos simples, la normalidad de un espacio de Mróka, a través de la combinatoria de su familia asociada.

    \begin{proposicion}\label{pro-ParticionadorCerrados}
        Sean $\ms{A}$ una familia casi ajena y $F,G \subseteq \Psi(\omega)$ cerrados ajenos. Son equivalentes:
        \begin{enumerate}
            \item $F$ y $G$ se separan por abiertos ajenos de $\Psi(\ms{A})$.
            \item $F \cap \ms{A}$ y $G \cap \ms{A}$ se separan por abiertos ajenos de $\Psi(\ms{A})$.
            \item La grieta $(F \cap \ms{A},G \cap \ms{A})$ está separada.
        \end{enumerate}
    \end{proposicion}

    \begin{proof}
        La implicación (i) $\to$ (ii) es inmediata.

        (ii) $\to$ (iii) Supóngase que $U,V \subseteq \Psi(\ms{A})$ son abiertos ajenos tales que $F \cap \ms{A} \subseteq U$ y $G \cap \ms{A} \subseteq U$.
        
        Si $a \in F \cap \ms{A}$ es cualquiera, entonces $a \in U$ y por definición de la topología en $\Psi(\ms{A})$ resulta que $a \subseteq^* U$. Ahora, si $b \in G \cap \ms{A}$ es cualquiera, entonces $b \subseteq^* V \subseteq \omega \setminus U$; de donde, $b \cap U =* \emptyset$. Por lo tanto, $U$ es particionador de $F \cap \ms{A}$ y $G \cap \ms{A}$.

        (iii) $\to$ (i) Supóngase que $D \subseteq \omega$ es particionador de $F \cap \ms{A}$ y $G \cap \ms{A}$. Nótese que $F \subseteq U:=F \cup D\setminus G$; y además, $U$ es abierto. Efectivamente, dado $a \in U \cap \ms{A} \subseteq F \cap \ms{A}$ se tiene que $a \subseteq^* D$ (por ser $D$ particionador de $F \cap \ms{A}$ y $G \cap \ms{A}$) y $a \subseteq^* \Psi(\ms{A}) \setminus G$ (por ser $G$ cerrado y ajeno a $F$), en consecuencia $a \subseteq^* D\setminus G \subseteq U$.

        Como $D$ es particionador de $F \cap \ms{A}$ y $G \cap \ms{A}$; $\omega \setminus D$ es particionador de $G\cap \ms{A}$ y $F \cap \ms{A}$, y resulta análogo que $G \subseteq V:=G \cup (\omega \setminus D)\setminus F$ y $V$ es abierto. Probando que $F$ y $G$ se separan por los abiertos ajenos $U$ y $V$.
    \end{proof}

    Debido al \autoref{lem-primerosSubs} y la \autoref{obs-GrietasSimple}, se deprende:

    \begin{corolario}\label{cor-tra-NormalParticionador}\index[trad]{Normalidad de $\Psi(\ms{A})$ (con grietas separables)}
        Para cada familia casi ajena $\ms{A}$ son equivalentes:
        \begin{enumerate}
            \item $\Psi(\ms{A})$ es normal.
            \item Para cada $\ms{B} \subseteq \ms{A}$, la grieta $(\ms{B},\ms{A} \setminus \ms{B})$ está separada.
        \end{enumerate}
    \end{corolario}

    La traducción de la \autoref{pro-ObsNormalidad} en términos combinatorios es la siguiente proposición (lo cual, por cierto, demuestra el \autoref{ej-interrelacion} de la \autoref{Sec-Luzin}):

    \begin{corolario}\label{col-tra-interrelacion}
        Sea $\ms{C}\in \Ad(\omega)$, entonces:
        \begin{enumerate}
            \item Si $|\ms{C}|\leq \aleph_0$, toda grieta contenida en $\ms{C}$ está separada.
            \item Si $\ms{C}$ es inifnita y, $|\ms{C}|=\mathfrak{c}$ o $\ms{C}\in \Mad(\omega)$; entonces $\ms{C}$ contiene una grieta que no está separada.
        \end{enumerate}
    \end{corolario}
    
    En términos topológicos, todo espacio $\Psi(\ms{A})$ no normal (con $\ms{A}$ infinita) contiene dos cerrados ajenos que no se pueden separar por abiertos ajenos, y en virtud del punto (i) del anterior corolario, alguno de ellos debe ser no numerable. Surge la siguiente cuestión:, ¿cúandoque ninún par de cerrados ajenos no numerables se pueden separar?.

    El análisis expuesto en \autoref{Sec-Luzin} establece que estos espacios son, exactamente, aquellos generados por una familia inseparable (en el sentido de lo compentado en la \pageref{def-FamInseparable}), particularmente:

    \begin{corolario}\label{cor-MrowkaLuzin}
        Si $\ms{A}$ es una familia de Luzin, ninún par de cerrados ajenos no numerables de $\Psi(\ms{A})$ se pueden separar por abiertos ajenos.

        Particularmente, hay un espacio de Mrówka de tamaño $\aleph_1$ no normal.
    \end{corolario}

    \subsection{Consistencia de \textsf{WMC}}
    \label{Sec-PDM}

    Siguiendo la técnica de Tall, para probar la independencia de la Conjetura Débil de Moore (de $\zfc$), se utilizarán a modo de intermediario los espacios metrizables conocidos como $Q$-sets.

    \begin{definicion}\label{def-Qset}\index[alph]{$Q$-set}
        Un $Q$-set es un espacio metrizable, separable y tal que todos sus subespacios son de tipo $G_\delta$ (equivalentemente; $F_\sigma$).
    \end{definicion}

    \begin{ejemplo}\label{ej-QsetFacil}
        Cualquier espacio $X$ a lo más numerable y metrizable es un $Q$-set. Efectivamente, nótese que $X$ es separable. Y además, si $A \subseteq X$ es cualquiera, entonces $A=\midcup\Set*{\{a\} \given a \in X}$ es de tipo $F_\sigma$.
    \end{ejemplo}

    Se comenzará por observar que todo $Q$-set es; salvo homeomorfismos, un subespacio de $\mathbb{R}$ (o del conjunto de cantor, $2^\omega$). El siguiente lema, incluido en \cite[Teo.~1, p.~286]{kuratowskiTopology} por Kuratowski; se enunciará y demostrará con terminología moderna.
    \begin{lema}
        Sea $X$ un espacio metrizable por la métrica $d$. Si $|X|<\mathfrak{c}$, entonces $X$ es cero-dimensional.
    \end{lema}
    \begin{proof}
        Supóngase que $|X|<\mathfrak{c}$. Basta corroborar que cada $x \in X$ admite una base local de abiertos y cerrados. Sean $x \in X$ y $\varepsilon>0$.

        Supóngase ahora que para cada $\delta \in (0,\varepsilon)$, el conjunto $\fron(B(x,\delta))$ es no vacío, y fíjese ($\Ac$) un elemento $x_\delta \in \fron(B(x,\delta)) \subseteq X$. Como $|(0,\varepsilon)|=\mathfrak{c}$, la asignación $\delta \to x_\delta$ no puede ser inyectiva. Consecuentemente, existen distintos $\delta,\delta'\in (0,\varepsilon)$ de modo que $\fron(B(x,\delta)) \cap \fron(B(x,\delta')) \neq \emptyset$. Pero esto es imposible, dado que $\delta \neq \delta'$.
        
        Por lo tanto, para cada $\varepsilon>0$ se puede fijar ($\Ac$) cierto $\delta_\varepsilon \in (0,\varepsilon)$ tal que $\fron(B(x,\delta_\varepsilon))=\emptyset$; esto es, $B(x,\delta_\varepsilon)$ es abierto y cerrado a la vez. Claramente $\{B(x,\delta_\varepsilon) \tq \varepsilon>0\}$ es una base local para $x$ en $X$.
    \end{proof}

    \begin{proposicion}\label{prop-QsetEquivs}
        Para todo espacio $X$ son equivalentes:
        \begin{enumerate}
            \item $X$ es un $Q$-set.
            \item $X$ se encaja en $2^\omega$ y todos sus subespacios son de tipo $G_\delta$.
            \item $X$ se encaja en $\mathbb{R}$ y todos sus subespacios son de tipo $G_\delta$.
        \end{enumerate}
    \end{proposicion}
    \begin{proof}
        Dada la universalidad del conjunto de Cantor, $2^\omega \subseteq \mathbb{R}$, sobre la clase de espacios cero-dimensionales; y que todo subespacio de $\mathbb{R}$ es metrizable y separable, basta probar que todo $Q$-set es cero-dimensional.

        Supóngase que $X$ es un $Q$-set, como $X$ es metrizable y separable, entonces es $2\AN$. Sea $\mathcal{B}$ una base a lo más numerable $\mathcal{B}$ para $X$.

        Como $X$ es $Q$-set, para cada $A \subseteq X$ fíjese ($\Ac$) una colección a lo más numerable de abiertos $\mathcal{U}$ de modo que $A=\midcap \mathcal{U}$. Y como $\mathcal{B}$ es base; de nuevo haciendo uso de $\Ac$, para cada abierto $U$ fíjese $\mathcal{B}_U \subseteq \mathcal{B}$ de modo que $U=\midcup \mathcal{B}_U$.

        Lo anterior permite definir $\ms{P}(X) \to [\ms{P}(\mathcal{B})]^{\leq \omega}$ por medio de la correpondencia: $A \mapsto \{B_U \tq U \in \ms{A}\}$. Nótese que tal asignación es inyectiva, pues si $\{B_U \tq U \in \ms{A}\}=\{B_U \tq U \in \ms{B}\}$, entonces:
        \[ \mathcal{U}_A = \{\midcap \mathcal{B}_U \tq U \in \mathcal{U}_A\} = \{\midcap \mathcal{B}_U \tq U \in \mathcal{U}_B\} = \mathcal{U}_A \]
        y con ello $A=\midcap \mathcal{U}_A = \midcap \mathcal{U}_B = B$. De esta manera:
        \[ 2^{|X|} \leq \left| [\ms{P}(\mathcal{B})]^{\leq \omega} \right| \leq \left( 2^{|B|} \right)^{\aleph_0} \leq \left( 2^{\aleph_0} \right) ^{\aleph_0} = 2^{\aleph_0 \cdot \aleph_2} = 2^{\aleph_0} = \mathfrak{c} \]

        La última desigualdad implica que $|X| < \mathfrak{c}$. Siguiéndose del Lema previo, la cero-dimensionalidad de $X$.
    \end{proof}

    \begin{observacion}\label{obs-HCNoQset}
        Todo $Q$-set tiene tamaño menor que $\mathfrak{c}$. Consecuentemente, bajo $\HC$; no existen $Q$-sets más que numerables.
    \end{observacion}

    El paralelismo del resultado anterior con el \autoref{cor-HCnoMrowkasNormales} no es coincidencia. La meta ahora es mostrar que la existencia de $Q$-sets no numerables es equivalente a la existencia de espacios de Mrówka no numerables y normales; más aún, si estos espacios no existen, entonces $\Pdm$ se satisface.

    \begin{lema}\label{coro-Bing}
        Supóngase que $X$ es normal, de Moore, no metrizable y que $D \subseteq X$ es denso a lo más numerable; entonces existe $A \subseteq X \setminus D$ más que numerable, discreto y cerrado en $X$.
    \end{lema}

    \begin{proof}
        El Tereoma de Bing (\textbf{BINGEEE}) caracteriza la metrización de los espacios de Moore a través de la normalidad colectiva. Por ello, $X$ no es colectivamente normal y existe una familia discreta $\mathcal{A}$ de cerrados de $X$, cuyos elementos no se pueden separar por abiertos ajenos.
    
        Como $X$ es normal, para cada par de cerrados ajenos de $X$; digamos $F$ y $G$, elíjanse ($\Ac$) abiertos ajenos $W(F,G),S(F,G)$ de modo que $F\subseteq W(F,G)$ y $G \subseteq S(F,G)$. Es claro que $\mathcal{A}$ no puede ser finito.

        \begin{enumerate}[\hspace{1.5 cm}, listparindent=1.5em]
            \item \textit{Afirmación.} $\mathcal{A}$ es más que numerable.
            
            \item \textit{Demostración.} Supóngase que $\mathcal{A}$ está enumerado inyectivamente como $\{A_n \tq n \in \omega\}$. Por ser $\mathcal{A}$ familia discreta de cerrados, si $n \in \omega$, entonces $B_n:=\midcup\{A_m \tq m>n\}$ es cerrado.
            
            Por recursión, sean $U_0:=W(A_0,B_0)$ y $V_0:=S(A_0,B_0)$; y, para $n \in \omega$, $U_{n+1}:=W(A_{n+1},B_{n+1}) \cap V_n$ y $V_{n+1}= S(A_{n+1},B_{n+1}) \cap V_n$.

            Por construcción, $\{U_n \tq n\in \omega\}$ es una familia de abiertos, ajenos por pares tales que para cada $n \in \omega$ se tiene $A_n \subseteq U_n$. Así, los elementos de $\mathcal{A}$ se separan por abiertos ajenos; contradiciendo su elección. \hfill$\boxtimes$
        \end{enumerate}

        Dada la afirmación anterior, y fijando para cada $a \in \mathcal{A}$ un elemento $x_a \in \mathcal{A}$, se obtiene un conjunto mas que numerable $B:=\{x_a \tq a \in \mathcal{A}\}$; mismo que por ser $\ms{A}$ familia discreta y $X$ de Hausdorff, resulta ser cerrado y discreto.

        Por último nótese que cada subespacio de $B$ es discreto y cerrado en $X$; pues $B$ es discreto y cerrado en $X$. Particularmente, $A:=B \setminus D$ es discreto, discreto en $X$ y no numerable (pues $B$ es más que numerable y $D$ es numerable).
    \end{proof}

    El siguiente teorema aparece en la tesis doctoral de Franklin David Tall (ver \cite{tallTesis}), y es la pieza clave para atacar la Conjetura Débil de Moore.

    \begin{teorema}[Tall]\label{teo-EquivPDM}
        Si $\kappa$ es un cardinal infinito, son equivalentes:
        \begin{enumerate}
            \item Existe un espacio de Moore, normal, no metrizable de tamaño $\kappa$.
            \item Existe un espacio de Mrówka normal de tamaño $\kappa$.
            \item Existe un $Q$-set de tamaño $\kappa$.
        \end{enumerate}
    \end{teorema}
    \begin{proof}
        (i) $\to$ (ii) Supóngase que $X$ es un espacio, normal, de Moore y no metrizable de tamaño $\kappa$ y sea fíjese $D\subseteq X$ denso numerable de $X$. Por \ref{coro-Bing}, $D$ es infinito y existe un subespacio $A \subseteq X \setminus D$ más que numerable; discreto y cerrado de $X$. Como $X$ es de Hausdorff y primero numerable, considérese $ \ms{A}_{D,A}=\{A_x \in [D]^\omega \tq x \in A\} $; la familia de sucesiones en $D$ convergentes a $A$ (definida en \ref{def-FamSucesiones}), donde cada $A_x$ converge a $x$. Por la \autoref{prop-famSucesiones}, $|\ms{A}_{D,A}|=\kappa$ y así mismo, $\Psi_D(\ms{A}_{D,A})=\kappa$.
        
        Sea $\ms{B}:=\{A_x \tq x \in F\} \subseteq \ms{A}_{D,A}$ cualquiera. Como $A$ es discreto y cerrado en $X$, cualquiera de sus subespacios es cerrado en $X$; en consecuencia y por normalidad de $X$, existen abiertos $U,V \subseteq X$ ajenos, de modo que $F \subseteq U$ y $A \setminus F \subseteq V$. Si $x \in F$, entonces $A_x \to x$ y por ello $A_x \subseteq^* U$, similarmente, si $y \in A \setminus F$, entonces $A_y \subseteq^* V \subseteq X \setminus U$; de donde $A_y \cap U^* = \emptyset$. 

        Por tanto $(\ms{B}, \ms{A}_{D,A} \setminus \ms{B})$ está separada, obteniéndose de \ref{cor-tra-NormalParticionador} la normalidad de $\Psi_D(\ms{A}_{D,A})$.
        
        (ii) $\to$ (iii) Si $\kappa = \omega$, la implicación resulta vacua; pues todo subespacio numerable de $2^\omega$ es un $Q$-set (\autoref{ej-QsetFacil}). Supóngase pues, que $\Psi(\ms{A})$ es un espacio normal de tamaño $\kappa>\omega$; siendo necesario que $|\ms{A}|=\kappa$.
        
        Para cada $C \subseteq \omega$ denótese por $\varphi_C \in 2^ \omega$ a la función característica de $C$ y sea $X:=\{\varphi_A \in 2^\omega \tq A \in \ms{A} \}$. Obsérvese que $X$ es un espacio metrizable, separable (por ser subespacio del metrizable, separable, $2^\omega$) de tamaño $\kappa$. 
        
        Sea $Y = \{ \varphi_A \in X \tq A \in \ms{B} \} \subseteq X$ cualquiera. Dado el \autoref{cor-tra-NormalParticionador}, la normalidad de $\Psi(\ms{A})$ implica la existencia de un particionador de $\ms{A}\setminus \ms{B}$ y $\ms{B}$; de este modo:
        \begin{align*}
            Y   & = \{ \varphi_A \in X \tq A \in \ms{B} \} \\
                & = \{ \varphi_A \in X \tq A \cap D \neq^* \emptyset \} \\
                & = \{ \varphi_A \in X \tq \forall n \in \omega \: ( A \cap D \not\subseteq n) \} \\
                & = \bigcap_{n \in \omega} \{\varphi_A \in X \tq A \cap D \not\subseteq n\}
        \end{align*}
        
        Ahora, si $n \in \omega$ y $\varphi \in U_n:=\{\varphi_A \in X \tq A \cap D \not\subseteq n\}$ es cualquiera, existe cierto $k \in (A \cap D) \setminus n$. Por ello, si $x=\varphi_B \in X$ es tal que $x(k)=1$, entonces $k \in (B \cap D) \setminus n$ y $B \cap D \not \subseteq n$; es decir $f \in U_n$. Mostrando así que $\{ x \in X \tq x\upharpoonright \{k\} = \varphi \upharpoonright \{k\} \} \subseteq U_n$. Por lo tanto, cada $U_n$ es abierto en $X$. De esta manera, cualquier $Y \subseteq X$ es $G_\delta$ en $X$ y $X$ es un $Q$-set.

        (iii) $\to$ (i) Supóngase que $X$ es un $Q$-set de tamaño $\kappa$. En virtud del \autoref{prop-QsetEquivs}, supóngase sin pérdida de generalidad que $X \subseteq 2^\omega$. Para cada $E \subseteq X$ considérese $ \ms{A}_E:=\{ A_x \in [N]^\omega \tq x \in E \} $, la familia de las ramas de $E$ en $N$ (definida en \ref{def-FamRamas}); donde $N:=2^{<\omega}$ y cada $A_x$ es el conjunto $\{x \upharpoonright n \tq n \in \omega\} \subseteq N$. Puesto que $|X|=\kappa$, del \autoref{def-FamRamas} se sigue que $|\ms{A}_X|=\kappa$, y con ello $|\Psi_N(\ms{A}_X)|=\kappa$.
        
        Sea $Y \subseteq X$ cualquiera. Como $X$ es un $Q$-set, $Y=\midcup\{F_n \tq n \in \omega\}$ y $X \setminus Y=\midcup\{G_n \tq n\in \omega\}$; donde cada conjunto $F_n$ y $G_n$ es cerrado en $X\subseteq 2^\omega$. Para cada $n \in \omega$ defínanse los conjuntos:
        \begin{align*}
            D_n := \left( \midcup \ms{A}_{F_n} \right) \setminus \bigcup_{m<n} \left( \midcup \ms{A}_{G_m} \right) \\
            L_n := \left( \midcup \ms{A}_{G_n} \right) \setminus \bigcup_{m\leq n} \left( \midcup \ms{A}_{F_m} \right) 
        \end{align*}
        y sea $D:=\midcup\{D_n \tq n \in \omega\}$. Nótese que por construccion, si $m,n \in \omega$, se tiene $D_n \cap L_m = \emptyset$; consecuentemente, cada $L_n$ es ajeno con $D$.

        Sea $y \in A_Y$; entonces existe $n \in \omega$ de modo que $y \in F_n$. Por otra parte, cada $G_m\subseteq X\setminus Y$ (con $m<n$) es cerrado en $X$, por lo que existe $s \in \omega$ de modo que:
        $$ \{ x \in X \tq x \upharpoonright s = y \upharpoonright s \} \subseteq X \setminus \bigcup_{m<n} G_m $$
        
        Por ello, si $v \in A_y \setminus D_n\subseteq F_n$, existen $x \in F_n$ y $k \in \omega$ de modo que $v = x \upharpoonright k$. Así que $x \in \midcup \ms{A}_{F_n}$; y como $v \notin D_n$, existen $m<n$ y $g \in G_m$ de manera que $v = y \upharpoonright k = g \upharpoonright k$. Y a razón de ello, no puede ocurrir $s \subseteq k$. Por lo tanto $k<s$ y $A_y \setminus D_n \subseteq 2^{<s} =^* \emptyset$; esto es, $A_y \subseteq^* D_n \subseteq D$.

        Similarmente, para cada $y \in X \setminus Y$ existe un $n \in \omega$ tal que $A_y \subseteq^* L_n \subseteq N \setminus D$; de donde, $A_y \cap D ^= \emptyset$. Así que $D$ es separador de $\ms{B}$ y $\ms{A} \setminus \ms{B}$; probando por el \autoref{cor-tra-NormalParticionador} la normalidad de $\Psi_N(\ms{A}_X)$.
    \end{proof}
    
    Es inmediato al \autoref{teo-EquivPDM} (y a \ref{cor-HCnoMrowkasNormales}, o bien, \ref{obs-HCNoQset}) la siguiente consecuencia:

    \begin{corolario}\label{cor-PdmConsistente}
        Bajo $\HC$; se cumple $\Pdm$, y:
        \begin{enumerate}
            \item Ningún espacio de Mrówka no numerable es normal.
            \item Ningún $Q$-set es más que numerable.
        \end{enumerate}
        Consecuentemente $\Pdm$ es consistente con $\zfc$.
    \end{corolario}
    
    \subsection{Consistencia de \texorpdfstring{$\lnot$}\textsf{WMC}}
    
    Para la segunda parte de la prueba de independencia de $\Pdm$ se hará uso; como es previsible desde anteriores capítulos, de la negacion de $\HC$ con el Axioma de Martin. Se comenzará observando cómo se pueden caracterizar a todos los $Q$-sets haciendo uso del Lema de Solovay y $\Ma$.

    \begin{lema}
        Sea $X$ un espacio metrizable y separable. Entonces existe una base $\mathcal{B}=\{B_n \tq n \in \omega\}$ para $X$ de modo que $\{A_x \tq x \in X\}$ es familia casi ajena; donde, cada $A_x$ es $\{n \in \omega \tq x \in B_n\}$.
    \end{lema}
    \begin{proof}
        Por el teorema (\textbf{Arhangel’skii}), $X$ admite una base regular (véase \textbf{BsRg}) $\mathcal{C}$. Como $X$ es $2\AN$ (a consecuencia de ser metrizable y separable \textbf{VeR}), existe una base $\mathcal{B}=\{B_n \tq n \in \omega\} \subseteq \mathcal{C}$, claramente $\mathcal{B}$ sigue siendo regular.
        
        Dados $x,y \in X$ son distintos, sean $U,V$ abiertos ajenos que separan a $x$ y $y$. Por regulardidad de la base $\mathcal{B}$, existe $W \subseteq U$ abierto con $x\in W$ y
        $\mathcal{B}_W:=\{n \in \omega \tq B_n \cap W \neq \emptyset \land B_n \setminus W \neq \emptyset\} =^* \emptyset$. Por consiguiente, el conjunto $A_x \cap A_y \subseteq \mathcal{B}_W$ es finito.
    \end{proof}

    \begin{proposicion}[Tall, Silver]\label{pro-MaQsetChar}
        Sea $X$ espacio topológico. Bajo $\Ma$; $X$ es un $Q$-set si y sólo si es homeomorfo a un subespacio $X \in [\mathbb{R}]^{<\mathfrak{c}}$.
    \end{proposicion}
        \begin{proof}
            Supóngase $\Ma$. La suficiencia viene dada por \ref{teo-EquivPDM} y \ref{prop-QsetEquivs}. Para la necesidad supóngase que $X \in [\mathbb{R}]^{<\mathfrak{c}}$, si $X$ es a lo más numerable, de \ref{teo-EquivPDM} y \ref{pro-ObsNormalidad} se sigue que $X$ es un $Q$-set.
            
            Como $X$ es metrizable y sepable, sean $\mathcal{B}$, $B_x$ (para cada $x \in X$) y $\ms{A}$ como en el Lema previo. Tómense $Y \subseteq X$ cualquiera y $B:=\{B_y \in \ms{A} \tq y \in Y\}$.
            
            Como $|\ms{A} \setminus \ms{B}|,|\ms{B}|<\mathfrak{c}$ y se cumple $\Ma$, del \autoref{lem-Solovay} se desprende la existencia de cierto $D \subseteq \omega$ de modo que para cada $y \in Y$ y $x \in X \setminus Y$ se tiene $A_y \cap D\neq ^*\emptyset$ y $A_x \cap D= ^*\emptyset$.

            Para cada $n \in \omega$ sea $U_n:=\midcup\{ B_m \in \mathcal{B} \tq m \in D \setminus n \}$, se afirma que $Y=\midcap\{U_n \tq n \in \omega\}$. Efectivamente; si $y \in X \setminus Y$ y $n \in \omega$ son cualesquiera, $A_y \cap D$ es infinito, y por ello, existe $m \in D \setminus n$ tal que $y \in B_m$. En consecuencia $y \in U_n$, y así $Y \subseteq \midcap\{U_n \tq n \in \omega\}$.
            
            De manera similar, si $x \in X \setminus Y$, $A_x \cap D$ es finito y existe $n \in \omega$ de modo que $A_x \cap D \subseteq n$. Por lo que para cada $m > n$ se tiene que $x \notin B_m$; luego entonces, $x \notin U_n$. Lo anterior muestra que $X \setminus Y \subseteq X \setminus \midcap\{U_n \tq n \in \omega\}$.

            Por lo tanto $Y=\midcap\{U_n \tq n \in \omega\}$ y es $G_\delta$.
        \end{proof}

        Nótese que la influencia de $\Ma$ en la previa caracterización radica únicamente en la necesidad, cuando $\aleph_1 \leq |X| < \mathfrak{c}$.

        De la proposición recien mostrada, el \autoref{teo-EquivPDM} y el \autoref{cor-PdmConsistente} surge el resultado que pone punto final a la Conjetura Débil de Moore (y prueba la consistencia de la negación de la Conjetura de Moore).

        \begin{corolario}\label{cor-PdmIndependiente}
            Bajo $\Ma$; para cada cardinal infinito $\kappa<\mathfrak{c}$ existe un espacio de Mrówka normal, de tamaño $\kappa$. Consecuentemente:
            \begin{enumerate}
                \item Bajo $\Ma+\lnot \HC$; existen tales espacios.
                \item $\lnot\Pdm$ (y por ello, $\lnot\Pm$) es consistente con $\zfc$.
                \item $\Pdm$ es independiente de $\zfc$.
            \end{enumerate}
        \end{corolario}

        Contrastable con \ref{pro-MaQsetChar} es el hecho de que aún no se ha dado una caracterización para la normalidad de los espacios de Mrówka. Resulta seducto conjeturar que cualquier espacio de Isbell-Mrówka de tamaño menor al continuo es normal. Sin embargo, el \autoref{cor-MrowkaLuzin} muestra que; bajo $\Ma+\lnot \HC$, existe un espacio de Mrówka, no normal y de tamaño menor al continuo.
        
        El comentario anterior deja como consecuencia la falsedad de que cualquier familia casi ajena sea \textit{esencialmente igual} a alguna de las definidas en \ref{def-FamRamas} (en el sentido lo comentado en la \autoref{pDif-esencial}); de lo contrario, cualquier espacio espacio de Isbell-Mrówka de tamaño menor al continuo sería es normal, cosa que es falsa (al menos desde $\zfc$ únicamente).

        %Esta breve discusión abre las puertas al estudio del enunciado ``para cada cardinal $\kappa$ entre $\aleph_1$ y $\mathfrak{c}$ existen espacios de Mrówka no normales de tamaño $\kappa$''; y para dar una aproximación inicial a tal estudio, se caracterizará (en $\zfc+\Ma+\lnot\HC$) la normalidad de los espacios de Mrówka.

        


        
    






\end{document}