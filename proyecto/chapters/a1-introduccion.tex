Corría el año de 1954 cuando Stanisław G. Mrówka (1933-2010); ``hijo'' doctoral de Kazimierz Kuratowski, introdujo en su artículo \textbf{On completely regular spaces} un ejemplo de espacio topológico de Tychonoff, pseudocompacto, pero no compacto; su construcción parte de una familia de subconjuntos de $\omega$ con características especiales. Este avance teórico es considerado un antecedente temprano de los hoy conocidos como \textit{espacios de Isbell-Mrówka}; espacios cuyo nombre hace honor tanto al propio Stanisław Mrówka, como a John R. Isbell (1930-2005); quienes, durante la década de los sesenta le dieron forma al concepto y mostraron por qué tales espacios son dignos de poseer gran interés dentro de la topología general.

Cada espacio de de Isbell-Mrówka es de Tychonoff, de Moore y cero-dimensional; además, la razón por la que el ejemplo dado por Mrówka en 1954 es considerado como el antecesor de estos espacios es porque cada uno de ellos determinado (biunívocamente) por una \textit{familia casi ajena}, un tipo especial de familia de subconjuntos de $\omega$. Tal correspondencia es la responsable de que se puedan ``codificar'' sus propiedades topológicas en términos de las propiedades que tienen las familias casi ajenas como conjuntos. La idea del presente documento es dar un primer acercamiento al estudio de estos espacios, para lo cual es necesario dar un panorama general sobre qué son las familias casi ajenas y cómo se comportan; explicar exactamente cuáles son estas ``codificaciones'' y exhibir algunos de los métodos para conseguirlas.

Este trabajo se divide en dos grandes secciones; el estudio de las familias casi ajenas (Capítulo 1), lo que se puede tomar como la parte ``conjuntista'' del escrito; y, su contraparte topológica, el estudio de los espacios de Isbell-Mrówka (Capítulos 2-4). En el Capítulo 1 se presentarán construcciones clásicas de familias casi ajenas y se expondrá la teoría básica sobre su combinatoria infinita asociada: Teorema de Simon, familias de Luzin y el Lema de Solovay. El Capítulo 2 tiene por objetivo presentar el Teorema de Kannan y Rajagopalan; una caracterización por propiedades topológicas de los espacios de Isbell-Mrówka, aquí, el entendimiento del comportamiento esencial de estos espacios es clave, así que este también será un aspecto central del capítulo. Los Capítulos 3 y 4 abordan problemas específicos que son testigo de la versatilidad de los objetos de estudio de esta tesis; en el Capítulo 3 se exhibirá la relación que hay entre la propiedad de Fréchet y la compactación unipuntual de los espacios de Isbell-Mrówka (el compacto de Franklin); y, en el Capítulo 4, se estudiarán en detalle los aspectos primordiales para poder ``codificar'' la propiedad topológica de normalidad, presentando principalmente los resultados de Silver y Tall.

Se dará por hecho que el lector de este documento tiene conocimientos elementales tanto de la teoría axiomática (usual) de conjuntos, $\zfc$; así como de topología general.

