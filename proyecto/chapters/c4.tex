\chapter{Normalidad en los espacios de Mrówka}

    \index[alph]{Conjetura!de Moore}\index[alph]{Conjetura!débil de Moore}\index[alph]{Moore!conjetura de}\index[alph]{Moore!conjetura débil de}\index[sym]{$\Pm$}\index[sym]{$\Pdm$}
    \emph{\small La hoy conocida como <<Conjetura de Moore>> (\,$\Pm$), establece que todo espacio de Moore normal es metrizable; se trata de un prblema lanzado a la comunidad matemática por Jones en 1933 que atiende a la cuestión ¿qué requiere un espacio de Moore para ser metrizable?. Este problema marcó un antes y un despues para la topología general, consolidándose como uno de los problemas (sino el que más) importantes en la topología y la teoría de conjuntos. $\Pm$ tiene, presumiblemente, una solución independiente a la axiomática $\zfc$ (\cite[p.~429-435]{nyikosMoore}).}
   
    \emph{\small En el año 1937 (véase \cite[Teo~5, p.~ 676]{jonesCM}), el propio Jones muestra la consistencia de la <<Conjetura Débil de Moore>> (\,$\Pdm$); esto es, cualquier espacio separable, normal y de Mooore, debe ser metrizable. Pero no sería sino hasta 1969 cuando Tall, en su tesis doctoral \cite{tallTesis}, logra establecer una equivalencia para $\Pdm$ en términos de la existencia ciertos espacios topológicos ($Q$-sets, \autoref{def-Qset}) no numerables; mismos para los cuales, Silver mostró consistente su existencia.}

    \emph{\small La meta de este capítulo será exponer las contribuciones de Jones, Silver y Tall; las cuales conjuntamente, permiten mostrar la independecia de $\Pdm$ de la axiomática usual de $\zfc$.}

    \section{Independencia de la Conjetura Débil de Moore}

    Por otra parte, todo espacio de Mrówka es de Moore y separable (\autoref{cor-MrwokaSiempre}); así que el enunciado $\Pdm$ (por consiguiente, $\Pm$) implica que ningún espacio $\Psi(\ms{A})$; con $\ms{A}$ una familia casi ajena más que numerable, puede ser normal. Lo que ataña a la presente sección; y claramente presenta una dificultad mayor, es mostrar el recíproco de la anterior implicación.

    Se comenzará exponiendo una condición necesaria que dicta ''dónde buscar'' espacios de Mrówka que sirvan de contrajemplo para $\Pdm$.
    
    \begin{proposicion}\label{pro-ObsNormalidad}
        Sea $\ms{A} \in \Ad(\omega)$, se cumple:
        \begin{enumerate}
            \item Si $|\ms{A}| \leq \aleph_0$, entonces $\Psi(\ms{A})$ es normal.
            \item Si $\ms{A}$ es infinita y $\Psi(\ms{A})$ es normal, $\ms{A}$ no es maximal y además $\aleph_1 \leq |\ms{A}|< \mathfrak{c}$.
        \end{enumerate}
    \end{proposicion}
    \begin{proof}
        (i) Cualquier espacio de Mrówka numerable es metrizable (por \ref{prop-tra-numerable}), particularmente normal (\textbf{Ree BSB}).

        (ii) Supóngase que $\ms{A}$ es infinita y que $\Psi(\ms{A})$ es normal. Si $\ms{A}$ fuera maximal, entonces por \ref{prop-tra-pseudoCaract} y \ref{prop-tra-compacidad}, se tiene que $\Psi(\ms{A})$ es pseudocompacto pero no numerablemente compacto. Lo cual (por el \textbf{RAKA}) imposibilita que $\Psi(\ms{A})$ sea normal. Por tanto, $\ms{A}$ no es maximal. Esto también implica que $\aleph_1 \leq |\ms{A}|$ (en virtud de la \autoref{prop-MADnoNum}).
        
        Finalmente, $|\ms{A}|=\mathfrak{c}$, debido a \ref{lem-primerosSubs}, $\ms{A}$ es un subespacio cerrado y discreto de $\Psi(\ms{A})$ de tamaño $\mathfrak{c}$. Así, de la separabilidad y normalidad de $\Psi(\ms{A})$ se desprende; por el Lema de Jones (léase \textbf{TAL}), que $2^\mathfrak{c} = 2^{|\ms{A}|} \leq 2^{\aleph_0} = \mathfrak{c}$, lo cual es imposible. Por lo tanto $|\ms{A}|< \mathfrak{c}$.
    \end{proof}

    \begin{corolario}\label{cor-HCnoMrowkasNormales}
        Bajo $\HC$; ningún espacio de Mrówka más que numerable, es normal.
    \end{corolario}

    Lo subsecuente caracteriza; en términos simples, la normalidad de un espacio de Mróka, a través de la combinatoria de su familia asociada.

    \begin{proposicion}\label{pro-ParticionadorCerrados}
        Sean $\ms{A}$ una familia casi ajena y $F,G \subseteq \Psi(\omega)$ cerrados ajenos. Son equivalentes:
        \begin{enumerate}
            \item $F$ y $G$ se separan por abiertos ajenos de $\Psi(\ms{A})$.
            \item $F \cap \ms{A}$ y $G \cap \ms{A}$ se separan por abiertos ajenos de $\Psi(\ms{A})$.
            \item La grieta $(F \cap \ms{A},G \cap \ms{A})$ está separada.
        \end{enumerate}
    \end{proposicion}

    \begin{proof}
        La implicación (i) $\to$ (ii) es inmediata.

        (ii) $\to$ (iii) Supóngase que $U,V \subseteq \Psi(\ms{A})$ son abiertos ajenos tales que $F \cap \ms{A} \subseteq U$ y $G \cap \ms{A} \subseteq U$.
        
        Si $a \in F \cap \ms{A}$ es cualquiera, entonces $a \in U$ y por definición de la topología en $\Psi(\ms{A})$ resulta que $a \subseteq^* U$. Ahora, si $b \in G \cap \ms{A}$ es cualquiera, entonces $b \subseteq^* V \subseteq \omega \setminus U$; de donde, $b \cap U =* \emptyset$. Por lo tanto, $U$ es particionador de $F \cap \ms{A}$ y $G \cap \ms{A}$.

        (iii) $\to$ (i) Supóngase que $D \subseteq \omega$ es particionador de $F \cap \ms{A}$ y $G \cap \ms{A}$. Nótese que $F \subseteq U:=F \cup D\setminus G$; y además, $U$ es abierto. Efectivamente, dado $a \in U \cap \ms{A} \subseteq F \cap \ms{A}$ se tiene que $a \subseteq^* D$ (por ser $D$ particionador de $F \cap \ms{A}$ y $G \cap \ms{A}$) y $a \subseteq^* \Psi(\ms{A}) \setminus G$ (por ser $G$ cerrado y ajeno a $F$), en consecuencia $a \subseteq^* D\setminus G \subseteq U$.

        Como $D$ es particionador de $F \cap \ms{A}$ y $G \cap \ms{A}$; $\omega \setminus D$ es particionador de $G\cap \ms{A}$ y $F \cap \ms{A}$, y resulta análogo que $G \subseteq V:=G \cup (\omega \setminus D)\setminus F$ y $V$ es abierto. Probando que $F$ y $G$ se separan por los abiertos ajenos $U$ y $V$.
    \end{proof}

    Debido al \autoref{lem-primerosSubs} y la \autoref{obs-GrietasSimple}, se deprende:

    \begin{corolario}\label{cor-tra-NormalParticionador}\index[trad]{Normalidad de $\Psi(\ms{A})$ (con grietas separables)}
        Para cada familia casi ajena $\ms{A}$ son equivalentes:
        \begin{enumerate}
            \item $\Psi(\ms{A})$ es normal.
            \item Para cada $\ms{B} \subseteq \ms{A}$, la grieta $(\ms{B},\ms{A} \setminus \ms{B})$ está separada.
        \end{enumerate}
    \end{corolario}

    La traducción de la \autoref{pro-ObsNormalidad} en términos combinatorios es la siguiente proposición (lo cual, por cierto, demuestra el \autoref{ej-interrelacion} de la \autoref{Sec-Luzin}):

    \begin{corolario}\label{col-tra-interrelacion}
        Sea $\ms{C}\in \Ad(\omega)$, entonces:
        \begin{enumerate}
            \item Si $|\ms{C}|\leq \aleph_0$, toda grieta contenida en $\ms{C}$ está separada.
            \item Si $\ms{C}$ es inifnita y, $|\ms{C}|=\mathfrak{c}$ o $\ms{C}\in \Mad(\omega)$; entonces $\ms{C}$ contiene una grieta que no está separada.
        \end{enumerate}
    \end{corolario}
    
    En términos topológicos, todo espacio $\Psi(\ms{A})$ no normal (con $\ms{A}$ infinita) contiene dos cerrados ajenos que no se pueden separar por abiertos ajenos, y en virtud del punto (i) del anterior corolario, alguno de ellos debe ser no numerable. Surge la siguiente cuestión:, ¿cúandoque ninún par de cerrados ajenos no numerables se pueden separar?.

    El análisis expuesto en \autoref{Sec-Luzin} establece que estos espacios son, exactamente, aquellos generados por una familia inseparable (en el sentido de lo compentado en la \pageref{def-FamInseparable}), particularmente:

    \begin{corolario}\label{cor-MrowkaLuzin}
        Si $\ms{A}$ es una familia de Luzin, ninún par de cerrados ajenos no numerables de $\Psi(\ms{A})$ se pueden separar por abiertos ajenos.

        Particularmente, hay un espacio de Mrówka de tamaño $\aleph_1$ no normal.
    \end{corolario}

    \subsection{Consistencia de \textsf{WMC}}
    \label{Sec-PDM}

    Siguiendo la técnica de Tall, para probar la independencia de la Conjetura Débil de Moore (de $\zfc$), se utilizarán a modo de intermediario los espacios metrizables conocidos como $Q$-sets.

    \begin{definicion}\label{def-Qset}\index[alph]{$Q$-set}
        Un $Q$-set es un espacio metrizable, separable y tal que todos sus subespacios son de tipo $G_\delta$ (equivalentemente; $F_\sigma$).
    \end{definicion}

    \begin{ejemplo}\label{ej-QsetFacil}
        Cualquier espacio $X$ a lo más numerable y metrizable es un $Q$-set. Efectivamente, nótese que $X$ es separable. Y además, si $A \subseteq X$ es cualquiera, entonces $A=\midcup\big\{\{a\} \tq a \in X\big\}$ es de tipo $F_\sigma$.
    \end{ejemplo}

    Se comenzará por observar que todo $Q$-set es; salvo homeomorfismos, un subespacio de $\mathbb{R}$ (o del conjunto de cantor, $2^\omega$). El siguiente lema, incluido en \cite[Teo.~1, p.~286]{kuratowskiTopology} por Kuratowski; se enunciará y demostrará con terminología moderna.
    \begin{lema}
        Sea $X$ un espacio metrizable por la métrica $d$. Si $|X|<\mathfrak{c}$, entonces $X$ es cero-dimensional.
    \end{lema}
    \begin{proof}
        Supóngase que $|X|<\mathfrak{c}$. Basta corroborar que cada $x \in X$ admite una base local de abiertos y cerrados. Sean $x \in X$ y $\varepsilon>0$.

        Supóngase ahora que para cada $\delta \in (0,\varepsilon)$, el conjunto $\fron(B(x,\delta))$ es no vacío, y fíjese ($\Ac$) un elemento $x_\delta \in \fron(B(x,\delta)) \subseteq X$. Como $|(0,\varepsilon)|=\mathfrak{c}$, la asignación $\delta \to x_\delta$ no puede ser inyectiva. Consecuentemente, existen distintos $\delta,\delta'\in (0,\varepsilon)$ de modo que $\fron(B(x,\delta)) \cap \fron(B(x,\delta')) \neq \emptyset$. Pero esto es imposible, dado que $\delta \neq \delta'$.
        
        Por lo tanto, para cada $\varepsilon>0$ se puede fijar ($\Ac$) cierto $\delta_\varepsilon \in (0,\varepsilon)$ tal que $\fron(B(x,\delta_\varepsilon))=\emptyset$; esto es, $B(x,\delta_\varepsilon)$ es abierto y cerrado a la vez. Claramente $\{B(x,\delta_\varepsilon) \tq \varepsilon>0\}$ es una base local para $x$ en $X$.
    \end{proof}

    \begin{proposicion}\label{prop-QsetEquivs}
        Para todo espacio $X$ son equivalentes:
        \begin{enumerate}
            \item $X$ es un $Q$-set.
            \item $X$ se encaja en $2^\omega$ y todos sus subespacios son de tipo $G_\delta$.
            \item $X$ se encaja en $\mathbb{R}$ y todos sus subespacios son de tipo $G_\delta$.
        \end{enumerate}
    \end{proposicion}
    \begin{proof}
        Dada la universalidad del conjunto de Cantor, $2^\omega \subseteq \mathbb{R}$, sobre la clase de espacios cero-dimensionales; y que todo subespacio de $\mathbb{R}$ es metrizable y separable, basta probar que todo $Q$-set es cero-dimensional.

        Supóngase que $X$ es un $Q$-set, como $X$ es metrizable y separable, entonces es $2\AN$. Sea $\mathcal{B}$ una base a lo más numerable $\mathcal{B}$ para $X$.

        Como $X$ es $Q$-set, para cada $A \subseteq X$ fíjese ($\Ac$) una colección a lo más numerable de abiertos $\mathcal{U}$ de modo que $A=\midcap \mathcal{U}$. Y como $\mathcal{B}$ es base; de nuevo haciendo uso de $\Ac$, para cada abierto $U$ fíjese $\mathcal{B}_U \subseteq \mathcal{B}$ de modo que $U=\midcup \mathcal{B}_U$.

        Lo anterior permite definir $\ms{P}(X) \to [\ms{P}(\mathcal{B})]^{\leq \omega}$ por medio de la correpondencia: $A \mapsto \{B_U \tq U \in \ms{A}\}$. Nótese que tal asignación es inyectiva, pues si $\{B_U \tq U \in \ms{A}\}=\{B_U \tq U \in \ms{B}\}$, entonces:
        \[ \mathcal{U}_A = \{\midcap \mathcal{B}_U \tq U \in \mathcal{U}_A\} = \{\midcap \mathcal{B}_U \tq U \in \mathcal{U}_B\} = \mathcal{U}_A \]
        y con ello $A=\midcap \mathcal{U}_A = \midcap \mathcal{U}_B = B$. De esta manera:
        \[ 2^{|X|} \leq \left| [\ms{P}(\mathcal{B})]^{\leq \omega} \right| \leq \left( 2^{|B|} \right)^{\aleph_0} \leq \left( 2^{\aleph_0} \right) ^{\aleph_0} = 2^{\aleph_0 \cdot \aleph_2} = 2^{\aleph_0} = \mathfrak{c} \]

        La última desigualdad implica que $|X| < \mathfrak{c}$. Siguiéndose del Lema previo, la cero-dimensionalidad de $X$.
    \end{proof}

    \begin{observacion}\label{obs-HCNoQset}
        Todo $Q$-set tiene tamaño menor que $\mathfrak{c}$. Consecuentemente, bajo $\HC$; no existen $Q$-sets más que numerables.
    \end{observacion}

    El paralelismo del resultado anterior con el \autoref{cor-HCnoMrowkasNormales} no es coincidencia. La meta ahora es mostrar que la existencia de $Q$-sets no numerables es equivalente a la existencia de espacios de Mrówka no numerables y normales; más aún, si estos espacios no existen, entonces $\Pdm$ se satisface.

    \begin{lema}\label{coro-Bing}
        Supóngase que $X$ es normal, de Moore, no metrizable y que $D \subseteq X$ es denso a lo más numerable; entonces existe $A \subseteq X \setminus D$ más que numerable, discreto y cerrado en $X$.
    \end{lema}

    \begin{proof}
        El Tereoma de Bing (\textbf{BINGEEE}) caracteriza la metrización de los espacios de Moore a través de la normalidad colectiva. Por ello, $X$ no es colectivamente normal y existe una familia discreta $\mathcal{A}$ de cerrados de $X$, cuyos elementos no se pueden separar por abiertos ajenos.
    
        Como $X$ es normal, para cada par de cerrados ajenos de $X$; digamos $F$ y $G$, elíjanse ($\Ac$) abiertos ajenos $W(F,G),S(F,G)$ de modo que $F\subseteq W(F,G)$ y $G \subseteq S(F,G)$. Es claro que $\mathcal{A}$ no puede ser finito.

        \begin{enumerate}[\hspace{1.5 cm}, listparindent=1.5em]
            \item \textit{Afirmación.} $\mathcal{A}$ es más que numerable.
            
            \item \textit{Demostración.} Supóngase que $\mathcal{A}$ está enumerado inyectivamente como $\{A_n \tq n \in \omega\}$. Por ser $\mathcal{A}$ familia discreta de cerrados, si $n \in \omega$, entonces $B_n:=\midcup\{A_m \tq m>n\}$ es cerrado.
            
            Por recursión, sean $U_0:=W(A_0,B_0)$ y $V_0:=S(A_0,B_0)$; y, para $n \in \omega$, $U_{n+1}:=W(A_{n+1},B_{n+1}) \cap V_n$ y $V_{n+1}= S(A_{n+1},B_{n+1}) \cap V_n$.

            Por construcción, $\{U_n \tq n\in \omega\}$ es una familia de abiertos, ajenos por pares tales que para cada $n \in \omega$ se tiene $A_n \subseteq U_n$. Así, los elementos de $\mathcal{A}$ se separan por abiertos ajenos; contradiciendo su elección. \hfill$\boxtimes$
        \end{enumerate}

        Dada la afirmación anterior, y fijando para cada $a \in \mathcal{A}$ un elemento $x_a \in \mathcal{A}$, se obtiene un conjunto mas que numerable $B:=\{x_a \tq a \in \mathcal{A}\}$; mismo que por ser $\ms{A}$ familia discreta y $X$ de Hausdorff, resulta ser cerrado y discreto.

        Por último nótese que cada subespacio de $B$ es discreto y cerrado en $X$; pues $B$ es discreto y cerrado en $X$. Particularmente, $A:=B \setminus D$ es discreto, discreto en $X$ y no numerable (pues $B$ es más que numerable y $D$ es numerable).
    \end{proof}

    El siguiente teorema aparece en la tesis doctoral de Franklin David Tall (ver \cite{tallTesis}), y es la pieza clave para atacar la Conjetura Débil de Moore.

    \begin{teorema}[Tall]\label{teo-EquivPDM}
        Si $\kappa$ es un cardinal infinito, son equivalentes:
        \begin{enumerate}
            \item Existe un espacio de Moore, normal, no metrizable de tamaño $\kappa$.
            \item Existe un espacio de Mrówka normal de tamaño $\kappa$.
            \item Existe un $Q$-set de tamaño $\kappa$.
        \end{enumerate}
    \end{teorema}
    \begin{proof}
        (i) $\to$ (ii) Supóngase que $X$ es un espacio, normal, de Moore y no metrizable de tamaño $\kappa$ y sea fíjese $D\subseteq X$ denso numerable de $X$. Por \ref{coro-Bing}, $D$ es infinito y existe un subespacio $A \subseteq X \setminus D$ más que numerable; discreto y cerrado de $X$. Como $X$ es de Hausdorff y primero numerable, considérese $ \ms{A}_{D,A}=\{A_x \in [D]^\omega \tq x \in A\} $; la familia de sucesiones en $D$ convergentes a $A$ (definida en \ref{def-FamSucesiones}), donde cada $A_x$ converge a $x$. Por la \autoref{prop-famSucesiones}, $|\ms{A}_{D,A}|=\kappa$ y así mismo, $\Psi_D(\ms{A}_{D,A})=\kappa$.
        
        Sea $\ms{B}:=\{A_x \tq x \in F\} \subseteq \ms{A}_{D,A}$ cualquiera. Como $A$ es discreto y cerrado en $X$, cualquiera de sus subespacios es cerrado en $X$; en consecuencia y por normalidad de $X$, existen abiertos $U,V \subseteq X$ ajenos, de modo que $F \subseteq U$ y $A \setminus F \subseteq V$. Si $x \in F$, entonces $A_x \to x$ y por ello $A_x \subseteq^* U$, similarmente, si $y \in A \setminus F$, entonces $A_y \subseteq^* V \subseteq X \setminus U$; de donde $A_y \cap U^* = \emptyset$. 

        Por tanto $(\ms{B}, \ms{A}_{D,A} \setminus \ms{B})$ está separada, obteniéndose de \ref{cor-tra-NormalParticionador} la normalidad de $\Psi_D(\ms{A}_{D,A})$.
        
        (ii) $\to$ (iii) Si $\kappa = \omega$, la implicación resulta vacua; pues todo subespacio numerable de $2^\omega$ es un $Q$-set (\autoref{ej-QsetFacil}). Supóngase pues, que $\Psi(\ms{A})$ es un espacio normal de tamaño $\kappa>\omega$; siendo necesario que $|\ms{A}|=\kappa$.
        
        Para cada $C \subseteq \omega$ denótese por $\varphi_C \in 2^ \omega$ a la función característica de $C$ y sea $X:=\{\varphi_A \in 2^\omega \tq A \in \ms{A} \}$. Obsérvese que $X$ es un espacio metrizable, separable (por ser subespacio del metrizable, separable, $2^\omega$) de tamaño $\kappa$. 
        
        Sea $Y = \{ \varphi_A \in X \tq A \in \ms{B} \} \subseteq X$ cualquiera. Dado el \autoref{cor-tra-NormalParticionador}, la normalidad de $\Psi(\ms{A})$ implica la existencia de un particionador de $\ms{A}\setminus \ms{B}$ y $\ms{B}$; de este modo:
        \begin{align*}
            Y   & = \{ \varphi_A \in X \tq A \in \ms{B} \} \\
                & = \{ \varphi_A \in X \tq A \cap D \neq^* \emptyset \} \\
                & = \{ \varphi_A \in X \tq \forall n \in \omega \: ( A \cap D \not\subseteq n) \} \\
                & = \bigcap_{n \in \omega} \{\varphi_A \in X \tq A \cap D \not\subseteq n\}
        \end{align*}
        
        Ahora, si $n \in \omega$ y $\varphi \in U_n:=\{\varphi_A \in X \tq A \cap D \not\subseteq n\}$ es cualquiera, existe cierto $k \in (A \cap D) \setminus n$. Por ello, si $x=\varphi_B \in X$ es tal que $x(k)=1$, entonces $k \in (B \cap D) \setminus n$ y $B \cap D \not \subseteq n$; es decir $f \in U_n$. Mostrando así que $\{ x \in X \tq x\upharpoonright \{k\} = \varphi \upharpoonright \{k\} \} \subseteq U_n$. Por lo tanto, cada $U_n$ es abierto en $X$. De esta manera, cualquier $Y \subseteq X$ es $G_\delta$ en $X$ y $X$ es un $Q$-set.

        (iii) $\to$ (i) Supóngase que $X$ es un $Q$-set de tamaño $\kappa$. En virtud del \autoref{prop-QsetEquivs}, supóngase sin pérdida de generalidad que $X \subseteq 2^\omega$. Para cada $E \subseteq X$ considérese $ \ms{A}_E:=\{ A_x \in [N]^\omega \tq x \in E \} $, la familia de las ramas de $E$ en $N$ (definida en \ref{def-FamRamas}); donde $N:=2^{<\omega}$ y cada $A_x$ es el conjunto $\{x \upharpoonright n \tq n \in \omega\} \subseteq N$. Puesto que $|X|=\kappa$, del \autoref{def-FamRamas} se sigue que $|\ms{A}_X|=\kappa$, y con ello $|\Psi_N(\ms{A}_X)|=\kappa$.
        
        Sea $Y \subseteq X$ cualquiera. Como $X$ es un $Q$-set, $Y=\midcup\{F_n \tq n \in \omega\}$ y $X \setminus Y=\midcup\{G_n \tq n\in \omega\}$; donde cada conjunto $F_n$ y $G_n$ es cerrado en $X\subseteq 2^\omega$. Para cada $n \in \omega$ defínanse los conjuntos:
        \begin{align*}
            D_n := \left( \midcup \ms{A}_{F_n} \right) \setminus \bigcup_{m<n} \left( \midcup \ms{A}_{G_m} \right) \\
            L_n := \left( \midcup \ms{A}_{G_n} \right) \setminus \bigcup_{m\leq n} \left( \midcup \ms{A}_{F_m} \right) 
        \end{align*}
        y sea $D:=\midcup\{D_n \tq n \in \omega\}$. Nótese que por construccion, si $m,n \in \omega$, se tiene $D_n \cap L_m = \emptyset$; consecuentemente, cada $L_n$ es ajeno con $D$.

        Sea $y \in A_Y$; entonces existe $n \in \omega$ de modo que $y \in F_n$. Por otra parte, cada $G_m\subseteq X\setminus Y$ (con $m<n$) es cerrado en $X$, por lo que existe $s \in \omega$ de modo que:
        $$ \{ x \in X \tq x \upharpoonright s = y \upharpoonright s \} \subseteq X \setminus \bigcup_{m<n} G_m $$
        
        Por ello, si $v \in A_y \setminus D_n\subseteq F_n$, existen $x \in F_n$ y $k \in \omega$ de modo que $v = x \upharpoonright k$. Así que $x \in \midcup \ms{A}_{F_n}$; y como $v \notin D_n$, existen $m<n$ y $g \in G_m$ de manera que $v = y \upharpoonright k = g \upharpoonright k$. Y a razón de ello, no puede ocurrir $s \subseteq k$. Por lo tanto $k<s$ y $A_y \setminus D_n \subseteq 2^{<s} =^* \emptyset$; esto es, $A_y \subseteq^* D_n \subseteq D$.

        Similarmente, para cada $y \in X \setminus Y$ existe un $n \in \omega$ tal que $A_y \subseteq^* L_n \subseteq N \setminus D$; de donde, $A_y \cap D ^= \emptyset$. Así que $D$ es separador de $\ms{B}$ y $\ms{A} \setminus \ms{B}$; probando por el \autoref{cor-tra-NormalParticionador} la normalidad de $\Psi_N(\ms{A}_X)$.
    \end{proof}
    
    Es inmediato al \autoref{teo-EquivPDM} (y a \ref{cor-HCnoMrowkasNormales}, o bien, \ref{obs-HCNoQset}) la siguiente consecuencia:

    \begin{corolario}\label{cor-PdmConsistente}
        Bajo $\HC$; se cumple $\Pdm$, y:
        \begin{enumerate}
            \item Ningún espacio de Mrówka no numerable es normal.
            \item Ningún $Q$-set es más que numerable.
        \end{enumerate}
        Consecuentemente $\Pdm$ es consistente con $\zfc$.
    \end{corolario}
    
    \subsection{Consistencia de \texorpdfstring{$\lnot$}\textsf{WMC}}
    
    Para la segunda parte de la prueba de independencia de $\Pdm$ se hará uso; como es previsible desde anteriores capítulos, de la negacion de $\HC$ con el Axioma de Martin. Se comenzará observando cómo se pueden caracterizar a todos los $Q$-sets haciendo uso del Lema de Solovay y $\Ma$.

    \begin{lema}
        Sea $X$ un espacio metrizable y separable. Entonces existe una base $\mathcal{B}=\{B_n \tq n \in \omega\}$ para $X$ de modo que $\{A_x \tq x \in X\}$ es familia casi ajena; donde, cada $A_x$ es $\{n \in \omega \tq x \in B_n\}$.
    \end{lema}
    \begin{proof}
        Por el teorema (\textbf{Arhangel’skii}), $X$ admite una base regular (véase \textbf{BsRg}) $\mathcal{C}$. Como $X$ es $2\AN$ (a consecuencia de ser metrizable y separable \textbf{VeR}), existe una base $\mathcal{B}=\{B_n \tq n \in \omega\} \subseteq \mathcal{C}$, claramente $\mathcal{B}$ sigue siendo regular.
        
        Dados $x,y \in X$ son distintos, sean $U,V$ abiertos ajenos que separan a $x$ y $y$. Por regulardidad de la base $\mathcal{B}$, existe $W \subseteq U$ abierto con $x\in W$ y
        $\mathcal{B}_W:=\{n \in \omega \tq B_n \cap W \neq \emptyset \land B_n \setminus W \neq \emptyset\} =^* \emptyset$. Por consiguiente, el conjunto $A_x \cap A_y \subseteq \mathcal{B}_W$ es finito.
    \end{proof}

    \begin{proposicion}[Tall, Silver]\label{pro-MaQsetChar}
        Sea $X$ espacio topológico. Bajo $\Ma$; $X$ es un $Q$-set si y sólo si es homeomorfo a un subespacio $X \in [\mathbb{R}]^{<\mathfrak{c}}$.
    \end{proposicion}
        \begin{proof}
            Supóngase $\Ma$. La suficiencia viene dada por \ref{teo-EquivPDM} y \ref{prop-QsetEquivs}. Para la necesidad supóngase que $X \in [\mathbb{R}]^{<\mathfrak{c}}$, si $X$ es a lo más numerable, de \ref{teo-EquivPDM} y \ref{pro-ObsNormalidad} se sigue que $X$ es un $Q$-set.
            
            Como $X$ es metrizable y sepable, sean $\mathcal{B}$, $B_x$ (para cada $x \in X$) y $\ms{A}$ como en el Lema previo. Tómense $Y \subseteq X$ cualquiera y $B:=\{B_y \in \ms{A} \tq y \in Y\}$.
            
            Como $|\ms{A} \setminus \ms{B}|,|\ms{B}|<\mathfrak{c}$ y se cumple $\Ma$, del \autoref{lem-Solovay} se desprende la existencia de cierto $D \subseteq \omega$ de modo que para cada $y \in Y$ y $x \in X \setminus Y$ se tiene $A_y \cap D\neq ^*\emptyset$ y $A_x \cap D= ^*\emptyset$.

            Para cada $n \in \omega$ sea $U_n:=\midcup\{ B_m \in \mathcal{B} \tq m \in D \setminus n \}$, se afirma que $Y=\midcap\{U_n \tq n \in \omega\}$. Efectivamente; si $y \in X \setminus Y$ y $n \in \omega$ son cualesquiera, $A_y \cap D$ es infinito, y por ello, existe $m \in D \setminus n$ tal que $y \in B_m$. En consecuencia $y \in U_n$, y así $Y \subseteq \midcap\{U_n \tq n \in \omega\}$.
            
            De manera similar, si $x \in X \setminus Y$, $A_x \cap D$ es finito y existe $n \in \omega$ de modo que $A_x \cap D \subseteq n$. Por lo que para cada $m > n$ se tiene que $x \notin B_m$; luego entonces, $x \notin U_n$. Lo anterior muestra que $X \setminus Y \subseteq X \setminus \midcap\{U_n \tq n \in \omega\}$.

            Por lo tanto $Y=\midcap\{U_n \tq n \in \omega\}$ y es $G_\delta$.
        \end{proof}

        Nótese que la influencia de $\Ma$ en la previa caracterización radica únicamente en la necesidad, cuando $\aleph_1 \leq |X| < \mathfrak{c}$.

        De la proposición recien mostrada, el \autoref{teo-EquivPDM} y el \autoref{cor-PdmConsistente} surge el resultado que pone punto final a la Conjetura Débil de Moore (y prueba la consistencia de la negación de la Conjetura de Moore).

        \begin{corolario}\label{cor-PdmIndependiente}
            Bajo $\Ma$; para cada cardinal infinito $\kappa<\mathfrak{c}$ existe un espacio de Mrówka normal, de tamaño $\kappa$. Consecuentemente:
            \begin{enumerate}
                \item Bajo $\Ma+\lnot \HC$; existen tales espacios.
                \item $\lnot\Pdm$ (y por ello, $\lnot\Pm$) es consistente con $\zfc$.
                \item $\Pdm$ es independiente de $\zfc$.
            \end{enumerate}
        \end{corolario}

        Contrastable con \ref{pro-MaQsetChar} es el hecho de que aún no se ha dado una caracterización para la normalidad de los espacios de Mrówka. Resulta seducto conjeturar que cualquier espacio de Isbell-Mrówka de tamaño menor al continuo es normal. Sin embargo, el \autoref{cor-MrowkaLuzin} muestra que; bajo $\Ma+\lnot \HC$, existe un espacio de Mrówka, no normal y de tamaño menor al continuo.
        
        El comentario anterior deja como consecuencia la falsedad de que cualquier familia casi ajena sea \textit{esencialmente igual} a alguna de las definidas en \ref{def-FamRamas} (en el sentido lo comentado en la \hyperref[Dif-esencial]{Página~\ref{Dif-esencial}}); de lo contrario, cualquier espacio espacio de Isbell-Mrówka de tamaño menor al continuo sería es normal, cosa que es falsa (al menos desde $\zfc$ únicamente).

        %Esta breve discusión abre las puertas al estudio del enunciado ``para cada cardinal $\kappa$ entre $\aleph_1$ y $\mathfrak{c}$ existen espacios de Mrówka no normales de tamaño $\kappa$''; y para dar una aproximación inicial a tal estudio, se caracterizará (en $\zfc+\Ma+\lnot\HC$) la normalidad de los espacios de Mrówka.

        


        
    





