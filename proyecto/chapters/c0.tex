\setcounter{chapter}{-1}
\chapter{Preeliminares}

\emph{\small En este capítulo se convendrán tanto las notaciones, resultados clásicos y conceptos que se utilizarán a lo largo del texto.}

\section{Conjuntos}

Dado un conjunto $A$, se denotará por $\ms{P}(A)$ a su conjunto potencia; además, si $B$ es un conjunto, $A^B$ denotará el conjunto de todas las posibles funciones $f:B \to A$.

Se utilizará la notación estándar para ordinales, cardinales y naturales; esto es, usualmente se reservarán las letras $\alpha,\beta,\gamma,\dots$, $\kappa,\lambda,\mu,\dots$ y $n,m,k,\dots$ para los números ordinales, cardinales y naturales, respectivamente. Se seguirá la construcción de Von Neumann, así que siempre que $\alpha$ y $\beta$ sean ordinales, $\alpha<\beta$ significará que $\alpha \in \beta$. El primer ordinal infinito (exactamente el conjunto de números naturales) será denotado por la letra griega $\omega$; y, se utilizarán las notaciones $\aleph_-$ y $\omega_-$ cuando se requiera.

Con frecuencia, en diversas pruebas de este documento se involucrarán tanto inducción como recursión; las versiones de los teormas de recursión que utilizaremos son: primer y segundo Teorema de recursión en $\omega$, segundo Teorema de recursión transfinita y su versión restringida a $\omega_1$. El lector puede consultar el enunciado completo, y la teoría subyacente, de estos Teoremas en: \textbf{JECHHH}.

En ocasiones, se realizarán cálculos de aritmética cardinal; en esto, se utilizarán con frecuencia los teoremas de suma y producto ordinal. El lector interesado puede indagar más al respecto en \textbf{AmorIntermedio}.

Dados un conjunto $A$ y un cardinal $\kappa$ arbitrarios; se utilizarán la notaciones:
\[[A]^\kappa,[A]^{\leq \kappa},[A]^{< \kappa},[A]^{\geq \kappa},[A]^{> \kappa}\]
para las colecciones de subconjuntos de $A$ de tamaño: igual a $\kappa$; menor o igual a $\kappa$; menor que $\kappa$; mayor o igual a $\kappa$ y mayor que $\kappa$; respectivamente. Además, se denotará por $A^{<\kappa}$ al conjunto $\midcup \{A^{<\alpha} \tq \alpha < \beta\}$.

Los órdenes parciales reflexivos serán denotados por $\leq$; esto claro, a menos que se indique lo contrario, o bien, que el contexto dicte que ``$\leq$'' denota una desigualdad entre ordinales. Los órdenes parciales antirreflexivos se denotarán por $<$ (realizando las mismas consideraciones que con su símbolo análogo); y, en caso alguno de los dos no sea definido, siempre se puede pensar que $ \leq = < \cup \vartriangle$, o bien, $< = \leq \setminus \vartriangle$. Si $(P,\leq)$ es un conjunto ordenado y $A \subseteq P$ es cualquiera, se denotan (en caso de existir) por $\min(A), \max(A), \inf(A)$ y $\sup(A)$ a su elemento mínimo, máximo, ínfimo y supremo, respectivamente; además, se conviene que un elemento $p \in P$ es minimal (maximal, respectivamente) de $A$ si $p \in A$ y no existe $a \in A$ de modo que $a<p$ ($p<a$, respectivamente).

