%Formato e idioma
    \documentclass[letterpaper,DIV=12,12pt]{scrbook}
    \usepackage[spanish,mexico,shorthands=off,es-lcroman]{babel}
%Utilidades
    \usepackage{array}
    \usepackage[x11names]{xcolor}
    \usepackage{lipsum}
%Matemáticos
    \usepackage{amsmath}
    \usepackage{amsthm}
    \usepackage{amssymb} % eliminar si se usa unicode-math
    \usepackage{mathrsfs} % agregar después de fuente unicode-math, si se usa
%Etiquetas de las enumeraciones
    \usepackage[shortlabels]{enumitem}
    \setenumerate[1]{label=\MakeLowercase{\roman*}), ref=\roman*}
    \setenumerate[2]{label=\MakeLowercase{\alph*}), ref=\alph*}
%Cajas de Teoremas
    \usepackage{thmtools}
    \usepackage[framemethod=TikZ]{mdframed}
    \newtheorem{definicion}{Definición}[section]
    \newtheorem{proposicion}[definicion]{Proposición}
    \newtheorem{lema}[definicion]{Lema}
    \newtheorem{corolario}[definicion]{Corolario}
    \newtheorem{observacion}[definicion]{Observación}
    \newtheorem{ejemplo}[definicion]{Ejemplo}
    \newtheorem{consideracion}[definicion]{Consideración}
    \newtheorem{teorema}[definicion]{Teorema}
%Comandos Creados
	%GENERALES
	\newcommand{\tq}{\text{ $|$ }}
	\newcommand{\midcup}{\mbox{$\bigcup$}\,}
	\newcommand{\midcap}{\mbox{$\bigcap$}\,}
	\newcommand{\ms}[1]{\mathscr{#1}}
%RENOVACIÓN DE COMANDOS
	\renewcommand{\emptyset}{\varnothing}
	\renewcommand{\tau}{\ms{T}}
	%Nuevo "Setminus"
	\newcommand{\mysetminusD}{\hbox{\tikz{\draw[line width=0.6pt,line cap=round] (3pt,0) -- (0,6pt);}}}
	\newcommand{\mysetminusT}{\mysetminusD}
	\newcommand{\mysetminusS}{\hbox{\tikz{\draw[line width=0.45pt,line cap=round] (2pt,0) -- (0,4pt);}}}
	\newcommand{\mysetminusSS}{\hbox{\tikz{\draw[line width=0.4pt,line cap=round] (1.5pt,0) -- (0,3pt);}}}
	\newcommand{\mysetminus}{\mathbin{\mathchoice{\mysetminusD}{\mysetminusT}{\mysetminusS}{\mysetminusSS}}}
	\renewcommand{\setminus}{\mysetminus}
%OPERADORES
	%\makeatletter
	%	\renewcommand{\operator@font}{\opfont}
	%\makeatother

	\DeclareMathOperator{\inte}{int}
	\DeclareMathOperator{\ext}{ext}
	\DeclareMathOperator{\cla}{cl}
	\DeclareMathOperator{\der}{der}
	\DeclareMathOperator{\fron}{fr}
	\DeclareMathOperator{\scl}{sqcl}
	\DeclareMathOperator{\cf}{cf}
	\DeclareMathOperator{\Id}{Id}
	\DeclareMathOperator{\ima}{ima}
	\DeclareMathOperator{\dom}{dom}
	\DeclareMathOperator{\St}{St}
	\DeclareMathOperator{\Osq}{O_{sq}}
%TEXTOS
	\newcommand{\T}{\text{\textsf{\small T}}}
	\newcommand{\AN}{\text{\textsf{\small AN}}}
	\newcommand{\OR}{\text{\textsf{\small ON}}}
	\newcommand{\CAR}{\text{\textsf{\small CAR}}}
	\newcommand{\zfc}{\text{\textsf{\small ZFC}}}
	\newcommand{\zf}{\text{\textsf{\small ZF}}}
	\newcommand{\HC}{\text{\textsf{\small HC}}}
	\newcommand{\Ma}{\text{\textsf{\small MA}}}
	\newcommand{\Ac}{\text{\textsf{\small AC}}}
	\newcommand{\Pm}{\text{\textsf{\small MC}}}
	\newcommand{\Pdm}{\text{\textsf{\small WMC}}}
	\newcommand{\Ad}{\operatorname{\text{\textsf{\small AD}}}\,}
	\newcommand{\Mad}{\operatorname{\text{\textsf{\small MAD}}}\,}
%COCIENTE
	%\newcommand{\quot}[2]{{\raisebox{.2em}{$#1$}\left/\raisebox{-.2em}{$#2$}\right.}}

%Exclusivo del minimal
    \renewcommand{\ref}[1]{??}
    \renewcommand{\pageref}[1]{??}
    \newcommand{\autoref}[1]{??}
    \renewcommand{\cite}[2][]{??}
    \renewcommand{\index}[2][]{}
    \newcommand{\hyperref}[2][]{#2}
    \providecommand{\texorpdfstring}[2]{#1}

\begin{document}
    \chapter*{Pseudo estructura}
        \begin{enumerate}
            \item Potencia, exponencial y $A^{< algo}$
            \item $\omega$, naturales y ordinales
            \item Inducción y recursión
            \item Cardinalidad y cardinales
            \item Cofinalidad y regularidad
            \item Notaciones $[X]^{algo}$
            \item Familias (?)
            \item Órdenes parciales
            \begin{enumerate}
                \item elementos especiales
                \item subconjuntos especiales
                \item filtros, ideales
            \end{enumerate}
            \item Casi ajenidad
            \item Árboles
            \item 
        \end{enumerate}
        ----------------
        \begin{enumerate}
            \item Espacios topológicos que son y convenios generales. Notación de operadores, homeomorfismos, encajes, producto y coproducto. Que es propiedad topológica, prod, fac, hered, debil hered.
            \item Fréchet, sqcl, sucesiones cjto, sec compacto, secuencial, orden secuancial
            \item Axiomas de separación, caracter, peso ,numerabilidad.
            \item Tipos de compacidades. Resultados de compacidad y pseudocompacidad.
            \item Compactaciones de hausdorff
            \item Teorema de Categoría de Baire
            \item Metrizable, caracteriszaciones ded compacidad en metrizables. Separabilidad en metrizables
            \item Desarrollos, refinamientos, de Moore y teoremas de metrización. Bing, Arhangel'skii
        \end{enumerate}
        ----------------
        \begin{enumerate}
            \item Consistencia relativa
            \item Lógica, lenguajes, teorías, modelos
            \item $\vDash$ y $\vdash$, correctud y contraejemplos
            \item Modelos de conjntos
            \item Clases y notación
            \item Consistencia relativa, técnicas en conjuntos:
            \begin{enumerate}
                \item Genérico, densos
                \item MA
                \item Forcing y uso de Forcing
            \end{enumerate}
        \end{enumerate}

        \chapter*{Cosas puntuales}
        \begin{enumerate}
            \item Notación para funciones cardinales
            \item G deltas y F sigmas
            \item 
        \end{enumerate}
\end{document}