%%%%%%%%%%%%%%%%%%%%%%%%
%%% FORMATO E IDIOMA %%%
%%%%%%%%%%%%%%%%%%%%%%%%
    \documentclass[letterpaper,DIV=12,12pt]{scrbook}
    \usepackage[spanish,mexico,shorthands=off,es-lcroman]{babel}
    \usepackage{scrlayer-scrpage}
    \input{../resources/koma.tex}
%%%%%%%%%%%%%%%%
%%% PAQUETES %%%
%%%%%%%%%%%%%%%%
    \usepackage{array}
    \usepackage[x11names]{xcolor}
    \usepackage{lipsum}
    \usepackage[shortlabels]{enumitem}
        \setenumerate[1]{label=\MakeLowercase{\roman*}), ref=\roman*}
        \setenumerate[2]{label=\MakeLowercase{\alph*}), ref=\alph*}
%%%%%%%%%%%%%%%%%%%%%%%%%%%%%
%%% FUENTES Y MATEMÁTICAS %%%
%%%%%%%%%%%%%%%%%%%%%%%%%%%%%
    \usepackage{fontspec}
    \setmainfont{GFSArtemisia.otf}[ 
    Path = ../fonts/GFSArtemisia/,
    BoldFont = GFSArtemisiaBold.otf,
    ItalicFont = GFSArtemisiaIt.otf,
    BoldItalicFont = GFSArtemisiaBoldIt.otf
    ]

    % --- Matemáticas ---
    \usepackage{amsmath}   % fórmulas (solo macros, no mete fuentes)
    \usepackage{amsthm}    % teoremas

    \usepackage{unicode-math} % control de tipografía matemática
    \setmathfont{STIXTwoMath-Regular.ttf}[
    Path = ../fonts/StixTwo/,
    ]
    \usepackage{mathrsfs}
    
%%%%%%%%%%%%%%%%%%%
%%% REFERENCIAS %%%
%%%%%%%%%%%%%%%%%%%
    \usepackage[colorlinks=true, linkcolor=rosa, citecolor=azulC, urlcolor=dorado]{hyperref}
    \usepackage[backend=biber, style=numeric, sortcites, url=true]{biblatex}
    \usepackage{csquotes,url}
        \addbibresource{../chapters/referencias.bib}
%%%%%%%%%%%%%%%
%%% ÍNDICES %%%
%%%%%%%%%%%%%%%
    \usepackage{imakeidx}
    \usepackage{etoolbox}
        \makeindex[columns=1, intoc, title=PRUEBA]
        \makeindex[name=trad, columns=1, intoc, title=Caracterizaciones]
        \makeindex[name=sym, columns=3, intoc, title=Índice Simbólico]
        \makeindex[name=alph, columns=2, intoc, title=Índice Alfabético]
        \indexsetup{firstpagestyle=beginstyle}
%%%%%%%%%%%%%
%%% CAJAS %%%
%%%%%%%%%%%%%
    \usepackage{thmtools}
    \usepackage[framemethod=TikZ]{mdframed}
    \allowdisplaybreaks
    \input{../resources/cajas-print.tex}
%%%%%%%%%%%%%%%%
%%% COMANDOS %%%
%%%%%%%%%%%%%%%%
    %GENERALES
	\newcommand{\tq}{\text{ $|$ }}
	\newcommand{\midcup}{\mbox{$\bigcup$}\,}
	\newcommand{\midcap}{\mbox{$\bigcap$}\,}
	\newcommand{\ms}[1]{\mathscr{#1}}
%RENOVACIÓN DE COMANDOS
	\renewcommand{\emptyset}{\varnothing}
	\renewcommand{\tau}{\ms{T}}
	%Nuevo "Setminus"
	\newcommand{\mysetminusD}{\hbox{\tikz{\draw[line width=0.6pt,line cap=round] (3pt,0) -- (0,6pt);}}}
	\newcommand{\mysetminusT}{\mysetminusD}
	\newcommand{\mysetminusS}{\hbox{\tikz{\draw[line width=0.45pt,line cap=round] (2pt,0) -- (0,4pt);}}}
	\newcommand{\mysetminusSS}{\hbox{\tikz{\draw[line width=0.4pt,line cap=round] (1.5pt,0) -- (0,3pt);}}}
	\newcommand{\mysetminus}{\mathbin{\mathchoice{\mysetminusD}{\mysetminusT}{\mysetminusS}{\mysetminusSS}}}
	\renewcommand{\setminus}{\mysetminus}
%OPERADORES
	%\makeatletter
	%	\renewcommand{\operator@font}{\opfont}
	%\makeatother

	\DeclareMathOperator{\inte}{int}
	\DeclareMathOperator{\ext}{ext}
	\DeclareMathOperator{\cla}{cl}
	\DeclareMathOperator{\der}{der}
	\DeclareMathOperator{\fron}{fr}
	\DeclareMathOperator{\scl}{sqcl}
	\DeclareMathOperator{\cf}{cf}
	\DeclareMathOperator{\Id}{Id}
	\DeclareMathOperator{\ima}{ima}
	\DeclareMathOperator{\dom}{dom}
	\DeclareMathOperator{\St}{St}
	\DeclareMathOperator{\Osq}{O_{sq}}
%TEXTOS
	\newcommand{\T}{\text{\textsf{\small T}}}
	\newcommand{\AN}{\text{\textsf{\small AN}}}
	\newcommand{\OR}{\text{\textsf{\small ON}}}
	\newcommand{\CAR}{\text{\textsf{\small CAR}}}
	\newcommand{\zfc}{\text{\textsf{\small ZFC}}}
	\newcommand{\zf}{\text{\textsf{\small ZF}}}
	\newcommand{\HC}{\text{\textsf{\small HC}}}
	\newcommand{\Ma}{\text{\textsf{\small MA}}}
	\newcommand{\Ac}{\text{\textsf{\small AC}}}
	\newcommand{\Pm}{\text{\textsf{\small MC}}}
	\newcommand{\Pdm}{\text{\textsf{\small WMC}}}
	\newcommand{\Ad}{\operatorname{\text{\textsf{\small AD}}}\,}
	\newcommand{\Mad}{\operatorname{\text{\textsf{\small MAD}}}\,}
%COCIENTE
	%\newcommand{\quot}[2]{{\raisebox{.2em}{$#1$}\left/\raisebox{-.2em}{$#2$}\right.}}
%%%%%%%%%%%%%%%%%
%%% DOCUMENTO %%%
%%%%%%%%%%%%%%%%%
\begin{document}

    \chapter{Prueba de Place Holder}
    \lipsum[1-2]

Un concepto importante en matemáticas es la definición de límite.  
Dado $f(x)$ definida en un entorno de $a$, se dice que
\[
  \lim_{x \to a} f(x) = L
\]
si para todo $\varepsilon > 0$ existe $\delta > 0$ tal que 
$0 < |x-a| < \delta \implies |f(x)-L| < \varepsilon$.

\lipsum[3]

\section{Álgebra y ecuaciones}
Consideremos la ecuación cuadrática general
\begin{equation}
  ax^2 + bx + c = 0, \quad a \neq 0.
\end{equation}

Su discriminante es $\Delta = b^2 - 4ac$, y entonces:
\[
  x = \frac{-b \pm \sqrt{\Delta}}{2a}.
\]

\lipsum[4-5]

\section{Series y sucesiones}
Las series infinitas juegan un rol importante. Por ejemplo,
\begin{equation}
  \sum_{n=1}^\infty \frac{1}{n^2} = \frac{\pi^2}{6}.
\end{equation}

De hecho, una sucesión $(a_n)$ converge a $L$ si para todo $\varepsilon>0$ existe $N$ tal que $n>N \implies |a_n - L| < \varepsilon$.

\lipsum[6-7]

\section{Cálculo diferencial}
Si $f(x)=x^3$, entonces su derivada es $f'(x)=3x^2$.
Más generalmente, si $f(x)=x^n$, se tiene
\begin{equation}
  \frac{d}{dx} x^n = n x^{n-1}.
\end{equation}

\lipsum[8]

Otra fórmula útil es la de integración por partes:
\[
  \int u \, dv = uv - \int v \, du.
\]

\lipsum[9-10]

\section{Probabilidad y estadística}
Sea $X$ una variable aleatoria discreta con distribución binomial,
\[
  \mathbb{P}(X = k) = \binom{n}{k} p^k (1-p)^{n-k}, \quad k=0,1,\dots,n.
\]

La esperanza es
\begin{equation}
  \mathbb{E}[X] = np,
\end{equation}
y la varianza es $\mathrm{Var}(X) = np(1-p)$.

\lipsum[11-12]

\section{Conclusión}
\lipsum[13-14]

Ahora consideremos las fuentes:
\[ \mathscr{A}, \mathscr{D}, \mathscr{K}, \omega, \mathscr{A}^\omega \]
\[ \mathcal{A}, \mathcal{D}, \mathcal{K}, \omega, \mathcal{A}^\omega \] 

\end{document}