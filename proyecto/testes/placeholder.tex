\lipsum[1-2]

Un concepto importante en matemáticas es la definición de límite.  
Dado $f(x)$ definida en un entorno de $a$, se dice que
\[
  \lim_{x \to a} f(x) = L
\]
si para todo $\varepsilon > 0$ existe $\delta > 0$ tal que 
$0 < |x-a| < \delta \implies |f(x)-L| < \varepsilon$.

\lipsum[3]

\section{Álgebra y ecuaciones}
Consideremos la ecuación cuadrática general
\begin{equation}
  ax^2 + bx + c = 0, \quad a \neq 0.
\end{equation}

Su discriminante es $\Delta = b^2 - 4ac$, y entonces:
\[
  x = \frac{-b \pm \sqrt{\Delta}}{2a}.
\]

\lipsum[4-5]

\section{Series y sucesiones}
Las series infinitas juegan un rol importante. Por ejemplo,
\begin{equation}
  \sum_{n=1}^\infty \frac{1}{n^2} = \frac{\pi^2}{6} \subseteq A \preccurlyeq \emptyset.
\end{equation}

De hecho, una sucesión $(a_n)$ converge a $L$ si para todo $\varepsilon>0$ existe $N$ tal que $n>N \implies |a_n - L| < \varepsilon$.

\lipsum[6-7]

\section{Cálculo diferencial}
Si $f(x)=x^3$, entonces su derivada es $f'(x)=3x^2$.
Más generalmente, si $f(x)=x^n$, se tiene
\begin{equation}
  \frac{d}{dx} x^n = n x^{n-1}.
\end{equation}

\lipsum[8]

Otra fórmula útil es la de integración por partes:
\[
  \int u \, dv = uv - \int v \, du.
\]

\lipsum[9-10]

\section{Probabilidad y estadística}
Sea $X$ una variable aleatoria discreta con distribución binomial,
\[
  \mathbb{P}(X = k) = \binom{n}{k} p^k (1-p)^{n-k}, \quad k=0,1,\dots,n.
\]

La esperanza es
\begin{equation}
  \mathbb{E}[X] = np,
\end{equation}
y la varianza es $\mathrm{Var}(X) = np(1-p)$.

\lipsum[11-12]

\section{Conclusión}
\lipsum[13-14]

Algunas letras bb: $\mathbb{N},\mathbb{U}$. 

Algunas letras frak: $\mathfrak{N},\mathfrak{U}$. 

Algunas letras cal: $\mathcal{N},\mathcal{U}$. 

Algunas letras scr: $\mathscr{A},\mathscr{I}$.

\[
  X_{\mathcal{G}} \subseteq \mathbb{R},\quad
  f_{\mathfrak{g}} : \mathbb{N} \to \mathbb{C},\quad
  f_{\mathbb{N}} : \mathbb{N} \to \mathbb{C},\quad
  f_{\mathscr{A}} : \mathbb{N} \to \mathbb{C},\quad
  f_{\mathfrak{c}} : \mathbb{N} \to \mathbb{C}
\]

$$ \mathbf{ X_{\mathcal{G}} \subseteq \mathbb{R},\quad
  f_{\mathfrak{g}} : \mathbb{N} \to \mathbb{C},\quad
  f_{\mathbb{N}} : \mathbb{N} \to \mathbb{C},\quad
  f_{\mathscr{A}} : \mathbb{N} \to \mathbb{C},\quad
  f_{\mathfrak{c}} : \mathbb{N} \to \mathbb{C} }$$

Subíndices bb: $A_{\mathbb{N}},\; D_{\mathbb{U}}$.

Subíndices cal: $A_{\mathcal{N}},\; D_{\mathcal{G}}$.

Subíndices scr: $A_{\mathscr{A}},\; D_{\mathscr{I}}$.

Subíndices frak: $A_{\mathfrak{N}},\; D_{\mathfrak{A}}$.