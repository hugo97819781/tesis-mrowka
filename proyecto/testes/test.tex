\documentclass{scrartcl} % o scrbook/scrreprt

% --- Texto ---
\usepackage{fontspec}
\setmainfont{GFSArtemisia.otf}[
  Path = ../fonts/GFSArtemisia/,
  BoldFont = GFSArtemisiaBold.otf,
  ItalicFont = GFSArtemisiaIt.otf,
  BoldItalicFont = GFSArtemisiaBoldIt.otf
]

% --- Matemáticas ---
\usepackage{amsmath}   % fórmulas (solo macros, no mete fuentes)
\usepackage{amsthm}    % teoremas

\usepackage{unicode-math} % control de tipografía matemática
\setmathfont{STIXTwoMath-Regular.ttf}[
  Path = ../fonts/StixTwo/,
]
\usepackage{mathrsfs}

\begin{document}

Texto normal: Hola mundo.

Ejemplo matemático:
\[
  X_{\mathcal{G}} \subseteq \mathbb{R},\quad
  f_{\mathfrak{g}} : \mathbb{N} \to \mathbb{C},\quad
  f_{\mathbb{N}} : \mathbb{N} \to \mathbb{C},\quad
  f_{\mathscr{A}} : \mathbb{N} \to \mathbb{C},\quad
  f_{\mathfrak{c}} : \mathbb{N} \to \mathbb{C}
\]

Subíndices bb: \(A_{\symbb{N}},\; D_{\symbb{U}}\).

Subíndices cal: \(A_{\symcal{N}},\; D_{\symcal{G}}\).

Subíndices scr: \(A_{\mathscr{A}},\; D_{\mathscr{I}}\).

Subíndices frak: \(A_{\symfrak{N}},\; D_{\symfrak{U}}\).

\end{document}